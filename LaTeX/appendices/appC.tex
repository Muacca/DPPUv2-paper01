%==============================================================================
% Appendix C: Numerical Computation Details
%==============================================================================
\section{Numerical Computation Details}
\label{app:numerical}

This appendix documents the numerical methods, parameters, and algorithms 
used in the parameter scanning and Type classification.

%------------------------------------------------------------------------------
\subsection{Scan Parameters}
\label{app:scan_parameters}

\subsubsection{Parameter ranges}

\begin{table}[H]
\centering
\caption{Parameter ranges for phase diagram generation.}
\begin{tabular}{@{}lcccc@{}}
\toprule
Parameter & Min & Max & Grid points & Spacing \\
\midrule
$V$ & 0.0 & 5.0 & 51 & 0.1 (linear) \\
$\eta$ & $-10.0$ & 5.0 & 151 & 0.099 (linear) \\
$\theta_{\NY}$ & 0.0 & 5.0 & 51 & 0.098 (linear) \\
\bottomrule
\end{tabular}
\end{table}

\subsubsection{Fixed parameters}

\begin{table}[H]
\centering
\caption{Fixed parameters.}
\begin{tabular}{@{}lll@{}}
\toprule
Parameter & Value & Description \\
\midrule
$\kappa$ & 1.0 & Gravitational coupling \\
$L$ & 1.0 & $S^1$ circumference \\
$r_{\min}$ & 0.01 & Lower bound for $r$ search \\
$r_{\max}$ & $10^6$ & Upper bound for $r$ search \\
\bottomrule
\end{tabular}
\end{table}

\subsubsection{Notes on cutoff dependence}
\label{app:robustness}

The Type classification necessarily depends on numerical cutoffs such as the
$r$-search interval $(r_{\min}, r_{\max})$ and the boundary threshold $\delta$
in the ``bound hit'' criterion.
In practice, these cutoffs are chosen wide enough to separate genuinely boundary-attached
profiles (Type~III) from well-formed minima (Type~I/II). 
Points extremely close to a phase boundary are expected to be the most sensitive 
to finite grid resolution.
For this reason, our interpretation of phase diagrams focuses on robust features
(e.g., the existence or absence of wide Type-I regions and the systematic topology dependence),
rather than on single-grid-point fluctuations near boundaries.

\subsubsection{Total evaluations}

For each topology-variant combination:
\begin{equation}
  N_{\text{eval}} = 51 \times 151 \times 51 = 392{,}751 \text{ points}.
\end{equation}

Total for 3 topologies $\times$ 3 variants = 9 combinations:
\begin{equation}
  N_{\text{total}} = 9 \times 392{,}751 = 3{,}534{,}759 \text{ evaluations}.
\end{equation}

%------------------------------------------------------------------------------
\subsection{Extremum Search Algorithm}
\label{app:search_algorithm}

\subsubsection{Primary method: Brent's method}

For finding minima of $\Veff(r)$, we use Brent's method as implemented in 

\texttt{scipy.optimize.minimize\_scalar} with the \texttt{bounded} option.

\begin{table}[H]
\centering
\caption{Brent's method parameters.}
\begin{tabular}{@{}ll@{}}
\toprule
Parameter & Value \\
\midrule
Method & \texttt{bounded} \\
Bounds & $[r_{\min}, r_{\max}] = [0.01, 10^6]$ \\
Tolerance (\texttt{xatol}) & $10^{-8}$ \\
Max iterations & 500 \\
\bottomrule
\end{tabular}
\end{table}

%------------------------------------------------------------------------------
\subsection{Type Classification Algorithm}
\label{app:classification_algorithm}

The Type classification follows the flowchart in Figure~\ref{fig:flowchart}.

\begin{figure}[htbp]
\centering
  \includegraphics[width=0.8\textwidth]{figures/Fig19_Type_classification_flowchart.png}
  \caption{Type classification flowchart.}
\label{fig:flowchart}
\end{figure}

\subsubsection{Step-by-step algorithm}

\begin{enumerate}
  \item \textbf{Extremum search}: Find candidate minimum $r_0$ using Brent's method
  
  \item \textbf{Existence check}: If no minimum found (optimizer returns boundary), 
        classify as Type III
  
  \item \textbf{Bound hit check}: If $r_0 < r_{\min} + \delta$ or 
        $r_0 > r_{\max} - \delta$ (with $\delta = 0.02$), classify as Type III
  
  \item \textbf{Curvature check}: Compute $\dd^2\Veff/\dd r^2|_{r_0}$ numerically. 
        If $\leq 0$, classify as Type III
  
  \item \textbf{Barrier check}: Evaluate gradient near origin:
    \begin{equation}
      s_0 = \left.\frac{\dd\Veff}{\dd r}\right|_{r = r_{\min}}
    \end{equation}
    \begin{itemize}
      \item If $s_0 > 0$: Type I (barrier exists)
      \item If $s_0 \leq 0$: Type II (rolling, no barrier)
    \end{itemize}
\end{enumerate}

\subsubsection{Numerical differentiation}

Derivatives are computed using central differences:
\begin{align}
  \frac{\dd\Veff}{\dd r} &\approx \frac{\Veff(r + h) - \Veff(r - h)}{2h}, \\
  \frac{\dd^2\Veff}{\dd r^2} &\approx \frac{\Veff(r + h) - 2\Veff(r) + \Veff(r - h)}{h^2},
\end{align}
with step size $h = 10^{-5}$.

%------------------------------------------------------------------------------
\subsection{Barrier Height Calculation}
\label{app:barrier_height}

\subsubsection{Type I: Barrier height}

For Type I configurations, the barrier height $\Delta V$ is computed as:
\begin{equation}
  \Delta V = \max_{r \in [r_{\min}, r_0]} \Veff(r) - \Veff(r_0).
\end{equation}

\begin{enumerate}
  \item Generate dense grid in $[r_{\min}, r_0]$: 100 points, logarithmic spacing
  \item Evaluate $\Veff(r)$ at all grid points
  \item Find maximum value $\Veff^{\max}$
  \item Compute $\Delta V = \Veff^{\max} - \Veff(r_0)$
\end{enumerate}

This represents the depth of the potential well from the near-origin value 
to the minimum.

\begin{itemize}
  \item If $\Veff^{\max} \leq \Veff(r_0)$: $\Delta V = 0$ (no barrier, 
        reclassify as Type II)
  \item If $r_0 < 2 r_{\min}$: Insufficient range, $\Delta V$ marked as undefined
\end{itemize}

\subsubsection{Type II: Well depth}

For Type II configurations, the well depth $\Delta V$ is defined as:
\begin{equation}
  \Delta V = \Veff(r_{\min}) - \Veff(r_0).
\end{equation}


%------------------------------------------------------------------------------
\subsection{Numerical Precision and Uncertainties}
\label{app:numerical_precision}

\subsubsection{Grid resolution uncertainty}

Phase boundary positions have uncertainties on the order of the grid spacing:
\begin{itemize}
  \item $\Delta\eta \approx 0.1$
  \item $\Delta V \approx 0.1$
  \item $\Delta\theta_{\NY} \approx 0.1$
\end{itemize}

As shown in Sec.~\ref{sec:numerical}, the analytical boundary conditions 
and numerical scan results agree within this uncertainty.

\subsubsection{Extremum search precision}

The convergence tolerance of Brent's method ($\sim 10^{-8}$) is sufficient 
for the physically meaningful precision of $r_0$ (3--4 significant digits).

\subsubsection{Curvature verification step size}

The step size $h = 10^{-5}$ for numerical differentiation was chosen to 
balance truncation and round-off errors. Results were verified to be 
insensitive to variations in the range $h = 10^{-4}$ to $10^{-6}$.

%------------------------------------------------------------------------------
\subsection{Output Format}
\label{app:output_format}

\subsubsection{CSV output structure}

Results are saved in CSV format with the following columns:

\begin{table}[H]
\centering
\caption{CSV output columns.}
\begin{tabular}{@{}lll@{}}
\toprule
Column & Type & Description \\
\midrule
\texttt{V} & float & Vector torsion parameter \\
\texttt{eta} & float & Axial torsion parameter \\
\texttt{theta\_NY} & float & Nieh--Yan coupling \\
\texttt{type} & int & Classification (1, 2, or 3) \\
\texttt{r0} & float & Stable radius (NaN for Type III) \\
\texttt{Veff\_min} & float & Potential at minimum \\
\texttt{delta\_V} & float & Barrier height (Type I only) \\
\texttt{log10\_r0} & float & $\log_{10}(r_0)$ \\
\texttt{log10\_deltaV} & float & $\log_{10}(\Delta V)$ \\
\texttt{status} & string & Convergence status \\
\bottomrule
\end{tabular}
\end{table}

\subsubsection{File naming convention}

Output files follow the naming pattern:
\begin{center}
\texttt{<topology>\_<variant>\_<mode>\_<timestamp>.csv}
\end{center}

Example: \texttt{S3S1\_FULL\_MX\_20241214\_135031.csv}

\subsubsection{Code access and execution}

For code access, installation instructions, and detailed execution procedures, 
see Appendix~\ref{app:access}.

%------------------------------------------------------------------------------
\subsection{Error Handling}
\label{app:error_handling}

\subsubsection{Convergence failures}

If the optimizer fails to converge within the maximum iterations:
\begin{itemize}
  \item Log warning with parameter values
  \item Attempt grid search fallback
  \item If still unsuccessful, classify as Type III with status \texttt{FAILED}
\end{itemize}

\subsubsection{Numerical overflow}

For extreme parameter values where $\Veff(r)$ overflows:
\begin{itemize}
  \item Detected by checking for \texttt{inf} or \texttt{nan} values
  \item Point classified as Type III with status \texttt{OVERFLOW}
\end{itemize}

\subsubsection{Statistics}

Typical failure rates in production runs:
\begin{itemize}
  \item Convergence failures: $< 0.1\%$
  \item Overflow errors: $< 0.01\%$
  \item Total valid classifications: $> 99.8\%$
\end{itemize}
