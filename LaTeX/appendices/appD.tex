%==============================================================================
% Appendix D: Visualization Tools
%==============================================================================
\section{Visualization Tools}
\label{app:visualization}

This appendix documents the visualization tools developed for analyzing 
and presenting the results of this paper. For code access and installation 
instructions, see Appendix~\ref{app:access}.

%------------------------------------------------------------------------------
\subsection{Overview of Visualization Pipeline}
\label{app:viz_overview}

The visualization pipeline consists of the following components:

\begin{enumerate}
  \item \textbf{Phase map generator}: Creates 2D phase diagrams on $(V, \eta)$ plane
  \item \textbf{Phase matrix generator}: Creates comparison grids across 
        topologies and variants
  \item \textbf{Potential plotter}: Generates $\Veff(r)$ curves for 
        representative points
  \item \textbf{Interactive viewer}: Jupyter-based tool for exploration
\end{enumerate}

%------------------------------------------------------------------------------
\subsection{Phase Diagram Generation}
\label{app:phase_diagram}

\subsubsection{Color mapping}

Phase diagrams use the following visual encoding:

\begin{table}[H]
\centering
\caption{Visual encoding for phase diagrams.}
\begin{tabular}{@{}lll@{}}
\toprule
Element & Encoding & Description \\
\midrule
Type I region & Solid color & Metastable with barrier \\
Type II region & Hatched color & Rolling, no barrier \\
Type III region & White & Unstable \\
Color gradient & Purple $\to$ Yellow & $\log_{10}(r_0)$ value \\
White contours & Solid lines & $\log_{10}(\Delta V)$ levels \\
\bottomrule
\end{tabular}
\end{table}

\subsubsection{Colormap specification}

\begin{itemize}
  \item Base colormap: \texttt{viridis} (perceptually uniform)
  \item Range: $\log_{10}(r_0) \in [-2, 5]$
  \item Out-of-range handling: Clipped to bounds
\end{itemize}

\subsubsection{Hatching for Type II}

Type II regions are distinguished from Type I by diagonal hatching:
\begin{itemize}
  \item Pattern: $45^\circ$ diagonal lines
  \item Density: 4 lines per unit
  \item Color: Semi-transparent gray overlay
\end{itemize}

\subsubsection{Implementation}

\begin{lstlisting}[language=Python, basicstyle=\ttfamily\footnotesize]
import matplotlib.pyplot as plt
import numpy as np

def plot_phase_diagram(V_grid, eta_grid, types, r0_values, 
                       delta_V_values, theta_NY):
    fig, ax = plt.subplots(figsize=(8, 6))
    
    # Create masked arrays for each type
    type1_mask = (types == 1)
    type2_mask = (types == 2)
    
    # Plot Type I (solid)
    log_r0 = np.log10(r0_values)
    im = ax.pcolormesh(V_grid, eta_grid, log_r0, 
                       cmap='viridis', vmin=-2, vmax=5,
                       shading='auto')
    
    # Overlay hatching for Type II
    ax.contourf(V_grid, eta_grid, type2_mask, 
                levels=[0.5, 1.5], hatches=['//'],
                colors='none', alpha=0.3)
    
    # Contours for barrier height
    cs = ax.contour(V_grid, eta_grid, 
                    np.log10(delta_V_values),
                    levels=[1, 2, 3, 4], colors='white',
                    linewidths=0.8)
    ax.clabel(cs, inline=True, fontsize=8)
    
    # Labels and colorbar
    ax.set_xlabel(r'$V$')
    ax.set_ylabel(r'$\eta$')
    ax.set_title(f'Phase diagram ($\\theta_{{NY}} = {theta_NY}$)')
    plt.colorbar(im, ax=ax, label=r'$\log_{10}(r_0)$')
    
    return fig, ax
\end{lstlisting}

%------------------------------------------------------------------------------
\subsection{Phase Matrix Generation}
\label{app:phase_matrix}

Phase matrices display 3 topologies $\times$ 3 variants in a single figure 
for comparison.

\subsubsection{Layout}

\begin{itemize}
  \item Grid: $3 \times 3$ subplots
  \item Rows: Topologies ($\Sthree$, $\Tthree$, $\Nilthree$)
  \item Columns: NY variants (FULL, TT, REE)
  \item Shared colorbar: Single colorbar for entire figure
\end{itemize}

\subsubsection{Filename convention}

Phase matrix files are named:
\begin{center}
\texttt{fig12\_phase\_matrix\_XXXX.png}
\end{center}
where \texttt{XXXX} is Serial number.

%------------------------------------------------------------------------------
\subsection{Potential Curve Plotting}
\label{app:potential_curves}

\subsubsection{Standard plot format}

For representative points, $\Veff(r)$ curves are plotted with:

\begin{itemize}
  \item $x$-axis: $r$ (linear or logarithmic scale)
  \item $y$-axis: $\Veff(r)$ (linear scale)
  \item Markers: Minimum position $r_0$ (if exists)
  \item Annotations: Type classification, $\Delta V$ value
\end{itemize}

\subsubsection{Multi-panel comparison}

For phase-potential correspondence figures (e.g., Figure~\ref{fig:phase_potential_S3}):

\begin{itemize}
  \item Left panel: Phase diagram with marked points
  \item Right panels: $\Veff(r)$ curves for each marked point
  \item Color coding: Consistent colors between diagram markers and curves
\end{itemize}

%------------------------------------------------------------------------------
\subsection{Interactive Viewer}
\label{app:interactive}

\subsubsection{Jupyter notebook interface}

The interactive viewer (\texttt{DPPUv2\_interactive\_viewer\_v3.py}) provides:

\paragraph{Configuration selection}
\begin{itemize}
  \item Topology switching: $\Sthree \times \Sone$, $\Tthree \times \Sone$, $\Nilthree \times \Sone$
  \item Torsion mode: MX (mixed), AX (axial only), VT (vector only)
  \item NY variant: FULL, TT, REE
  \item $\Tthree$ anisotropy parameters: $\alpha = R_2/R_1$, $\beta = R_3/R_1$
\end{itemize}

\paragraph{Phase diagram display}
\begin{itemize}
  \item Type I/II/III classification on $(V, \eta)$ plane
  \item Real-time update via $\theta_{\NY}$ slider
\end{itemize}

\paragraph{Potential plotting}
\begin{itemize}
  \item Click-to-select points on phase diagram (up to 3 simultaneous points)
  \item $\Veff(r)$ shape comparison
  \item Axis scale switching (linear / log / symlog)
  \item Dynamic range adjustment
\end{itemize}

\subsubsection{Usage}

\begin{lstlisting}[language=Python, basicstyle=\ttfamily\footnotesize]
%matplotlib widget
from DPPUv2_interactive_viewer_v3 import DPPUv2InteractiveViewer

# Initialize and launch viewer
viewer = DPPUv2InteractiveViewer()
viewer.display()

# Optional: Customize slider ranges before instantiation
DPPUv2InteractiveViewer.SLIDER_V_MAX_MAX = 30.0
DPPUv2InteractiveViewer.SLIDER_ETA_MIN_MIN = -30.0
viewer = DPPUv2InteractiveViewer()
viewer.display()
\end{lstlisting}

\subsubsection{Operation instructions}

\paragraph{Generating phase diagrams}
\begin{enumerate}
  \item Select topology, mode, and variant in the Configuration panel
  \item Adjust the $\theta_{\NY}$ slider
  \item Click the ``Draw Phase Diagram'' button
\end{enumerate}

\paragraph{Comparing potentials}
\begin{enumerate}
  \item Click any point on the phase diagram (registered as Point 1)
  \item Select ``Point 2'' in Point Selection and click another point
  \item Up to 3 points of $\Veff(r)$ are displayed overlaid in the right panel
\end{enumerate}

\paragraph{Exploring $\Tthree$ anisotropy}
\begin{enumerate}
  \item Select ``T3'' as topology
  \item The Anisotropy panel becomes visible
  \item Adjust $\alpha$ and $\beta$ sliders to set anisotropy ratios
  \item Redraw the phase diagram
\end{enumerate}

\subsubsection{Widget layout}

\begin{table}[H]
\centering
\caption{Interactive viewer widgets.}
\begin{tabular}{@{}lll@{}}
\toprule
Widget & Type & Range/Options \\
\midrule
$V$ slider & FloatSlider & $[0, V_{\max}]$, step 0.1 \\
$\eta$ slider & FloatSlider & $[-10, 5]$, step 0.1 \\
$\theta_{\NY}$ slider & FloatSlider & $[0, 5]$, step 0.1 \\
Topology dropdown & Dropdown & S3, T3, Nil3 \\
Variant dropdown & Dropdown & FULL, TT, REE \\
Scale toggle & ToggleButton & Linear / Log \\
\bottomrule
\end{tabular}
\end{table}

%------------------------------------------------------------------------------
\subsection{Figure Export Settings}
\label{app:export}

\subsubsection{Resolution and format}

\begin{table}[H]
\centering
\caption{Figure export settings.}
\begin{tabular}{@{}lll@{}}
\toprule
Setting & Value & Notes \\
\midrule
Format & PNG & For raster figures \\
DPI & 300 & Publication quality \\
Figure size & Variable & Typically 8$\times$6 inches \\
Font & Computer Modern & LaTeX compatible \\
Font size & 12pt & Axis labels \\
\bottomrule
\end{tabular}
\end{table}

\subsubsection{LaTeX integration}

Figures are generated with LaTeX rendering enabled:

\begin{lstlisting}[language=Python, basicstyle=\ttfamily\footnotesize]
import matplotlib.pyplot as plt

plt.rcParams.update({
    'text.usetex': True,
    'font.family': 'serif',
    'font.serif': ['Computer Modern'],
    'font.size': 12,
    'axes.labelsize': 14,
    'legend.fontsize': 10,
})
\end{lstlisting}

%------------------------------------------------------------------------------
\subsection{List of Generated Figures}
\label{app:figure_list}

\begin{table}[H]
\centering
\caption{List of figures in this paper.}
\begin{tabular}{@{}lll@{}}
\toprule
Figure & Filename & Description \\
\midrule
\ref{fig:schematic} & \texttt{fig01\_Schematic\_Classification.png} & Type I/II/III schematic \\
\ref{fig:variant_S3} & \texttt{fig03\_variant\_comparison\_S3.png} & $\Sthree$ variant comparison \\
\ref{fig:decomposition_S3} & \texttt{fig04\_decomposition\_S3.png} & $\Sthree$ potential decomposition \\
\ref{fig:variant_T3} & \texttt{fig05\_variant\_comparison\_T3.png} & $\Tthree$ variant comparison \\
\ref{fig:decomposition_T3} & \texttt{fig06\_decomposition\_T3.png} & $\Tthree$ potential decomposition \\
\ref{fig:variant_Nil3} & \texttt{fig07\_variant\_comparison\_Nil3.png} & $\Nilthree$ variant comparison \\
\ref{fig:decomposition_Nil3} & \texttt{fig08\_decomposition\_Nil3.png} & $\Nilthree$ potential decomposition \\
\ref{fig:phase_S3} & \texttt{fig09\_phase\_diagram\_S3-FULL.png} & $\Sthree$ phase diagram \\
\ref{fig:phase_T3} & \texttt{fig10\_phase\_diagram\_T3-FULL.png} & $\Tthree$ phase diagram \\
\ref{fig:phase_Nil3} & \texttt{fig11\_phase\_diagram\_Nil3-FULL.png} & $\Nilthree$ phase diagram \\
\ref{fig:phase_matrix} & \texttt{fig12\_phase\_matrix\_0010.png} & Phase matrix ($\theta_{\NY}=1$) \\
\ref{fig:phase_potential_S3} & \texttt{fig13\_Phase\_Potential\_S3.png} & $\Sthree$ phase-potential \\
\ref{fig:phase_potential_T3} & \texttt{fig14\_Phase\_Potential\_T3.png} & $\Tthree$ phase-potential \\
\ref{fig:phase_potential_Nil3} & \texttt{fig15\_Phase\_Potential\_Nil3.png} & $\Nilthree$ phase-potential \\
\ref{fig:scaling_S3} & \texttt{fig16\_Scaling\_laws\_S3.png} & $\Sthree$ scaling laws \\
\ref{fig:scaling_T3} & \texttt{fig17\_Scaling\_laws\_T3.png} & $\Tthree$ scaling laws \\
\ref{fig:scaling_Nil3} & \texttt{fig18\_Scaling\_laws\_Nil3.png} & $\Nilthree$ scaling laws \\
\bottomrule
\end{tabular}
\end{table}
