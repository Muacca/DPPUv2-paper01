%==============================================================================
% Appendix E: Volume Rescaling Invariance in $T^3 \times S^1$
%==============================================================================
\clearpage
\section{Volume Rescaling Invariance in $\Tthree \times \Sone$}
\label{app:rescaling}

This appendix discusses the volume rescaling properties of the effective 
potential for $\Tthree \times \Sone$ and justifies the isotropic setting 
$R_1 = R_2 = R_3 = r$ adopted in the main text.

%------------------------------------------------------------------------------
\subsection{General Anisotropic Setting}
\label{app:anisotropic}

\subsubsection{Parameterization}

For $\Tthree$, the three circumferences $R_1$, $R_2$, $R_3$ are independent 
parameters. The coframe is:
\begin{equation}
  e^0 = R_1 \, \dd x^1, \quad e^1 = R_2 \, \dd x^2, \quad 
  e^2 = R_3 \, \dd x^3, \quad e^3 = L \, \dd\tau,
\end{equation}
where $x^i \in [0, 2\pi)$ are periodic coordinates.

\subsubsection{Volume element}

The total volume is:
\begin{equation}
  \mathrm{Vol}(\Tthree \times \Sone) = (2\pi)^4 L R_1 R_2 R_3.
\end{equation}

\subsubsection{Effective potential (general form)}

With all three radii independent, the effective potential for the FULL 
variant is:
\begin{equation}
  \Veff^{(\Tthree)}(R_1, R_2, R_3) = \frac{(2\pi)^4 L}{3\kappa^2} 
  \left[ V^2 R_1 R_2 R_3 + 6V\eta\kappa^2\theta_{\NY} R_2 R_3 
         + \frac{9\eta^2 R_2 R_3}{R_1} \right].
  \label{eq:Veff_T3_general}
\end{equation}

Note the asymmetric dependence on $R_1$ versus $R_2$, $R_3$: this arises 
because the torsion ansatz singles out the $e^0$ direction for the axial 
component normalization.

%------------------------------------------------------------------------------
\subsection{Rescaling Properties}
\label{app:rescaling_properties}

\subsubsection{Uniform rescaling}

Under uniform rescaling $R_i \to \lambda R_i$ (for all $i = 1, 2, 3$):
\begin{equation}
  \Veff \to \lambda^3 \cdot \frac{(2\pi)^4 L}{3\kappa^2} 
  \left[ V^2 R_1 R_2 R_3 + \frac{6V\eta\kappa^2\theta_{\NY}}{\lambda} R_2 R_3 
         + \frac{9\eta^2}{\lambda^2} \frac{R_2 R_3}{R_1} \right].
\end{equation}

The three terms scale as $\lambda^3$, $\lambda^2$, and $\lambda^1$ respectively. 
This non-uniform scaling means that the \textit{shape} of the potential 
changes under rescaling, but the \textit{Type classification} (existence 
of minima, barriers) is preserved.

\subsubsection{Anisotropic rescaling relative to $R_1$}

Consider the parameterization $R_2 = \alpha R_1$ and $R_3 = \beta R_1$, 
where $\alpha$ and $\beta$ are dimensionless anisotropy ratios. 
Substituting into Eq.~\eqref{eq:Veff_T3_general}:
\begin{equation}
  \frac{R_2 R_3}{R_1} = \alpha \beta R_1,
\end{equation}
which yields the key relation:
\begin{equation}
  \Veff(\alpha, \beta) = \alpha \beta \, \Veff(1, 1).
  \label{eq:Veff_rescaling}
\end{equation}

This demonstrates that anisotropic rescaling merely multiplies $\Veff$ by 
an overall factor $\alpha\beta$, leaving the extremum position with respect 
to $R_1$ and the sign of $\partial_{R_1}\Veff$ unchanged. Consequently, 
the \textbf{phase boundary positions are invariant} under this rescaling, 
while the potential depth scales as $\alpha\beta$.

\subsubsection{Shape-preserving property}

Define dimensionless ratios:
\begin{equation}
  \rho_1 = \frac{R_1}{R_{\text{ref}}}, \quad 
  \rho_2 = \frac{R_2}{R_{\text{ref}}}, \quad 
  \rho_3 = \frac{R_3}{R_{\text{ref}}},
\end{equation}
where $R_{\text{ref}}$ is a reference scale.

The Type classification depends only on the ratios $\rho_i$ and the 
dimensionless combinations:
\begin{equation}
  \tilde{V} = V R_{\text{ref}}, \quad 
  \tilde{\eta} = \eta, \quad 
  \tilde{\theta} = \theta_{\NY}.
\end{equation}

This means that phase diagrams are independent of the overall scale 
$R_{\text{ref}}$, justifying the use of $\kappa = L = 1$ units.

%------------------------------------------------------------------------------
\subsection{Isotropic Reduction}
\label{app:isotropic}

\subsubsection{Motivation}

For comparison with $\Sthree$ and $\Nilthree$ (which have single scale 
parameter $r$), we adopt the isotropic setting:
\begin{equation}
  R_1 = R_2 = R_3 = r.
\end{equation}

This reduces the 3-parameter family to a 1-parameter family, enabling 
direct comparison of phase structures across topologies.

\subsubsection{Resulting potential}

Substituting $R_1 = R_2 = R_3 = r$ into Eq.~\eqref{eq:Veff_T3_general}:
\begin{equation}
  \Veff^{(\Tthree,\text{iso})}(r) = \frac{(2\pi)^4 L}{3\kappa^2} 
  \left[ V^2 r^3 + 6V\eta\kappa^2\theta_{\NY} r^2 + 9\eta^2 r \right].
\end{equation}

This has the same $r \cdot (Ar^2 + Br + C)$ structure as $\Sthree$ and 
$\Nilthree$, with:
\begin{equation}
  A = V^2, \quad B = 6V\eta\kappa^2\theta_{\NY}, \quad C = 9\eta^2.
\end{equation}

\subsubsection{Comparison with anisotropic case}

The key difference from the anisotropic case:

\begin{itemize}
  \item \textbf{Anisotropic}: Three independent variables $(R_1, R_2, R_3)$; 
        minima form a 2-dimensional surface in parameter space
  \item \textbf{Isotropic}: Single variable $r$; minima are isolated points 
        on the $r$-axis
\end{itemize}

The isotropic setting captures the ``diagonal slice'' of the full 
3-dimensional configuration space.

%------------------------------------------------------------------------------
\subsection{Validity of Isotropic Approximation}
\label{app:validity}

\subsubsection{Stability analysis}

For the isotropic minimum $r = r_0$ to be stable against anisotropic 
perturbations, we require:
\begin{equation}
  \left.\frac{\partial^2 \Veff}{\partial R_i \partial R_j}\right|_{R_1=R_2=R_3=r_0} 
  \quad \text{positive definite}.
\end{equation}

\subsubsection{Hessian computation}

The Hessian matrix at the isotropic point has the structure:
\begin{equation}
  H_{ij} = \begin{pmatrix}
    H_{11} & H_{12} & H_{13} \\
    H_{12} & H_{22} & H_{23} \\
    H_{13} & H_{23} & H_{33}
  \end{pmatrix},
\end{equation}
where the specific values depend on $(V, \eta, \theta_{\NY}, r_0)$.

For the parameter ranges studied in this paper, numerical evaluation shows 
that the Hessian is positive definite at Type I minima, confirming stability 
of the isotropic configuration.

\subsubsection{Physical interpretation}

The isotropic setting corresponds to a ``cubic torus'' where all three 
directions are equivalent. Perturbations toward anisotropy (e.g., elongation 
in one direction) increase the potential, so the system returns to the 
isotropic state.

This justifies using the isotropic reduction for phase classification, 
as it captures the stable configurations.

%------------------------------------------------------------------------------
\subsection{Alternative Reduction Scheme (Not Adopted)}
\label{app:alternatives}

\subsubsection{Alternative: Fix $R_2 = R_3 = 1$}

During the development of this work, an alternative approach was considered 
that fixes two dimensions and varies only $R_1 = r$:

\begin{equation}
  \Veff^{(\text{alt})}(r) = \frac{(2\pi)^4 L}{3\kappa^2} 
  \left[ V^2 r + 6V\eta\kappa^2\theta_{\NY} + \frac{9\eta^2}{r} \right].
\end{equation}

This gives a different functional form ($r + \text{const} + 1/r$) that 
does not match $\Sthree$ or $\Nilthree$.

\subsubsection{Comparison of phase structures}

\begin{table}[H]
\centering
\caption{Comparison of $\Tthree$ reduction schemes.}
\begin{tabular}{@{}lll@{}}
\toprule
Scheme & Potential structure & Compatible with $\Sthree$/$\Nilthree$? \\
\midrule
Isotropic ($R_1=R_2=R_3=r$) & $r^3 + r^2 + r$ & Yes \\
Fixed dimensions ($R_2=R_3=1$) & $r + \text{const} + 1/r$ & No \\
\bottomrule
\end{tabular}
\end{table}

The isotropic scheme is preferred for cross-topology comparison because 
it produces the same polynomial structure across all three test beds.

%------------------------------------------------------------------------------
\subsection{Summary}
\label{app:rescaling_summary}

\begin{enumerate}
  \item The isotropic setting $R_1 = R_2 = R_3 = r$ is adopted for $\Tthree$ 
        to enable fair comparison with $\Sthree$ and $\Nilthree$
  
  \item This setting produces the same $r \cdot (Ar^2 + Br + C)$ structure 
        as the other topologies
  
  \item Stability analysis confirms that isotropic minima are stable against 
        anisotropic perturbations in the parameter ranges studied
  
  \item The Type classification is independent of overall scale, depending 
        only on dimensionless parameter combinations
\end{enumerate}
