%==============================================================================
% Topology-Dependent Phase Classification of Effective Potentials
% in Einstein--Cartan + Nieh--Yan Minisuperspace
%==============================================================================
\documentclass[12pt,a4paper]{article}

%------------------------------------------------------------------------------
% Packages
%------------------------------------------------------------------------------
% Math
\usepackage{amsmath,amssymb,amsfonts}
\usepackage{mathtools}
\usepackage{bm}

% Graphics and figures
\usepackage{graphicx}
\usepackage{float}
\usepackage{subcaption}

% Tables
\usepackage{booktabs}
\usepackage{tabularx}
\usepackage{array}
\usepackage{multirow}

% TikZ for diagrams
\usepackage{tikz}
\usetikzlibrary{shapes.geometric, arrows.meta, positioning, calc, fit, backgrounds}

% Layout and formatting
\usepackage[margin=2.5cm]{geometry}
\usepackage{setspace}
\onehalfspacing

% References and links
\usepackage[colorlinks=true,linkcolor=blue,citecolor=blue,urlcolor=blue]{hyperref}
\usepackage{cleveref}

% Code listings (for appendices)
\usepackage{listings}
\lstset{
  basicstyle=\ttfamily\small,
  breaklines=true,
  frame=single,
  language=Python
}

% section break
\usepackage{titlesec}
\newcommand{\sectionbreak}{\clearpage}

% Misc
\usepackage{enumitem}
\usepackage{xcolor}

%------------------------------------------------------------------------------
% Custom commands
%------------------------------------------------------------------------------
% Math shortcuts
\newcommand{\dd}{\mathrm{d}}
\newcommand{\Veff}{V_{\mathrm{eff}}}
\newcommand{\Ric}{R}
\newcommand{\LC}{\mathrm{LC}}
\newcommand{\EC}{\mathrm{EC}}
\newcommand{\NY}{\mathrm{NY}}
\newcommand{\TT}{\mathrm{TT}}
\newcommand{\REE}{\mathrm{REE}}
\newcommand{\FULL}{\mathrm{FULL}}

% Topology notation
\newcommand{\Sthree}{S^3}
\newcommand{\Tthree}{T^3}
\newcommand{\Nilthree}{\mathrm{Nil}^3}
\newcommand{\Sone}{S^1}

% Differential forms
\newcommand{\wedgep}{\wedge}

%------------------------------------------------------------------------------
% Title and authors
%------------------------------------------------------------------------------
\title{Topology-Dependent Phase Classification of Effective Potentials\\
in Einstein--Cartan + Nieh--Yan Minisuperspace}

\author{Muacca}

\date{\today}

%==============================================================================
\begin{document}
%==============================================================================

\maketitle

%------------------------------------------------------------------------------
% Abstract
%------------------------------------------------------------------------------
\begin{abstract}
We study Einstein--Cartan (EC) gravity supplemented with the Nieh--Yan (NY) term 
in a Euclidean--signature minisuperspace framework, and classify the resulting 
effective potential into distinct ``phases.'' Based on the presence or absence 
of local minima and barriers in the effective potential, we define three types: 
(I) metastable well with barrier, (II) barrier-free rolling, and 
(III) unstable/boundary-attached configurations.

While the Nieh--Yan density (a 4-form) can be written as an exact derivative, 
it is geometrically defined through the coframe and torsion. To facilitate 
meaningful comparison, we evaluate topology dependence under a unified ansatz. 
We adopt spatial sections that admit left-invariant coframes, enabling 
systematic description within a common minisuperspace framework.

As concrete test beds, we consider: (i) $\Sthree$, (ii) $\Tthree$, and 
(iii) $\Nilthree$. For the NY term, we focus on the complete form (FULL) 
as the primary object, while also examining TT (torsion-torsion component only) 
and REE (the remaining component) as diagnostic comparisons to disentangle 
the contributions within FULL.

Our investigation addresses two main questions: (1) How does the phase 
(Type I/II/III) of the effective potential depend on topology? 
(2) Can we identify, through numerical scanning, critical conditions 
corresponding to well formation/disappearance and barrier collapse, 
and organize their geometric dependence?

Our scanning results suggest the following:
\begin{itemize}
  \item The phase structure of the effective potential varies systematically 
        with topology.
  \item Within the NY term, some contributions exhibit relative insensitivity 
        to topology, while others appear selectively depending on geometric conditions.
\end{itemize}
\end{abstract}

%------------------------------------------------------------------------------
% Main text
%------------------------------------------------------------------------------
%==============================================================================
% Section 1: Introduction
%==============================================================================
\section{Introduction}
\label{sec:introduction}

%------------------------------------------------------------------------------
\subsection{Background: Correspondence Between Geometric Input and Phase Structure of Effective Potentials}
\label{sec:background}

In theories where the coframe and connection are treated as independent variables, 
curvature and torsion are introduced on equal footing as geometric degrees of freedom. 
In particular, Einstein--Cartan (EC) theory and its extensions allow torsion and 
topological densities to appear in the action, potentially making the structure 
of the reduced effective theory sensitive to geometric input.

Meanwhile, minisuperspace reduction based on symmetry assumptions provides a 
powerful procedure for mapping infinite-dimensional field theories to 
finite-dimensional systems. However, even within the same reduction framework, 
how differences in spatial topology or structure constants of homogeneous spaces 
affect the shape of the effective potential $\Veff(r)$---including the presence 
or absence of minima and barriers, and transitions to rolling behavior---has not 
been systematically organized.

For this reason, it is important to classify and visualize, in a reproducible manner, 
the correspondence between geometric input and the phase structure of the reduced 
effective potential. This work organizes this correspondence using representative 
homogeneous spaces as test beds, presenting the results as phase diagrams and 
representative points.

%------------------------------------------------------------------------------
\subsection{Our Approach: Treating EC+NY as a ``Phase Classification Problem for Effective Potentials''}
\label{sec:approach}

In this work, we take EC+NY (Einstein--Cartan gravity with the Nieh--Yan term) as 
our object of study and classify the phases of the effective potential in a 
Euclidean-signature minisuperspace from the perspective of topology dependence. 
In EC theory, torsion is naturally introduced as geometry, and while the NY density 
(a 4-form) can be written as an exact derivative, it is geometrically defined 
through the coframe and torsion. Therefore, by comparing how the effective potential 
changes when varying the spatial topology under the same minisuperspace ansatz, 
we diagnose the phase structure provided by EC+NY from the viewpoint of geometric input.

Although EC+NY originates as a gravitational theory, the focus of this paper is 
on systematically organizing the influence of geometric input (topology, structure 
constants, torsion, and NY contributions) on the phase structure after reduction.

%------------------------------------------------------------------------------
\subsection{Scope: Three Homogeneous Spaces as Test Beds}
\label{sec:scope}

To enable comparison from the perspective of topology dependence, we adopt spatial 
sections that possess homogeneity and can be uniformly described using left-invariant 
coframes. Specifically, as three representative examples with contrasting properties 
in structure constants and curvature, we adopt:
\begin{enumerate}[label=(\roman*)]
  \item $\Sthree$ (SU(2); isotropic positive curvature),
  \item $\Tthree$ (flat),
  \item $\Nilthree$ (compact quotient of the Heisenberg group; anisotropic geometry).
\end{enumerate}
These serve as a minimal set for comparing how ``geometric data (structure constants, 
curvature characteristics)'' affect the phase structure of the effective potential 
under the same reduction procedure.

%------------------------------------------------------------------------------
\subsection{Deliverables: Type Classification, Critical Conditions, Phase Diagrams, and Representative Points}
\label{sec:deliverables}

The central output of this work is an operational classification based on the 
shape of $\Veff(r)$. Specifically, we introduce three types:
\begin{itemize}
  \item[(I)] Metastable well with barrier,
  \item[(II)] Barrier-free rolling,
  \item[(III)] Unstable/boundary-attached,
\end{itemize}
and visualize how these are arranged in parameter space for each topology as 
phase diagrams. Furthermore, we identify critical conditions corresponding to 
well formation/disappearance and barrier collapse through numerical scanning, 
and organize their geometric dependence.

For the NY term, we take the complete form (FULL) as the primary object, while 
also using TT and REE as diagnostic comparisons to disentangle the contributions 
(definitions in Sec.~\ref{sec:setup}). Note that TT/REE are not proposed as 
independent fundamental theories but serve as auxiliary comparisons for 
understanding the effects of FULL.

Through this approach, we provide a reproducible diagnostic pipeline that returns 
the phase (Type I/II/III) and critical boundaries of $\Veff(r)$ for given geometric 
input (topology, parameters). The resulting phase diagrams, representative points, 
and critical conditions can be used as foundational data for explicitly incorporating 
topology dependence in EC+NY minisuperspace analysis.

%------------------------------------------------------------------------------
\subsection{Paper Organization}
\label{sec:organization}

The organization of this paper is as follows. Section~\ref{sec:setup} summarizes 
the formulation of EC+NY and minisuperspace reduction, as well as the definitions 
and numerical criteria for Type I/II/III. Section~\ref{sec:reductions} presents 
the reduction results for $\Sthree$, $\Tthree$, and $\Nilthree$, showing the form 
of the effective potential. Section~\ref{sec:numerical} presents numerical scanning 
results as phase diagrams. Section~\ref{sec:mechanisms} interprets the phase 
diagram features based on the analytical structure of the effective potential. 
Section~\ref{sec:representative} summarizes representative points and stability 
metrics (minimum position, barrier height, etc.). Section~\ref{sec:interpretation} 
discusses geometric interpretation and the utility of this framework. 
Section~\ref{sec:conclusions} presents conclusions and future prospects.

%------------------------------------------------------------------------------
\subsection{Scope and Limitations}
\label{sec:limitations}

This work focuses on phase classification of effective potentials based on 
minisuperspace reduction. The following items are outside the scope of this paper 
and are discussed as future research directions in Sec.~\ref{sec:conclusions}.

\paragraph{Dynamical evolution:}
We do not treat time evolution including Friedmann-type constraint equations, 
nor quantitative calculations of tunneling rates via WKB approximation. 
The distinction between Type I/II is an operational classification based on 
the shape of $\Veff(r)$ and does not provide actual transition probabilities.

\paragraph{Quantum corrections:}
We do not consider quantum corrections at one-loop or higher, nor renormalization 
group running of coupling constants. The results of this work are based on the 
classical (tree-level) effective potential.

\paragraph{Matter field coupling:}
We do not include interactions with matter fields such as spinor or scalar fields. 
We focus on the geometric structure of pure EC+NY theory.

\paragraph{Inhomogeneous perturbations:}
We do not analyze inhomogeneous fluctuations beyond the minisuperspace ansatz 
(spatial homogeneity). Consequently, our results are limited to the homogeneous sector.

\paragraph{Analytic continuation to Lorentzian signature:}
This work performs calculations in Euclidean signature $(+,+,+,+)$. 
Wick rotation to Lorentzian signature $(-,+,+,+)$ and interpretation in 
real-time cosmology are not discussed in detail in this paper.
  % Introduction
%==============================================================================
% Section 2: Setup and Conventions
%==============================================================================
\section{Setup and Conventions}
\label{sec:setup}

This section summarizes the formulation of EC+NY used in this paper, the 
premises of minisuperspace reduction, parameters and notation, and the 
classification conventions (Type I/II/III) for numerical scanning. The main 
results of this paper (phase diagrams, classification tables, representative 
points) are obtained based on the conventions defined here.

%------------------------------------------------------------------------------
\subsection{Basic Variables and Notation for EC+NY}
\label{sec:basic_variables}

This paper treats Einstein--Cartan theory (with Euclidean signature) in the 
first-order formalism (coframe/connection). The basic variables are the 
coframe (tetrad 1-form) $\{e^a\}$ and the independent spin connection $\omega^{ab}$. 
The torsion 2-form $T^a$ and curvature 2-form $R^{ab}$ are defined as
\begin{align}
  T^a &:= \dd e^a + \omega^a{}_b \wedgep e^b, \\
  R^{ab} &:= \dd \omega^{ab} + \omega^a{}_c \wedgep \omega^{cb}.
\end{align}

\paragraph{EC connection and contortion:}
The Einstein--Cartan connection $\Gamma^a_{\EC,bc}$ is expressed as the sum of 
the Levi--Civita connection $\Gamma^a_{\LC,bc}$ and the contortion $K^a{}_{bc}$:
\begin{equation}
  \Gamma^a_{\EC,bc} = \Gamma^a_{\LC,bc} + K^a{}_{bc}.
\end{equation}
The contortion is determined from torsion as follows (following the convention 
of Hehl et al.~\cite{Hehl1976}):
\begin{equation}
  K_{abc} = \frac{1}{2}\left( T_{abc} + T_{bca} - T_{cab} \right).
  \label{eq:contortion}
\end{equation}
Here all indices are written in lowered form.

\paragraph{Sign conventions:}
This paper adopts the following conventions:
\begin{itemize}
  \item Frame metric: $\eta_{ab} = \mathrm{diag}(+1, +1, +1, +1)$ (Euclidean signature)
  \item Riemann tensor (antisymmetric in 3rd and 4th indices):
    \begin{equation}
      R^{a}{}_{bcd} = \partial_c \Gamma^{a}{}_{bd} - \partial_d \Gamma^{a}{}_{bc} 
                    + \Gamma^{a}{}_{ec}\Gamma^{e}{}_{bd} - \Gamma^{a}{}_{ed}\Gamma^{e}{}_{bc}
    \end{equation}
  \item Contortion: Following Hehl et al.~\cite{Hehl1976}, 
        $K_{abc} = \frac{1}{2}(T_{abc} + T_{bca} - T_{cab})$
  \item Levi--Civita symbol: $\varepsilon_{0123} = +1$
\end{itemize}

\noindent
\textbf{Remark on index labels.}
All internal indices $a,b,\ldots$ are \emph{Euclidean} frame labels.
In particular, the label $a=0$ does \emph{not} denote Lorentzian time in this paper;
it is simply one of the orthonormal directions on the homogeneous spatial section.
The $S^1$ direction is labeled $a=3$.

Hereafter, the wedge product is written as $\wedgep$. Details of index conventions 
(distinction between internal indices $a,b,\ldots$ and coordinate indices) and 
supplementary derivations are summarized in Appendix~\ref{app:theory}.

%------------------------------------------------------------------------------
\subsection{Nieh--Yan Density and Nieh--Yan Term}
\label{sec:NY_density}

The Nieh--Yan density (4-form) $N$ is a geometric quantity defined using the 
coframe and torsion, expressed as an exact derivative:
\begin{equation}
  N = \dd( e^a \wedgep T_a ).
\end{equation}
The ``Nieh--Yan (NY) term'' in this paper refers to the contribution 
$\theta_{\NY} N$ in the action, where $\theta_{\NY}$ is a coupling constant.

\noindent
In differential-form notation, the Nieh--Yan 4-form satisfies the identity
\begin{equation}
N \;=\; d\!\left(e^a \wedge T_a\right)
\;=\; T^a \wedge T_a \;-\; e^a \wedge e^b \wedge R_{ab},
\label{eq:NY_identity}
\end{equation}
which holds for an arbitrary coframe $e^a$ and independent connection $\omega^{ab}$.
For a constant coupling $\theta_{\rm NY}$ on a compact manifold without boundary,
$\int N$ is a boundary term and does not modify the bulk Euler--Lagrange equations.
In minisuperspace reduction, however, the NY term can shift the reduced
Euclidean action between homogeneous configurations once boundary conditions 
along the $S^1$ direction are specified.
Throughout this paper, we therefore treat $\theta_{\rm NY} N$ as part of the reduced
action density defining the effective potential, while noting its boundary-term 
character in the full 4D theory.

\paragraph{Component decomposition of NY density:}
In frame basis calculations, the NY density decomposes into two contributions:
\begin{equation}
  N = N_{\TT} - N_{\REE},
\end{equation}
where:
\begin{itemize}
  \item $N_{\TT}$: Torsion-torsion term, taking the form 
        $\frac{1}{4}\varepsilon^{abcd} T^{e}{}_{ab} T_{ecd}$.
  \item $N_{\REE}$: Riemann-torsion term, taking the form 
        $\frac{1}{4}\varepsilon^{abcd} R_{abcd}$ (curvature from EC connection).
\end{itemize}
In this paper, we take the complete NY density \textbf{FULL} as the primary 
object, while also using \textbf{TT} ($N_{\TT}$ only) and \textbf{REE} 
($N_{\REE}$ only) as diagnostic comparisons.

\noindent
We emphasize that $N_{\rm TT}$ and $N_{\rm REE}$ are \emph{diagnostic} pieces 
corresponding to the two terms in Eq.~\eqref{eq:NY_identity}:
\begin{equation}
N_{\rm TT} := T^a \wedge T_a, \qquad
N_{\rm REE} := e^a \wedge e^b \wedge R_{ab}, \qquad
N = N_{\rm TT} - N_{\rm REE}.
\label{eq:NY_split}
\end{equation}
These are introduced to disentangle contributions within FULL, 
not as independent fundamental theories.

\paragraph{Note on exact derivative property:}
The fact that the NY density can be written as an exact derivative implies that 
its integral over a closed manifold may become a topological invariant. However, 
in minisuperspace reduction, the boundary term contribution can become non-trivial 
due to the symmetry of the ansatz. Within the scope of this work, we operationally 
classify the influence on the phase structure of $\Veff(r)$ obtained through reduction.

%------------------------------------------------------------------------------
\subsection{Minisuperspace Reduction}
\label{sec:reduction_scheme}

\subsubsection{Basic strategy of reduction}

The minisuperspace reduction in this paper is a procedure that substitutes an 
ansatz assuming spatial homogeneity into the action and reduces it to an 
effective action with finite degrees of freedom through spatial integration.

Specifically, we decompose the 4-dimensional Euclidean manifold as 
$\mathcal{M}_4 = \mathcal{M}_3 \times \Sone$, adopting a compact quotient of a 
3-dimensional Lie group admitting left-invariant coframes for $\mathcal{M}_3$. 
Let $L$ denote the circumference in the $\Sone$ direction, and characterize the 
``size'' of $\mathcal{M}_3$ by a single scale parameter $r$.

With this setup, the field degrees of freedom are reduced to the following 
finite number of parameters:
\begin{itemize}
  \item Spatial scale variable $r$ (argument of the effective potential $\Veff(r)$)
  \item A finite number of parameters specifying torsion amplitude 
        (see Sec.~\ref{sec:torsion_ansatz})
\end{itemize}

\subsubsection{Coframe and structure constants}

We introduce left-invariant coframes $\{\sigma^i\}$ ($i = 0, 1, 2$) on the 
3-dimensional space $\mathcal{M}_3$ and construct the 4-dimensional coframe as
\begin{equation}
  \begin{aligned}
    e^a &= r \, \sigma^i \quad (a = i = 0,1,2), \\
    e^a &= L \, \dd\tau \quad (a = 3),
  \end{aligned}
\end{equation}
where $\tau \in [0, 1)$ is the periodic coordinate in the $\Sone$ direction.

The left-invariant coframes satisfy the Maurer--Cartan structure equation
\begin{equation}
  \dd\sigma^i = -\frac{1}{2} C^i{}_{jk} \, \sigma^j \wedgep \sigma^k.
\end{equation}
The structure constants $C^i{}_{jk}$ characterize the geometry of $\mathcal{M}_3$ 
and take the following values for the three test beds treated in this paper:

\begin{table}[H]
\centering
\caption{Structure constants and background Ricci scalar for each topology.}
\label{tab:structure_constants}
\begin{tabular}{@{}lll@{}}
\toprule
Topology & Structure constants & Background Ricci scalar $R_{\LC}$ \\
\midrule
$\Sthree$ (SU(2)) & $C^{i}{}_{jk} = \frac{4}{r} \varepsilon_{ijk}$ & $+24/r^2$ (positive curvature) \\
$\Tthree$ (Abelian) & $C^{i}{}_{jk} = 0$ & $0$ (flat) \\
$\Nilthree$ (Heisenberg) & $C^{2}{}_{01} = -1/r$, $C^{2}{}_{10} = +1/r$, others 0 & $-1/(2r^2)$ (negative curvature) \\
\bottomrule
\end{tabular}
\end{table}

Here $\varepsilon_{ijk}$ is the 3-dimensional Levi--Civita symbol.

Note that the $r$-dependence of the structure constants results from adopting 
the orthonormal frame $e^a = r \, \sigma^a$. Here $\sigma^a$ is the left-invariant 
coframe of ``unit size,'' and rescaling by the physical scale $r$ introduces 
the $1/r$ factor in the structure constants.

\subsubsection{Torsion ansatz: Parametrization based on irreducible decomposition}
\label{sec:torsion_ansatz}

In Einstein--Cartan theory, the torsion tensor $T^a{}_{bc}$ is a geometric degree 
of freedom independent of the coframe. Following the irreducible decomposition 
of torsion in 4 dimensions (Hehl et al.~\cite{Hehl1976}), we introduce the 
following two components:

\paragraph{T1 component (totally antisymmetric / Axial):}
As totally antisymmetric torsion in the spatial directions, we adopt
\begin{equation}
  T^{(1)}_{abc} = \frac{2\eta}{r} \, \varepsilon_{abc} \quad (a, b, c \in \{0, 1, 2\}).
\end{equation}
Here $\eta$ is a dimensionless parameter, scanned including its sign.

This component corresponds to the pseudovector (axial vector) part of torsion 
$S^\mu = \varepsilon^{\mu\nu\rho\sigma} T_{\nu\rho\sigma}$, with $S^3$ taking 
a nonzero value.

\paragraph{T2 component (vector trace):}
Using a vector $V_\mu = (0, 0, 0, V)$ along the $\Sone$ direction, we adopt
\begin{equation}
  T^{(2)}_{abc} = \frac{1}{3} \left( \eta_{ac} V_b - \eta_{ab} V_c \right).
\end{equation}
Here $V > 0$ is a positive-valued parameter, and $\eta_{ab}$ is the frame metric 
(in this paper, $\delta_{ab}$).

This component corresponds to the vector trace part of torsion 
$T_\mu = T^\lambda{}_{\mu\lambda}$.

\paragraph{Mode definitions:}
By combining T1 and T2, we define the following three computational modes:

\begin{table}[H]
\centering
\caption{Torsion mode definitions.}
\label{tab:modes}
\begin{tabular}{@{}llll@{}}
\toprule
Mode & T1 (Axial) & T2 (Vector) & Independent parameters \\
\midrule
AX & $\checkmark$ & --- & $\eta$ \\
VT & --- & $\checkmark$ & $V$ \\
MX & $\checkmark$ & $\checkmark$ & $\eta$, $V$ \\
\bottomrule
\end{tabular}
\end{table}

The main results of this paper are based on MX mode (mixed). AX and VT modes 
are used for diagnostic purposes to disentangle the contribution of each 
torsion component.

%------------------------------------------------------------------------------
\subsection{Parameters and Scanning Variables}
\label{sec:parameters}

\subsubsection{Parameter list}

The main parameters used in this paper are summarized below:

\paragraph{Geometric parameters:}
\begin{itemize}
  \item $r$: Spatial scale variable. Argument of $\Veff(r)$, scanned over $r > 0$.
  \item $L$: Circumference in the $\Sone$ direction. Fixed at $L = 1$ in this paper.
\end{itemize}

\paragraph{Torsion parameters:}
\begin{itemize}
  \item $\eta$: Axial torsion amplitude. Scanned over $\eta \in [-10, 5]$ 
        including sign.
  \item $V$: Vector torsion amplitude. Scanned over $V \in [0, 5]$.
\end{itemize}

\paragraph{Coupling constants:}
\begin{itemize}
  \item $\kappa$: Gravitational coupling constant. Fixed at $\kappa = 1$ in this paper.
  \item $\theta_{\NY}$: Nieh--Yan coupling. Scanned over $\theta_{\NY} \in [0, 5]$. 
        For $\theta_{\NY} < 0$, the sign of the $B$ term ($r^2$ coefficient) in 
        the effective potential is reversed, and the phase structure is mirrored 
        in the $\eta$ direction, so it can be inferred from results in the 
        positive range.
\end{itemize}

\subsubsection{Scanning strategy}

For phase diagram generation, we primarily visualize slices of the $(V, \eta)$ 
plane for representative values of $\theta_{\NY}$ (e.g., $\theta_{\NY} = 0, 1, 2$).

For each parameter point, we search for extrema of $\Veff(r)$ in the allowed 
region $[r_{\min}, r_{\max}] = [0.01, 10^6]$ and determine the Type I/II/III 
classification (Sec.~\ref{sec:type_definition}).

\subsubsection{Note on dimensional analysis}

Since we fix $\kappa = L = 1$, all quantities are dimensionless. To restore 
physical units, $r$ is measured in units of the Planck length 
$\ell_P = \sqrt{\hbar G / c^3}$, and $\Veff$ in units of $\hbar c / \ell_P$. 
However, the focus of this paper is on phase structure classification, and 
we do not discuss absolute scales.

%------------------------------------------------------------------------------
\subsection{Definition of Effective Potential and ``Static Points''}
\label{sec:effective_potential}

In this work, we define the effective potential $\Veff(r)$ for $r$ from the 
effective action $S_{\mathrm{eff}}$ obtained through reduction as follows.

\paragraph{Derivation procedure:}
\begin{enumerate}
  \item Substitute the ansatz from Sec.~\ref{sec:reduction_scheme} into the EC+NY action
  \item Perform spatial integration to obtain effective action 
        $S_{\mathrm{eff}}[r]$ depending only on $r$
  \item Define the effective potential as $\Veff(r) := -S_{\mathrm{eff}}[r]$
\end{enumerate}

With this sign convention, minima of $\Veff(r)$ correspond to maxima of the 
action (dominant contributions in the Euclidean path integral).

\paragraph{Definition of static points:}
Points $r = r_0$ satisfying the extremum condition
\begin{equation}
  \frac{\dd\Veff}{\dd r} = 0
\end{equation}
are called ``static points.'' Furthermore, if
\begin{equation}
  \left.\frac{\dd^2 \Veff}{\dd r^2}\right|_{r=r_0} > 0
\end{equation}
is satisfied, $r_0$ is a local minimum and represents a stable static point.

Hereafter, we operationally classify phases (Types) based on the shape of 
$\Veff(r)$ (extrema and barriers).

%------------------------------------------------------------------------------
\subsection{Type I/II/III: Operational Definition of Phases}
\label{sec:type_definition}

In this paper, we define the following three types based on the shape of $\Veff(r)$. 
``Phase'' is a convenient expression and refers to operational classification, 
not thermodynamic phase transitions. Figure~\ref{fig:schematic} shows the 
potential shapes for each phase.

\begin{figure}[htbp]
  \centering
  \includegraphics[width=1.0\textwidth]{figures/fig01_Schematic_Classification.png}
  \caption{Schematic illustration of Type I/II/III classification based on 
           effective potential shape.}
  \label{fig:schematic}
\end{figure}

\paragraph{Type I: Metastable well with barrier}
\begin{itemize}
  \item $\Veff(r)$ has a local minimum $r_0$ within the allowed region
  \item In the $r \to 0$ direction, there exists a region where $\Veff(r)$ 
        takes values larger than $\Veff(r_0)$ (barrier)
  \item Numerical criterion: Difference $\Delta V > 0$ between $\Veff$ at $r_0$ 
        and maximum value for $r < r_0$
\end{itemize}

\paragraph{Type II: Rolling (no barrier)}
\begin{itemize}
  \item $\Veff(r)$ has a local minimum $r_0$ within the allowed region
  \item However, no barrier exists in the $r \to 0$ direction 
        ($\Delta V \approx 0$ or $\Veff$ monotonically decreasing)
  \item Numerical criterion: $\dd\Veff/\dd r|_{r \to 0^+} < 0$ 
        (downward slope near origin)
\end{itemize}

\paragraph{Type III: Unstable / boundary-attached}
\begin{itemize}
  \item No stable local minimum exists within the allowed region
  \item Typical examples:
    \begin{itemize}
      \item Extremum search reaches $r_{\min}$ or $r_{\max}$ (bound hit)
      \item Curvature condition $\dd^2 V/\dd r^2 > 0$ not satisfied at minimum
      \item Numerical non-convergence
    \end{itemize}
\end{itemize}

\textbf{Note:} The distinction between Type I/II depends on the behavior of 
$\Veff$ near $r = 0$. In Type I, a ``wall'' exists near the origin requiring 
quantum tunneling, while in Type II, classical rolling down is possible. 
However, this paper does not treat dynamical evolution (tunneling rates, 
Friedmann equations).

%------------------------------------------------------------------------------
\subsection{Numerical Criteria}
\label{sec:numerical_criteria}

\subsubsection{Search region and bound hit}

Extremum search is performed over $r \in [r_{\min}, r_{\max}] = [0.01, 10^6]$. 
If the search result reaches near the boundary
\begin{equation}
  r_0 < r_{\min} + \delta \quad \text{or} \quad r_0 > r_{\max} - \delta
\end{equation}
(with $\delta = 0.02$), it is classified as bound hit and assigned Type III.

\subsubsection{Barrier height $\Delta V$ definition and threshold}

For Type I, barrier height $\Delta V$ is defined as:
\begin{equation}
  \Delta V := \max_{r \in [r_{\min}, r_0]} \Veff(r) - \Veff(r_0).
\end{equation}
Type I is determined when $\Delta V > 0$ and 
$\dd\Veff/\dd r|_{r \to 0^+} > 0$.

\subsubsection{Curvature condition}

The curvature at minimum $r_0$ is evaluated by numerical differentiation, and
\begin{equation}
  \left.\frac{\dd^2 \Veff}{\dd r^2}\right|_{r=r_0} > 0
\end{equation}
is confirmed. If this condition is not satisfied, it is classified as Type III.

Detailed numerical parameters and edge case handling are summarized in 
Appendix~\ref{app:numerical}.

%------------------------------------------------------------------------------
\subsection{Incorporation of Nieh--Yan: FULL and Diagnostic Comparisons (TT/REE)}
\label{sec:NY_variants}

The primary object of this work is the \textbf{FULL} case, which incorporates 
the complete form of the NY density into the action. Additionally, we use 
\textbf{TT} and \textbf{REE} as diagnostic comparisons to disentangle the 
origin of FULL's contributions.

\begin{table}[H]
\centering
\caption{NY variant definitions.}
\label{tab:ny_variants}
\begin{tabular}{@{}lll@{}}
\toprule
Variant & NY density adopted & Purpose \\
\midrule
FULL & $N = N_{\TT} - N_{\REE}$ & Main result (complete NY effect) \\
TT & $N_{\TT}$ only & Diagnosis of torsion-torsion contribution \\
REE & $N_{\REE}$ only & Diagnosis of Riemann-torsion contribution \\
\bottomrule
\end{tabular}
\end{table}

\paragraph{Specific processing in calculations:}
The engine (DPPUv2 Engine Core v3) computes $N_{\TT}$, $N_{\REE}$, and 
$N_{\FULL}$ at each step, and selects the NY density to incorporate into 
the Lagrangian according to the specified variant.

\paragraph{Significance of diagnostic comparisons:}
TT/REE are not proposed as independent fundamental theories but serve as 
auxiliary comparisons for understanding which contributions drive the phase 
structure of FULL. For example:
\begin{itemize}
  \item If phase boundaries differ significantly between TT and REE 
        $\rightarrow$ Competition between both contributions determines 
        phase structure
  \item If TT and FULL are nearly identical $\rightarrow$ Torsion-torsion 
        term is dominant
\end{itemize}
This diagnosis enables disentangling the geometric origin of topology dependence.

%------------------------------------------------------------------------------
\subsection{Overview of Computational Pipeline}
\label{sec:pipeline}

The numerical results of this work are obtained through the computational 
pipeline shown in Figure~\ref{fig:pipeline}. The pipeline consists of two 
major phases: the \textbf{theory building phase} (symbolic computation for 
deriving $\Veff(r)$) and the \textbf{numerical search phase} (phase 
classification through parameter scanning).

\subsubsection{Phase 1: Theory building phase (symbolic computation)}

\paragraph{Step 1: Geometric setup and connection calculation}
Set structure constants $C^i{}_{jk}$ according to topology 
(Sec.~\ref{sec:reduction_scheme}), compute the Levi--Civita connection via 
the generalized Koszul formula. Derive contortion from the torsion ansatz 
(Sec.~\ref{sec:torsion_ansatz}) and construct the EC connection. 
Automatically verify metric compatibility and Riemann tensor antisymmetry 
at each step.

Details of derivation are in Appendix~\ref{app:theory}; engine specifications 
and verification are in Appendix~\ref{app:reproducibility}.

\paragraph{Step 2: Effective potential derivation}
Compute Ricci scalar $R$, torsion scalar $T_{abc}T^{abc}$, and NY densities 
($N_{\TT}$, $N_{\REE}$, $N_{\FULL}$) from the EC connection. Construct the 
Lagrangian density
\begin{equation}
  \mathcal{L} = \frac{R}{2\kappa^2} + \theta_{\NY} N
\end{equation}
and obtain the effective action $S_{\mathrm{eff}}$ through angular integration. 
The effective potential is extracted as $\Veff(r) = -S_{\mathrm{eff}}$.

Analytical results for each topology and NY variant are summarized in 
Appendix~\ref{app:theory}.

\subsubsection{Phase 2: Numerical search phase (parameter scan)}

\paragraph{Step 3: Extremum search and stability determination}
For each parameter point $(V, \eta, \theta_{\NY})$, numerically search for 
extrema of $\Veff(r)$ in the range $r \in [r_{\min}, r_{\max}]$. Use Brent's 
method for extremum search and verify curvature condition 
$\dd^2\Veff/\dd r^2 > 0$ by numerical differentiation.

Details of search algorithm are in Appendix~\ref{app:numerical}.

\paragraph{Step 4: Type classification and phase diagram generation}
Classify each parameter point as Type I/II/III according to the criteria 
in Sec.~\ref{sec:type_definition}--\ref{sec:numerical_criteria}. Output 
classification results in CSV format and generate phase diagrams on the 
$(V, \eta)$ plane.

Type determination flowchart is in Appendix~\ref{app:numerical}; visualization 
tools are in Appendix~\ref{app:visualization}.

\subsubsection{Implementation and reproducibility}

Calculations are implemented as symbolic computation using SymPy (DPPUv2 Engine 
Core v3), with automatic execution of consistency checks at each step 
(metric compatibility, three-stage verification of Riemann antisymmetry, etc.). 
Parameter scanning is parallelized for speed.

Engine specifications and sanity check list are in Appendix~\ref{app:reproducibility}; 
code and data access information are in Appendix~\ref{app:access}.

\begin{figure}[htbp]
  \centering
  \includegraphics[width=0.75\textwidth]{figures/Fig02_Computational_pipeline_overview.png}
  \caption{Computational pipeline overview.}
\label{fig:pipeline}
\end{figure}
  % Setup and conventions
%==============================================================================
% Section 3: Topology-Specific Reductions
%==============================================================================
\section{Topology-Specific Reductions}
\label{sec:reductions}

This section applies the minisuperspace reduction defined in Sec.~\ref{sec:setup} 
to the three test beds ($\Sthree$, $\Tthree$, $\Nilthree$) and derives the 
explicit form of the effective potential $\Veff(r)$ for each topology.

%------------------------------------------------------------------------------
\subsection{$\Sthree \times \Sone$: Reduced Action and Effective Potential}
\label{sec:S3_reduction}

\subsubsection{Geometric setup}

$\Sthree$ is realized as the SU(2) group manifold, with left-invariant coframes 
$\{\sigma^i\}$ given by the Maurer-Cartan forms of SU(2). The structure constants are
\begin{equation}
  C^{i}{}_{jk} = \frac{4}{r} \varepsilon_{ijk},
\end{equation}
and the background Ricci scalar (without torsion) is
\begin{equation}
  R_{\LC} = \frac{24}{r^2} > 0
\end{equation}
(positive curvature). The volume element is
\begin{equation}
  \mathrm{Vol}(\Sthree \times \Sone) = 2\pi^2 L r^3.
\end{equation}

\subsubsection{Scalar quantities}

In MX mode ($\eta \neq 0$, $V \neq 0$), the scalar quantities computed by 
DPPUv2 Engine Core v3 are as follows:

\paragraph{Ricci scalar:}
\begin{equation}
  R = \frac{2(-V^2 r^2 - 9\eta^2 - 72\eta - 108)}{3r^2}
\end{equation}

\paragraph{Torsion scalar:}
\begin{equation}
  T_{abc}T^{abc} = \frac{2V^2}{3} + \frac{24\eta^2}{r^2}
\end{equation}

\paragraph{NY densities (each variant):}
\begin{equation}
  N_{\TT} = -\frac{4V\eta}{r}, \quad 
  N_{\REE} = -\frac{2V(\eta + 4)}{r}, \quad 
  N_{\FULL} = \frac{2V(4 - \eta)}{r}
\end{equation}

\subsubsection{Effective potential}

Integrating the Lagrangian $\mathcal{L} = R/(2\kappa^2) + \theta_{\NY} N$ 
over the volume and extracting $\Veff(r) = -S_{\mathrm{eff}}$, we obtain 
for each NY variant:

\paragraph{FULL ($N = N_{\FULL}$):}
\begin{equation}
  \Veff^{(\Sthree,\FULL)}(r) = \frac{2\pi^2 L}{3\kappa^2} r 
  \left[ V^2 r^2 + 6V\kappa^2\theta_{\NY}(\eta - 4) r + 9\eta^2 + 72\eta + 108 \right]
  \label{eq:Veff_S3_FULL}
\end{equation}

\paragraph{TT ($N = N_{\TT}$):}
\begin{equation}
  \Veff^{(\Sthree,\TT)}(r) = \frac{2\pi^2 L}{3\kappa^2} r 
  \left[ V^2 r^2 + 12V\eta\kappa^2\theta_{\NY} r + 9\eta^2 + 72\eta + 108 \right]
\end{equation}

\paragraph{REE ($N = N_{\REE}$):}
\begin{equation}
  \Veff^{(\Sthree,\REE)}(r) = \frac{2\pi^2 L}{3\kappa^2} r 
  \left[ V^2 r^2 + 6V\kappa^2\theta_{\NY}(\eta + 4) r + 9\eta^2 + 72\eta + 108 \right]
\end{equation}

These potential shapes are shown in Figure~\ref{fig:variant_S3}.

\begin{figure}[htbp]
  \centering
  \includegraphics[width=1.0\textwidth]{figures/fig03_variant_comparison_S3.png}
  \caption{Nieh--Yan variant comparison for $\Sthree \times \Sone$.}
  \label{fig:variant_S3}
\end{figure}

\subsubsection{Structural analysis}

The effective potential for $\Sthree$ has a cubic polynomial structure in $r$ 
(of the form $r^3 + r^2 + r$):
\begin{equation}
  \Veff(r) \propto r \cdot \left[ A r^2 + B r + C \right],
\end{equation}
where:
\begin{itemize}
  \item $A = V^2 > 0$: Governs divergence as $r \to \infty$ (always positive)
  \item $B$: Proportional to NY coupling $\theta_{\NY}$, depends on $\eta$ 
        (sign can vary)
  \item $C = 9\eta^2 + 72\eta + 108 = 9(\eta + 4)^2 - 36$: Takes minimum 
        value $-36$ at $\eta = -4$
\end{itemize}

From this structure, a local minimum can appear when $B < 0$ and $|B|$ is 
sufficiently large. In particular, for the FULL variant, $B \propto (\eta - 4)$, 
so $B < 0$ for $\eta < 4$, meaning local minima can appear even in the 
$\eta > 0$ region.

The contributions of $r^3$, $r^2$, and $r$ terms to the potential are shown 
in Figure~\ref{fig:decomposition_S3}.

\begin{figure}[htbp]
  \centering
  \includegraphics[width=0.9\textwidth]{figures/fig04_decomposition_S3.png}
  \caption{Potential decomposition for $\Sthree \times \Sone$.}
  \label{fig:decomposition_S3}
\end{figure}

%------------------------------------------------------------------------------
\subsection{$\Tthree \times \Sone$: Reduced Action and Effective Potential}
\label{sec:T3_reduction}

\subsubsection{Geometric setup}

$\Tthree$ is an Abelian group (discrete quotient of $\mathbb{R}^3$), with all 
structure constants vanishing:
\begin{equation}
  C^i{}_{jk} = 0.
\end{equation}
The background Ricci scalar is
\begin{equation}
  R_{\LC} = 0
\end{equation}
(flat). The volume element is
\begin{equation}
  \mathrm{Vol}(\Tthree \times \Sone) = (2\pi)^4 L R_1 R_2 R_3,
\end{equation}
where $R_1, R_2, R_3$ are the circumferences in each direction of $\Tthree$. 
In this paper, for comparison with $\Sthree$ and $\Nilthree$, we assume 
isotropic expansion $R_1 = R_2 = R_3 = r$. This allows description with a 
single scale parameter $r$, as with the other topologies.

\subsubsection{Scalar quantities}

In MX mode, the scalar quantities are:

\paragraph{Ricci scalar:}
\begin{equation}
  R = -\frac{2V^2}{3} - \frac{6\eta^2}{r^2}
\end{equation}

\paragraph{Torsion scalar:}
\begin{equation}
  T_{abc}T^{abc} = \frac{2V^2}{3} + \frac{24\eta^2}{r^2}
\end{equation}

\paragraph{NY densities (each variant):}
\begin{equation}
  N_{\TT} = -\frac{4V\eta}{r}, \quad 
  N_{\REE} = -\frac{2V\eta}{r}, \quad 
  N_{\FULL} = -\frac{2V\eta}{r}
\end{equation}

Notably, for $\Tthree$, $N_{\FULL} = N_{\REE}$. This is due to the zero 
background curvature.

\subsubsection{Effective potential}

With the isotropic setting $R_1 = R_2 = R_3 = r$, for each NY variant:

\paragraph{FULL (= REE):}
\begin{equation}
  \Veff^{(\Tthree,\FULL)}(r) = \frac{16\pi^4 L}{3\kappa^2} 
  \left[ V^2 r^3 + 6V\eta\kappa^2\theta_{\NY} r^2 + 9\eta^2 r \right]
  \label{eq:Veff_T3_FULL}
\end{equation}

\paragraph{TT:}
\begin{equation}
  \Veff^{(\Tthree,\TT)}(r) = \frac{16\pi^4 L}{3\kappa^2} 
  \left[ V^2 r^3 + 12V\eta\kappa^2\theta_{\NY} r^2 + 9\eta^2 r \right]
\end{equation}

These have the same $r^3 + r^2 + r$ structure as $\Sthree$.

The potential shapes are shown in Figure~\ref{fig:variant_T3}.

\begin{figure}[htbp]
  \centering
  \includegraphics[width=1.0\textwidth]{figures/fig05_variant_comparison_T3.png}
  \caption{Nieh--Yan variant comparison for $\Tthree \times \Sone$.}
  \label{fig:variant_T3}
\end{figure}

\subsubsection{Structural analysis}

With the isotropic setting, the effective potential for $\Tthree$ has the 
same structure as $\Sthree$ and $\Nilthree$:
\begin{equation}
  \Veff(r) \propto r \cdot \left[ V^2 r^2 + B r + C \right],
\end{equation}
where $B = 6V\eta\kappa^2\theta_{\NY}$ and $C = 9\eta^2$.

From this form:
\begin{itemize}
  \item $r \to 0$: $\Veff \to 0^+$ (always positive for $C > 0$ when $\eta \neq 0$)
  \item $r \to \infty$: $\Veff \to +\infty$ due to the $r^3$ term
  \item Minimum existence condition: $3V^2 r^2 + 2Br + C = 0$ has positive 
        real solutions
\end{itemize}

A characteristic of $\Tthree$ is that the coefficient $C = 9\eta^2$ of $r$ 
is always non-negative (zero only when $\eta = 0$). Therefore:
\begin{itemize}
  \item For $\theta_{\NY} = 0$, $B = 0$, and 
        $\dd\Veff/\dd r \propto 3V^2 r^2 + 9\eta^2 > 0$ (for $r > 0$), 
        so it is always monotonically increasing (Type III)
  \item For $\theta_{\NY} > 0$ and $\eta < 0$, $B < 0$, and minima can form
\end{itemize}

Under $\eta \to -\eta$ transformation, $B \to -B$, so for $\theta_{\NY} \neq 0$, 
symmetry is broken. For $\theta_{\NY} > 0$, stable regions appear only on 
the $\eta < 0$ side.

The contributions of $r^3$, $r^2$, and $r$ terms are shown in 
Figure~\ref{fig:decomposition_T3}.

\begin{figure}[htbp]
  \centering
  \includegraphics[width=0.9\textwidth]{figures/fig06_decomposition_T3.png}
  \caption{Potential decomposition for $\Tthree \times \Sone$.}
  \label{fig:decomposition_T3}
\end{figure}

%------------------------------------------------------------------------------
\subsection{$\Nilthree \times \Sone$: Reduced Action and Effective Potential}
\label{sec:Nil3_reduction}

\subsubsection{Geometric setup}

$\Nilthree$ is a compact quotient of the Heisenberg group (Bianchi Type II), 
with structure constants
\begin{equation}
  C^{2}{}_{01} = -\frac{1}{r}, \quad C^{2}{}_{10} = +\frac{1}{r}, \quad 
  \text{others } 0.
\end{equation}
An important point is that $\Nilthree$ is \textbf{not bi-invariant} (the 
Heisenberg group is nilpotent, not semisimple). Therefore, the generalized 
Koszul formula (see Appendix~\ref{app:theory}) must be used for connection 
calculations.

The background Ricci scalar is
\begin{equation}
  R_{\LC} = -\frac{1}{2r^2} < 0
\end{equation}
(negative curvature). The volume element is
\begin{equation}
  \mathrm{Vol}(\Nilthree \times \Sone) = (2\pi)^4 L r^3.
\end{equation}

\subsubsection{Scalar quantities}

In MX mode, the scalar quantities are:

\paragraph{Ricci scalar:}
\begin{equation}
  R = \frac{-4V^2 r^2 - 36\eta^2 + 24\eta + 9}{6r^2}
\end{equation}

\paragraph{Torsion scalar:}
\begin{equation}
  T_{abc}T^{abc} = \frac{2V^2}{3} + \frac{24\eta^2}{r^2}
\end{equation}

\paragraph{NY densities (each variant):}
\begin{equation}
  N_{\TT} = -\frac{4V\eta}{r}, \quad 
  N_{\REE} = \frac{2V(1 - 3\eta)}{3r}, \quad 
  N_{\FULL} = -\frac{2V(3\eta + 1)}{3r}
\end{equation}

\subsubsection{Effective potential}

For each NY variant:

\paragraph{FULL:}
\begin{equation}
  \Veff^{(\Nilthree,\FULL)}(r) = \frac{4\pi^4 L}{3\kappa^2} r 
  \left[ 4V^2 r^2 + 8V\kappa^2\theta_{\NY}(3\eta + 1) r 
         + 36\eta^2 - 24\eta - 9 \right]
  \label{eq:Veff_Nil3_FULL}
\end{equation}

\paragraph{TT:}
\begin{equation}
  \Veff^{(\Nilthree,\TT)}(r) = \frac{4\pi^4 L}{3\kappa^2} r 
  \left[ 4V^2 r^2 + 48V\eta\kappa^2\theta_{\NY} r + 36\eta^2 - 24\eta - 9 \right]
\end{equation}

\paragraph{REE:}
\begin{equation}
  \Veff^{(\Nilthree,\REE)}(r) = \frac{4\pi^4 L}{3\kappa^2} r 
  \left[ 4V^2 r^2 + 8V\kappa^2\theta_{\NY}(3\eta - 1) r 
         + 36\eta^2 - 24\eta - 9 \right]
\end{equation}

The potential shapes are shown in Figure~\ref{fig:variant_Nil3}.

\begin{figure}[htbp]
  \centering
  \includegraphics[width=1.0\textwidth]{figures/fig07_variant_comparison_Nil3.png}
  \caption{Nieh--Yan variant comparison for $\Nilthree \times \Sone$.}
  \label{fig:variant_Nil3}
\end{figure}

\subsubsection{Structural analysis}

The effective potential for $\Nilthree$ also has the $r^3 + r^2 + r$ structure 
like $\Sthree$, but the sign of coefficients brings differences:
\begin{equation}
  C_{\Nilthree} = 36\eta^2 - 24\eta - 9 = 36\left(\eta - \frac{1}{3}\right)^2 - 13.
\end{equation}

This quadratic function takes minimum value $-13$ at $\eta = 1/3$, and becomes 
negative in the range $\eta \in (-0.27, 0.94)$.

When the coefficient $C$ of $r$ is negative, $\Veff(r) \to 0^-$ as $r \to 0$, 
meaning the potential takes negative values near the origin. This contrasts 
with $\Sthree$ (where $C > 0$ over a wide region), suggesting that for 
$\Nilthree$ with small $\theta_{\NY}$ values, the stable region is limited 
to a narrow band.

The contributions of $r^3$, $r^2$, and $r$ terms are shown in 
Figure~\ref{fig:decomposition_Nil3}.

\begin{figure}[htbp]
  \centering
  \includegraphics[width=0.9\textwidth]{figures/fig08_decomposition_Nil3.png}
  \caption{Potential decomposition for $\Nilthree \times \Sone$.}
  \label{fig:decomposition_Nil3}
\end{figure}

%------------------------------------------------------------------------------
\subsection{Structural Comparison: Sources of Topology Dependence}
\label{sec:structure_comparison}

We compare the structure of effective potentials across the three topologies 
and organize the sources of topology dependence.

\subsubsection{General form of effective potential}

For all topologies, the effective potential in MX mode is expressed in the 
general form:
\begin{equation}
  \Veff(r) = \mathcal{N} \cdot r^{\alpha} \cdot P(r),
\end{equation}
where $\mathcal{N}$ is a normalization factor, $\alpha$ is a topology-dependent 
exponent, and $P(r)$ is a polynomial in $r$.

\begin{table}[H]
\centering
\caption{Effective potential structure comparison.}
\label{tab:structure_comparison}
\begin{tabular}{@{}llll@{}}
\toprule
Topology & Structure & $r \to 0$ & $r \to \infty$ \\
\midrule
$\Sthree$ & $r \cdot [Ar^2 + Br + C]$ & $\to 0$ & $\to +\infty$ \\
$\Tthree$ & $r \cdot [Ar^2 + Br + C]$ & $\to 0$ & $\to +\infty$ \\
$\Nilthree$ & $r \cdot [Ar^2 + Br + C]$ & $\to 0$ & $\to +\infty$ \\
\bottomrule
\end{tabular}
\end{table}

All of $\Sthree$, $\Tthree$, and $\Nilthree$ share the $r \cdot [Ar^2 + Br + C]$ 
structure.

\subsubsection{Comparison of NY densities}

\begin{table}[H]
\centering
\caption{Comparison of NY densities ($N_{\FULL}$) and background curvature 
dependence.}
\label{tab:NY_comparison}
\begin{tabular}{@{}lll@{}}
\toprule
Topology & $N_{\FULL}$ & Background curvature dependence \\
\midrule
$\Sthree$ & $\displaystyle\frac{2V(4 - \eta)}{r}$ & 
            $(4 - \eta)$: contribution from positive curvature \\
$\Tthree$ & $\displaystyle-\frac{2V\eta}{r}$ & 
            $\eta$ only: no curvature contribution \\
$\Nilthree$ & $\displaystyle-\frac{2V(3\eta + 1)}{3r}$ & 
              $(3\eta + 1)$: contribution from negative curvature \\
\bottomrule
\end{tabular}
\end{table}

The factor $(4 - \eta)$ appearing in $N_{\FULL}$ for $\Sthree$ originates from 
the $(\eta + 4)$ term in $N_{\REE}$. This $+4$ reflects coupling with the 
positive background curvature ($R_{\LC} = 24/r^2$).

In contrast, for $\Tthree$, the background curvature is zero, so there is no 
curvature contribution to $N_{\FULL}$.

\subsubsection{Geometric factors affecting stability}

The existence and position of the effective potential minimum are primarily 
determined by the following factors:

\begin{enumerate}
  \item \textbf{Coefficient of $r^3$ term} ($\propto V^2$): Always positive. 
        Governs growth as $r \to \infty$.
  
  \item \textbf{Coefficient of $r^2$ term} ($\propto V\theta_{\NY} \times f(\eta)$): 
        Function of NY coupling and $\eta$. When negative, promotes minimum formation.
  
  \item \textbf{Coefficient of $r$ term}: Topology-specific geometric contribution.
        \begin{itemize}
          \item $\Sthree$: $9(\eta + 4)^2 - 36$
          \item $\Tthree$: $9\eta^2$
          \item $\Nilthree$: $36(\eta - 1/3)^2 - 13$
        \end{itemize}
\end{enumerate}

The competition among these factors produces different phase structures for 
each topology. Detailed analysis of phase boundaries is presented in 
Sec.~\ref{sec:numerical}.
  % Topology-specific reductions
%==============================================================================
% Section 4: Numerical Results: Phase Diagrams and Boundaries
%==============================================================================
\section{Numerical Results: Phase Diagrams and Boundaries}
\label{sec:numerical}

This section presents the results of numerical scanning on the effective 
potentials derived in Sec.~\ref{sec:reductions}. We first explain the Type 
classification procedure and how to read phase diagrams, then show the phase 
diagrams for each topology. The mechanisms of phase boundary formation and 
geometric interpretation are discussed in Sec.~\ref{sec:mechanisms}.

%------------------------------------------------------------------------------
\subsection{Type Classification Procedure}
\label{sec:classification_procedure}

\subsubsection{Search region and numerical method}

For each parameter point $(V, \eta, \theta_{\NY})$, we numerically search for 
extrema of $\Veff(r)$ in the range $r \in [r_{\min}, r_{\max}] = [0.01, 10^6]$. 
We use Brent's method (\texttt{scipy.optimize.minimize\_scalar}) for extremum 
search and verify the curvature condition $\dd^2 V/\dd r^2 > 0$ by numerical 
differentiation.

\subsubsection{Type determination criteria}

Based on the definitions in Sec.~\ref{sec:type_definition}, we determine Types 
using the following criteria:

\begin{itemize}
  \item \textbf{Type I (stable with barrier)}: A local minimum $r_0$ exists 
        within the allowed region, and a barrier ($\Delta V > 0$) exists in 
        the $r \to 0$ direction
  \item \textbf{Type II (rolling)}: A local minimum $r_0$ exists within the 
        allowed region, but no barrier in the $r \to 0$ direction 
        ($\dd\Veff/\dd r|_{r \to 0^+} < 0$)
  \item \textbf{Type III (unstable)}: No stable local minimum exists within 
        the allowed region, or search reaches boundary ($r_{\min}$ or $r_{\max}$)
\end{itemize}

\subsubsection{Scan resolution}

The scan resolution is 51 points in the $V$ direction, 151 points in the 
$\eta$ direction, and 51 points in the $\theta_{\NY}$ direction (range $[0, 5]$), 
giving approximately 390,000 evaluations per topology. Note that this paper 
primarily presents results for $\theta_{\NY} \lesssim 2$. The uncertainty in 
boundary positions is on the order of the grid spacing 
($\Delta\eta \approx 0.1$, $\Delta V \approx 0.1$).

%------------------------------------------------------------------------------
\subsection{How to Read Phase Diagrams}
\label{sec:reading_diagrams}

In the phase diagrams of this paper, Type classification is displayed on the 
$(V, \eta)$ plane using the following visual conventions:

\begin{itemize}
  \item \textbf{Colored region (no hatching)}: Type I (metastable well with barrier)
  \item \textbf{Colored region (with hatching)}: Type II (rolling, no barrier)
  \item \textbf{White region}: Type III (unstable / boundary-attached)
\end{itemize}

The color gradient represents the logarithmic value of the stable radius 
$\log_{10}(r_0)$, ranging from purple (small) to yellow (large). White contour 
lines show the logarithmic barrier height/well depth $\log_{10}(\Delta V)$.

%------------------------------------------------------------------------------
\subsection{Phase Diagrams by Topology: FULL Variant}
\label{sec:phase_diagrams}

\subsubsection{$\Sthree \times \Sone$}

Figure~\ref{fig:phase_S3} shows the phase diagrams for $\Sthree$-FULL at 
$\theta_{\NY} = 0, 1, 2$.

\begin{figure}[htbp]
  \centering
  \includegraphics[width=\textwidth]{figures/fig09_phase_diagram_S3-FULL.png}
  \caption{Phase diagram: $\Sthree \times \Sone$ ($\theta_{\NY} = 0, 1, 2$).}
  \label{fig:phase_S3}
\end{figure}

\paragraph{For $\theta_{\NY} = 0$:}
\begin{itemize}
  \item $\eta \gtrsim -2$: Type III (white) dominates
  \item $-6 \lesssim \eta \lesssim -2$: Band-like region of Type II (hatched)
  \item $\eta \lesssim -6$: Type III again
\end{itemize}

\paragraph{For $\theta_{\NY} = 1$:}
\begin{itemize}
  \item $\eta \gtrsim 0.5$: Type III (white)
  \item $-2 \lesssim \eta \lesssim 0.5$: Type I (no hatching)
  \item $-6 \lesssim \eta \lesssim -2$: Band-like region of Type II (hatched)
  \item $\eta \lesssim -6$: Type I (no hatching) reappears
\end{itemize}

\paragraph{For $\theta_{\NY} = 2$:}
\begin{itemize}
  \item Stable region (Type I + II) expands toward $\eta > 0$
  \item I/III boundary moves to around $\eta \approx 2$
  \item Type II band also expands
\end{itemize}

\paragraph{Key observations:}
\begin{enumerate}
  \item $\theta_{\NY} = 0$: Only main band of Type II
  \item $\theta_{\NY} > 0$: Type I region appears and expands
  \item $\eta \to -\eta$ symmetry is broken
\end{enumerate}

\subsubsection{$\Tthree \times \Sone$}

Figure~\ref{fig:phase_T3} shows the phase diagrams for $\Tthree$-FULL at 
$\theta_{\NY} = 0, 1, 2$.

\begin{figure}[htbp]
  \centering
  \includegraphics[width=\textwidth]{figures/fig10_phase_diagram_T3-FULL.png}
  \caption{Phase diagram: $\Tthree \times \Sone$ ($\theta_{\NY} = 0, 1, 2$).}
  \label{fig:phase_T3}
\end{figure}

\paragraph{For $\theta_{\NY} = 0$:}
\begin{itemize}
  \item Entire region is Type III (monotonically increasing)
  \item No stable minimum exists
\end{itemize}

\paragraph{For $\theta_{\NY} = 1$:}
\begin{itemize}
  \item Type I (stable well) appears in wide region of $\eta < 0$
  \item $\eta \gtrsim 0$ is entirely Type III
\end{itemize}

\paragraph{For $\theta_{\NY} = 2$:}
\begin{itemize}
  \item Type I region expands further
  \item Contour lines of $r_0$ extend in negative $\eta$ direction
\end{itemize}

\paragraph{Key observations:}
\begin{enumerate}
  \item $\theta_{\NY} = 0$: Entire region Type III
  \item $\theta_{\NY} > 0$: Type I appears
  \item $\eta \to -\eta$ symmetry is broken
\end{enumerate}

\subsubsection{$\Nilthree \times \Sone$}

Figure~\ref{fig:phase_Nil3} shows the phase diagrams for $\Nilthree$-FULL at 
$\theta_{\NY} = 0, 1, 2$.

\begin{figure}[htbp]
  \centering
  \includegraphics[width=\textwidth]{figures/fig11_phase_diagram_Nil3-FULL.png}
  \caption{Phase diagram: $\Nilthree \times \Sone$ ($\theta_{\NY} = 0, 1, 2$).}
  \label{fig:phase_Nil3}
\end{figure}

\paragraph{For $\theta_{\NY} = 0$:}
\begin{itemize}
  \item $-0.3 \lesssim \eta \lesssim 1$: Narrow stable band (Type II)
  \item $\eta < -0.3$ and $\eta > 1$: Type III (white) dominates almost 
        the entire region
\end{itemize}

\paragraph{For $\theta_{\NY} = 1$:}
\begin{itemize}
  \item Main band ($-0.3 < \eta < 1$) is maintained
  \item \textbf{Separate stable region (Type I)} appears at $\eta \lesssim -4$
  \item Main band and separate stable region are separated by Type III region
\end{itemize}

\paragraph{For $\theta_{\NY} = 2$:}
\begin{itemize}
  \item Lower stable region expands, distributing widely at $\eta \lesssim -0.5$
  \item Width of main band ($-0.3 < \eta < 1$) gradually decreases in 
        large $V$ region
\end{itemize}

\paragraph{Key observations:}
\begin{enumerate}
  \item $\theta_{\NY} = 0$: Only main band of Type II
  \item $\theta_{\NY} > 0$: Type I region appears and expands
  \item $\eta \to -\eta$ symmetry is broken
\end{enumerate}

%------------------------------------------------------------------------------
\subsection{Summary of $\theta_{\NY}$ Dependence}
\label{sec:theta_dependence}

Tables~\ref{tab:boundary_S3}--\ref{tab:boundary_Nil3} summarize the 
$\theta_{\NY}$ dependence of phase boundaries for each topology.

\begin{table}[H]
\centering
\caption{$\Sthree$: $\theta_{\NY}$ dependence of phase boundary positions 
(FULL variant).}
\label{tab:boundary_S3}
\begin{tabular}{@{}llll@{}}
\toprule
Type & $\theta_{\NY} = 0$ & $\theta_{\NY} = 1$ & $\theta_{\NY} = 2$ \\
\midrule
Type I region & None & $\eta < -6$, $-2 < \eta < 0.5$ & Expanded \\
Type II (main band) & $-6 \lesssim \eta \lesssim -2$ & 
                      $-6 \lesssim \eta \lesssim -2$ & 
                      $-6 \lesssim \eta \lesssim -2$ \\
Type III region & $\eta \lesssim -6$, $-2 \lesssim \eta$ & 
                  $0.5 \lesssim \eta$ & Shrunk \\
\bottomrule
\end{tabular}
\end{table}

\begin{table}[H]
\centering
\caption{$\Tthree$: $\theta_{\NY}$ dependence of phase boundary positions 
(FULL variant).}
\label{tab:boundary_T3}
\begin{tabular}{@{}llll@{}}
\toprule
Type & $\theta_{\NY} = 0$ & $\theta_{\NY} = 1$ & $\theta_{\NY} = 2$ \\
\midrule
Type I region & None & $\eta \lesssim 0$ & $\eta \lesssim 0$ \\
Type II region & None & None & None \\
Type III region & Entire & $0 \lesssim \eta$ & $0 \lesssim \eta$ \\
\bottomrule
\end{tabular}
\end{table}

\begin{table}[H]
\centering
\caption{$\Nilthree$: $\theta_{\NY}$ dependence of phase boundary positions 
(FULL variant).}
\label{tab:boundary_Nil3}
\begin{tabular}{@{}llll@{}}
\toprule
Type & $\theta_{\NY} = 0$ & $\theta_{\NY} = 1$ & $\theta_{\NY} = 2$ \\
\midrule
Type I region & None & $\eta \lesssim -4$ & $\eta < -0.5$ \\
Type II (main band) & $-0.3 \lesssim \eta \lesssim 1$ & 
                      $-0.3 \lesssim \eta \lesssim 1$ & 
                      $-0.3 \lesssim \eta \lesssim 1$ \\
Type III region & $\eta \lesssim -0.3$, $1 \lesssim \eta$ & 
                  $-4 \lesssim \eta \lesssim -0.3$, $1 \lesssim \eta$ & 
                  $-0.5 \lesssim \eta \lesssim -0.3$, $1 \lesssim \eta$ \\
\bottomrule
\end{tabular}
\end{table}

\paragraph{Key observations:}
\begin{enumerate}
  \item \textbf{$\Sthree$}: Shows strong dependence on $\theta_{\NY}$, with 
        stable region expanding toward $\eta > 0$
  \item \textbf{$\Tthree$}: Entire region is Type III for 
        $\theta_{\NY} \lesssim 0.9$, but Type I appears in $\eta < 0$ for 
        $\theta_{\NY} \gtrsim 0.9$
  \item \textbf{$\Nilthree$}: Main band is insensitive to $\theta_{\NY}$, 
        but stable region in $\eta < 0$ grows with $\theta_{\NY}$
\end{enumerate}

%------------------------------------------------------------------------------
\subsection{Comparison with TT/REE Variants}
\label{sec:variant_comparison}

Figure~\ref{fig:phase_matrix} shows the phase diagram matrix for 
3 topologies $\times$ 3 variants at $\theta_{\NY} = 1.0$.

\begin{figure}[htbp]
  \centering
  \includegraphics[width=\textwidth]{figures/fig12_phase_matrix_0010.png}
  \caption{Phase diagram matrix: $\theta_{\NY} = 1.0$.}
  \label{fig:phase_matrix}
\end{figure}

\paragraph{$\Sthree$:}
\begin{itemize}
  \item Phase boundary positions clearly differ among FULL, TT, REE
  \item FULL has the widest stable region
  \item TT and REE have narrower stable regions than FULL
\end{itemize}

\paragraph{$\Tthree$:}
\begin{itemize}
  \item FULL and REE have completely identical phase diagrams 
        (because $N_{\FULL} = N_{\REE}$)
  \item TT differs by a factor of 2 in coefficient, but phase boundary 
        positions are similar
  \item For $\theta_{\NY} \gtrsim 0.9$, Type I region appears in $\eta < 0$ 
        for all variants
\end{itemize}

\paragraph{$\Nilthree$:}
\begin{itemize}
  \item Moderate differences observed among the 3 variants
  \item Main band positions are nearly identical
  \item Shape and extent of lower stable region differ
\end{itemize}
  % Numerical results
%==============================================================================
% Section 5: Mechanisms and Interpretation of Phase Structure
%==============================================================================
\section{Mechanisms and Interpretation of Phase Structure}
\label{sec:mechanisms}

This section interprets the features of the phase diagrams presented in 
Sec.~\ref{sec:numerical} based on the analytical structure of the effective 
potential $\Veff(r)$.

%------------------------------------------------------------------------------
\subsection{Shape Changes of $\Veff(r)$ at Type Transitions}
\label{sec:type_transitions}

\subsubsection{Type III $\to$ Type II transition (appearance of well)}

The transition from Type III to Type II corresponds to the appearance of a 
local minimum in $\Veff(r)$.

Consider the case of $\Sthree$-FULL with $\theta_{\NY} = 0$, $V = 2$ fixed, 
and $\eta$ varying from $0$ to $-2$:

\begin{itemize}
  \item \textbf{$\eta = 0$ (Type III)}: $\Veff(r)$ is monotonically increasing 
        with no minimum within the allowed region
  \item \textbf{$\eta = -2$ (Type II)}: A local minimum appears in $\Veff(r)$. 
        However, there is no barrier in the $r \to 0$ direction, with 
        $\dd\Veff/\dd r < 0$ near the origin
\end{itemize}

This transition can be understood from the effective potential structure 
derived in Sec.~\ref{sec:reductions}. For $\Sthree$:
\begin{equation}
  \Veff(r) \propto r \left[ V^2 r^2 + B r + C \right],
\end{equation}
where $B = 6V\kappa^2\theta_{\NY}(\eta - 4)$ and 
$C = 9\eta^2 + 72\eta + 108 = 9(\eta + 4)^2 - 36$.

For $\theta_{\NY} = 0$, $B = 0$, and minimum existence is determined solely 
by the sign of $C$. Minima can appear in the range $\eta \in (-8, 0)$ where 
$C < 0$, but due to the behavior as $r \to 0$, this results in Type II.

\subsubsection{Type II $\to$ Type I transition (barrier formation)}

The transition from Type II to Type I corresponds to barrier formation in 
the $r \to 0$ direction.

This transition is characterized by a sign change of the gradient of $\Veff(r)$ 
near the origin:
\begin{equation}
  \left.\frac{\dd\Veff}{\dd r}\right|_{r \to 0^+} \quad 
  \begin{cases}
    < 0 & \text{(Type II: rolling)} \\
    > 0 & \text{(Type I: barrier)}
  \end{cases}
\end{equation}

Differentiating the effective potential for $\Sthree$:
\begin{equation}
  \frac{\dd\Veff}{\dd r} \propto 3V^2 r^2 + 2B r + C.
\end{equation}

As $r \to 0$, $\dd\Veff/\dd r \propto C$, so \textbf{the sign of $C$ determines 
Type I/II}:
\begin{itemize}
  \item $C > 0$: $\dd\Veff/\dd r|_{r=0} > 0$ $\to$ Type I
  \item $C < 0$: $\dd\Veff/\dd r|_{r=0} < 0$ $\to$ Type II
\end{itemize}

For $\Sthree$, $C = 9(\eta + 4)^2 - 36$, and $C = 0$ at $\eta = -4 \pm 2$, 
i.e., $\eta = -2$ or $\eta = -6$. These correspond to the I/II boundary positions.

\subsubsection{Type I $\to$ Type II $\to$ Type I re-transition}

The ``Type I $\to$ Type II $\to$ Type I'' re-transition observed for 
$\Sthree$-FULL with $\theta_{\NY} \geq 1$ can be understood from the fact 
that $C(\eta) = 9(\eta + 4)^2 - 36$ is a quadratic function.

$C(\eta)$ takes minimum value $-36$ at $\eta = -4$, and $C = 0$ at 
$\eta = -2$ and $\eta = -6$. Therefore:
\begin{itemize}
  \item $\eta > -2$: $C > 0$ $\to$ Type I (or Type III)
  \item $-6 < \eta < -2$: $C < 0$ $\to$ Type II
  \item $\eta < -6$: $C > 0$ $\to$ Type I
\end{itemize}

This structure produces the Type I $\to$ Type II $\to$ Type I re-transition 
as $\eta$ varies from positive to negative.

%------------------------------------------------------------------------------
\subsection{Origin of Complex Phase Structure in $\Sthree$}
\label{sec:S3_complex}

\begin{figure}[htbp]
  \centering
  \includegraphics[width=1.0\textwidth]{figures/fig13_Phase_Potential_S3.png}
  \caption{Phase-potential correspondence for $\Sthree \times \Sone$ 
           ($\theta_{\NY} = 2$).}
  \label{fig:phase_potential_S3}
\end{figure}

\paragraph{Observed features:}
\begin{enumerate}
  \item For $\theta_{\NY} = 0$, only a band-like Type II region exists, 
        with no Type I
  \item For $\theta_{\NY} > 0$, as shown in Figure~\ref{fig:phase_potential_S3}, 
        Type I bands appear and expand with increasing $\theta_{\NY}$
  \item A ``re-transition'' structure of Type I $\to$ Type II $\to$ Type I 
        is observed for $\theta_{\NY} \geq 1$
  \item Stable radius $r_0$ tends to increase as $\eta$ becomes more negative
  \item $\eta \to -\eta$ symmetry is broken
\end{enumerate}

The reason $\Sthree$ exhibits complex phase structure is due to competition 
between two factors:

\subsubsection{Role of coefficient $B$ ($\theta_{\NY}$ dependence)}

The coefficient of the $r^2$ term, $B = 6V\kappa^2\theta_{\NY}(\eta - 4)$, 
has the following effects for $\theta_{\NY} > 0$:

\begin{itemize}
  \item For $\eta < 4$, $B < 0$: Adds negative contribution to $\Veff(r)$, 
        promoting minimum formation
  \item $|B|$ increases proportionally with $\theta_{\NY}$
\end{itemize}

For $\theta_{\NY} = 0$, $B = 0$, and minimum existence is determined by $C$ alone. 
When $\theta_{\NY} > 0$, the contribution of $B$ is added, and minima can form 
even in regions where $C > 0$. This is the mechanism for ``expansion of stable 
region with increasing $\theta_{\NY}$''.

\subsubsection{Role of coefficient $C$ (Type I/II boundary)}

The coefficient of $r$, $C = 9(\eta + 4)^2 - 36$, governs the behavior as 
$r \to 0$ and determines Type I/II.

Importantly, the factor $(\eta + 4)$ appearing in the expression for $C$ 
reflects coupling with the positive background curvature $R_{\LC} = 24/r^2$ 
of $\Sthree$. As seen in Sec.~\ref{sec:reductions}, $(\eta + 4)$ appears in 
$N_{\REE} = -2V(\eta + 4)/r$, which propagates to $C$.

\subsubsection{Origin of the factor $(\eta - 4)$}

The factor $B \propto (\eta - 4)$ in the FULL variant is directly derived from 
$N_{\FULL} = N_{\TT} - N_{\REE} = 2V(4 - \eta)/r$.

This factor differs from both TT ($B \propto \eta$) and REE ($B \propto \eta + 4$), 
being unique to FULL. Due to $(\eta - 4)$, $B < 0$ over the wide range $\eta < 4$, 
which is the reason FULL has stable regions even for $\eta > 0$.

%------------------------------------------------------------------------------
\subsection{Phase Structure of $\Tthree$: $\theta_{\NY}$ Threshold and $\eta$ Asymmetry}
\label{sec:T3_structure}

\begin{figure}[htbp]
  \centering
  \includegraphics[width=1.0\textwidth]{figures/fig14_Phase_Potential_T3.png}
  \caption{Phase-potential correspondence for $\Tthree \times \Sone$ 
           ($\theta_{\NY} = 2$).}
  \label{fig:phase_potential_T3}
\end{figure}

\paragraph{Observed features:}
\begin{itemize}
  \item $\theta_{\NY} \lesssim 0.9$: Entire region is Type III
  \item $\theta_{\NY} \gtrsim 0.9$: As shown in Figure~\ref{fig:phase_potential_T3}, 
        Type I appears in $\eta < 0$
  \item $\eta \to -\eta$ symmetry is broken (prominent for $\theta_{\NY} > 0$)
\end{itemize}

We analyze the phase structure of $\Tthree$ with the isotropic setting 
$R_1 = R_2 = R_3 = r$.

\subsubsection{Effective potential structure}

The effective potential for $\Tthree$ is:
\begin{equation}
  \Veff(r) \propto r \cdot \left[ V^2 r^2 + B r + C \right],
\end{equation}
where $B = 6V\eta\kappa^2\theta_{\NY}$ and $C = 9\eta^2$. This has the same 
$r^3 + r^2 + r$ structure as $\Sthree$.

\subsubsection{Entire Type III for $\theta_{\NY} = 0$}

For $\theta_{\NY} = 0$, $B = 0$, giving:
\begin{equation}
  \Veff(r) \propto r (V^2 r^2 + 9\eta^2).
\end{equation}

Since $C = 9\eta^2 \geq 0$ and $V^2 > 0$, the bracketed term is always positive, 
and $\Veff(r)$ is monotonically increasing for $r > 0$. Therefore, the entire 
parameter region is Type III (unstable).

This originates from the flatness of $\Tthree$ ($R_{\LC} = 0$): since there 
is no contribution from background curvature, the coefficient $C$ of $r$ 
contains no first-order term in $\eta$, and $C = 9\eta^2$ is always non-negative.

\subsubsection{Type I appearance for $\theta_{\NY} > 0$}

For $\theta_{\NY} > 0$, $B = 6V\eta\kappa^2\theta_{\NY}$ becomes effective.

The condition for minimum existence is that $\dd\Veff/\dd r = 0$ has positive 
solutions:
\begin{equation}
  3V^2 r^2 + 2Br + C = 0.
\end{equation}

The discriminant $D = 4B^2 - 12V^2 C = 4(B^2 - 3V^2 C)$ must be positive, 
and the solutions must be positive:
\begin{equation}
  B < 0 \quad \text{and} \quad B^2 > 3V^2 C.
\end{equation}

$B < 0$ occurs for $\eta < 0$ (when $\theta_{\NY} > 0$), and the second 
condition is:
\begin{equation}
  36V^2\eta^2\kappa^4\theta_{\NY}^2 > 27V^2\eta^2,
\end{equation}
which is satisfied when $\theta_{\NY} > \sqrt{3}/(2\kappa^2) \approx 0.87$ 
(for $\kappa = 1$).

This provides the analytical explanation for the observation ``Type I appears 
at $\theta_{\NY} \approx 0.9$''.

\subsubsection{Breaking of $\eta \to -\eta$ symmetry}

For $\Tthree$, since $B = 6V\eta\kappa^2\theta_{\NY}$ is linear in $\eta$, 
$B \to -B$ under $\eta \to -\eta$.

\begin{itemize}
  \item $\eta < 0$, $\theta_{\NY} > 0$: $B < 0$ (contributes to minimum formation)
  \item $\eta > 0$, $\theta_{\NY} > 0$: $B > 0$ (inhibits minimum formation)
\end{itemize}

Therefore, for $\theta_{\NY} > 0$, stable regions appear only on the $\eta < 0$ 
side, and $\eta \to -\eta$ symmetry is broken.

This asymmetry is also seen in $\Sthree$ ($B \propto (\eta - 4)$) and $\Nilthree$ 
($B \propto (3\eta + 1)$), but for $\Tthree$ where $C = 9\eta^2$ is symmetric 
in $\eta$, it is characteristic that the asymmetry derives purely from the $B$ term.

%------------------------------------------------------------------------------
\subsection{Conditions for Band + Separate Stable Region Structure in $\Nilthree$}
\label{sec:Nil3_structure}

\begin{figure}[htbp]
  \centering
  \includegraphics[width=1.0\textwidth]{figures/fig15_Phase_Potential_Nil3.png}
  \caption{Phase-potential correspondence for $\Nilthree \times \Sone$ 
           ($\theta_{\NY} = 2$).}
  \label{fig:phase_potential_Nil3}
\end{figure}

\paragraph{Observed features:}
\begin{enumerate}
  \item As shown in Figure~\ref{fig:phase_potential_Nil3}, Type II region 
        is limited to a narrow main band at $-0.3 < \eta < 1$
  \item For $\theta_{\NY} > 0$, a separate Type I region appears and expands 
        in the $\eta < 0$ region
  \item $\eta \to -\eta$ symmetry is broken
\end{enumerate}

We analyze why $\Nilthree$ exhibits a narrow stable band and separate stable 
region.

\subsubsection{Origin of the narrow main band}

The effective potential for $\Nilthree$ is:
\begin{equation}
  \Veff(r) \propto r \left[ 4V^2 r^2 + B r + C \right],
\end{equation}
where $C = 36\eta^2 - 24\eta - 9 = 36(\eta - 1/3)^2 - 13$.

The condition for $C$ to be negative is $36(\eta - 1/3)^2 < 13$, i.e.:
\begin{equation}
  \eta \in \left( \frac{1}{3} - \frac{\sqrt{13}}{6}, 
                  \frac{1}{3} + \frac{\sqrt{13}}{6} \right) 
       \approx (-0.27, 0.93).
\end{equation}

Only in this narrow range does $C < 0$, allowing stable minima to form. 
This is the origin of the ``narrow main band''.

\subsubsection{Conditions for separate stable region appearance}

The island appearing in the $\eta < 0$ region for $\theta_{\NY} > 0$ is due 
to the effect of coefficient $B = 8V\kappa^2\theta_{\NY}(3\eta + 1)$.

For $\eta < -1/3$, $B < 0$, and if $|B|$ is sufficiently large, minima can 
form even in regions where $C > 0$. Specifically, if the condition
\begin{equation}
  B^2 > 3V^2 C
\end{equation}
for $3V^2 r^2 + 2Br + C = 0$ to have positive real solutions is satisfied, 
minima exist. As $\theta_{\NY}$ increases, $|B|$ grows, and this condition 
is satisfied over a wider range of $\eta$. This is the mechanism for 
``stable region growing with $\theta_{\NY}$''.

\subsubsection{Breaking of $\eta \to -\eta$ symmetry}

For $\Nilthree$, $C(\eta) = 36\eta^2 - 24\eta - 9$ contains a first-order 
term in $\eta$, so $\eta \to -\eta$ symmetry is broken. This reflects the 
anisotropic structure constants of $\Nilthree$.

%------------------------------------------------------------------------------
\subsection{TT/REE Diagnosis: Role of Curvature Coupling}
\label{sec:TT_REE_diagnosis}

From comparison of FULL, TT, and REE, we identify which contributions drive 
the phase structure.

\subsubsection{$\Sthree$: Dominant role of $N_{\REE}$}

The $r^2$ term coefficients for the 3 variants in $\Sthree$:

\begin{table}[H]
\centering
\begin{tabular}{@{}lll@{}}
\toprule
Variant & Form of $B$ & Range of $\eta$ where $B < 0$ \\
\midrule
FULL & $\propto (\eta - 4)$ & $\eta < 4$ \\
TT & $\propto \eta$ & $\eta < 0$ \\
REE & $\propto (\eta + 4)$ & $\eta < -4$ \\
\bottomrule
\end{tabular}
\end{table}

The factor $(\eta - 4)$ in FULL reflects an additional shift arising from 
the combination of the $\eta$ factor in $N_{\TT}$ and the $(\eta + 4)$ factor 
in $N_{\REE}$: $\eta - (\eta + 4) = -4$.

FULL has the widest stable region because this factor maximizes the range 
of $\eta$ where $B < 0$ (the entire region $\eta < 4$).

\subsubsection{$\Tthree$: Disappearance of curvature contribution and role of NY term}

For $\Tthree$, since background curvature $R_{\LC} = 0$, the curvature coupling 
term from $N_{\REE}$ vanishes. As a result:
\begin{itemize}
  \item $N_{\FULL} = N_{\REE} = -2V\eta/r$ (completely identical)
  \item TT differs by a factor of 2, but phase boundary positions are similar
\end{itemize}

With the isotropic setting, the NY term contributes proportionally to $r^2$, 
so it contributes to minimum formation for $\theta_{\NY} > 0$. Unlike $\Sthree$, 
the coefficient $C = 9\eta^2$ of $r$ is always non-negative, so Type I appearance 
requires $\theta_{\NY}$ to exceed the threshold ($\approx 0.9$).

The similarity of phase structure across the 3 variants is because the lack 
of curvature coupling limits the differences between variants to coefficient 
differences only.

\subsubsection{General role of curvature coupling}

From the above comparisons, it is suggested that the $\theta_{\NY}$ dependence 
of phase boundaries is primarily driven by \textbf{coupling with background 
curvature} through $N_{\REE}$:

\begin{table}[H]
\centering
\begin{tabular}{@{}llll@{}}
\toprule
Topology & Background curvature $R_{\LC}$ & Curvature coupling strength & 
$\theta_{\NY}$ sensitivity \\
\midrule
$\Sthree$ & $+24/r^2$ (positive) & Strong & High \\
$\Tthree$ & $0$ (flat) & None & Moderate (threshold exists) \\
$\Nilthree$ & $-1/(2r^2)$ (negative) & Moderate & Moderate \\
\bottomrule
\end{tabular}
\end{table}

The larger the background curvature, the more prominent the contribution of 
$N_{\REE}$, and the higher the sensitivity of phase structure to $\theta_{\NY}$.

\textbf{Note:} Although $\Tthree$ has zero background curvature, with the 
isotropic setting $R_1 = R_2 = R_3 = r$, the NY term contributes to the 
effective potential in a form proportional to $r^2$, giving rise to 
$\theta_{\NY}$ sensitivity. However, since the coefficient $C = 9\eta^2$ of 
$r$ is always non-negative, a threshold of $\theta_{\NY} > 0.87$ is required 
for Type I appearance.
  % Mechanisms and interpretation
%==============================================================================
% Section 6: Representative Points and Stability Quality
%==============================================================================
\section{Representative Points and Stability Quality}
\label{sec:representative}

This section presents the specific shapes of the effective potential $\Veff(r)$ 
and stability metrics at representative parameter points on the phase diagrams. 
This clarifies how the operational definitions of Type I/II/III 
(Sec.~\ref{sec:type_definition}) correspond to actual potential shapes.

%------------------------------------------------------------------------------
\subsection{Selection Criteria for Representative Points}
\label{sec:selection_criteria}

For each topology, representative points are selected according to the 
following criteria:

\begin{enumerate}
  \item \textbf{Type I representative}: Points showing metastable wells with 
        barriers. $\Delta V > 0$ and $\dd\Veff/\dd r|_{r \to 0^+} > 0$
  \item \textbf{Type II representative}: Points showing rolling. Minimum exists 
        but no barrier in $r \to 0$ direction
  \item \textbf{Type III representative}: Points showing instability. No stable 
        minimum within allowed region
  \item \textbf{Near phase boundary}: Points in parameter regions where Type 
        transitions occur
\end{enumerate}

Below, we fix $\kappa = L = 1$ and focus on results for the FULL variant.

%------------------------------------------------------------------------------
\subsection{Representative Points for $\Sthree \times \Sone$}
\label{sec:S3_representatives}

\subsubsection{List of representative points}

\begin{table}[H]
\centering
\caption{$\Sthree$-FULL representative points and stability metrics.}
\label{tab:S3_representatives}
\begin{tabular}{@{}ccccccc@{}}
\toprule
Point & $(V, \eta, \theta_{\NY})$ & Type & $r_0$ & $\log_{10}(r_0)$ & 
$\Delta V$ & $\log_{10}(\Delta V)$ \\
\midrule
S1 & $(2.0, -4.0, 1.0)$ & II & $1.26$ & $0.10$ & --- & --- \\
S2 & $(2.0, -7.0, 1.0)$ & I & $3.98$ & $0.60$ & $1.2 \times 10^3$ & $3.1$ \\
S3 & $(2.0, 0.0, 1.0)$ & I & $0.32$ & $-0.50$ & $2.1 \times 10^2$ & $2.3$ \\
S4 & $(2.0, 1.0, 1.0)$ & III & --- & --- & --- & --- \\
S5 & $(2.0, -2.0, 1.0)$ & I/II boundary & $0.63$ & $-0.20$ & $\approx 0$ & --- \\
S6 & $(2.0, -4.0, 2.0)$ & II & $2.51$ & $0.40$ & --- & --- \\
\bottomrule
\end{tabular}
\end{table}

\subsubsection{Characteristics of each representative point}

\paragraph{Point S1 (Type II, $\eta = -4$):}
This is the point where $C = 9(\eta + 4)^2 - 36 = -36$ takes its minimum value. 
$\Veff(r)$ starts from negative values as $r \to 0$, monotonically increases, 
passes through a minimum, and increases again. Near the origin, 
$\dd\Veff/\dd r < 0$, so no barrier exists.

\paragraph{Point S2 (Type I, $\eta = -7$):}
$C = 9(-7 + 4)^2 - 36 = 45 > 0$. $\Veff(r)$ starts from positive values as 
$r \to 0$, forms a barrier before falling into the well. Barrier height 
$\Delta V \approx 10^3$ suggests strong suppression of quantum tunneling 
from the metastable state.

\paragraph{Point S3 (Type I, $\eta = 0$):}
$C = 9(4)^2 - 36 = 108 > 0$. Due to the effect of $\theta_{\NY} = 1$, 
$B = 6V\kappa^2\theta_{\NY}(\eta - 4) = -48 < 0$, and a minimum forms even 
at $\eta = 0$. This is an example where a region that was Type III at 
$\theta_{\NY} = 0$ transitions to Type I at $\theta_{\NY} > 0$.

\paragraph{Point S4 (Type III, $\eta = 1$):}
$C = 9(5)^2 - 36 = 189 > 0$, but the effect of $B = -36$ is insufficient, 
and no minimum forms within the allowed region. Search reaches $r_{\max}$.

\paragraph{Point S5 (I/II boundary, $\eta = -2$):}
This is the critical point where $C = 9(2)^2 - 36 = 0$. 
$\dd\Veff/\dd r|_{r=0} = 0$, positioned at the boundary between Type I and Type II.

%------------------------------------------------------------------------------
\subsection{Representative Points for $\Tthree \times \Sone$}
\label{sec:T3_representatives}

\subsubsection{List of representative points}

\begin{table}[H]
\centering
\caption{$\Tthree$-FULL representative points and stability metrics.}
\label{tab:T3_representatives}
\begin{tabular}{@{}ccccccc@{}}
\toprule
Point & $(V, \eta, \theta_{\NY})$ & Type & $r_0$ & $\log_{10}(r_0)$ & 
$\Delta V$ & $\log_{10}(\Delta V)$ \\
\midrule
T1 & $(2.0, -3.0, 1.0)$ & I & $2.5$ & $0.40$ & $4.2 \times 10^2$ & $2.6$ \\
T2 & $(2.0, -5.0, 1.5)$ & I & $5.0$ & $0.70$ & $2.1 \times 10^3$ & $3.3$ \\
T3 & $(2.0, 0.0, 1.0)$ & III & --- & --- & --- & --- \\
T4 & $(2.0, 2.0, 1.0)$ & III & --- & --- & --- & --- \\
T5 & $(2.0, -2.0, 0.5)$ & III & --- & --- & --- & --- \\
\bottomrule
\end{tabular}
\end{table}

\textbf{Note:} For $\theta_{\NY} < 0.87$, the entire parameter region is Type III. 
Type I appears only for $\theta_{\NY} \geq 0.87$ and $\eta < 0$.

\subsubsection{Characteristics of each representative point}

\paragraph{Points T1, T2 (Type I, $\eta < 0$, $\theta_{\NY} \geq 1$):}
With the isotropic setting for $\Tthree$, Type I (metastable well with barrier) 
forms when $\theta_{\NY}$ exceeds the threshold ($\approx 0.87$) and $\eta < 0$. 
The more negative $\eta$ and the larger $\theta_{\NY}$, the larger $r_0$ and 
$\Delta V$ become.

\paragraph{Points T3, T4 (Type III, $\eta \geq 0$ or small $\theta_{\NY}$):}
In the region $\eta \geq 0$, or when $\theta_{\NY} < 0.87$, 
$B = 6V\eta\kappa^2\theta_{\NY}$ does not contribute to minimum formation 
($B \geq 0$ or $B$'s effect is weak), resulting in Type III.

\paragraph{Point T5 (Type III, $\theta_{\NY}$ below threshold):}
Even with $\eta = -2 < 0$, at $\theta_{\NY} = 0.5 < 0.87$, the Type I condition 
$B^2 > 3V^2 C$ is not satisfied, resulting in Type III. This is numerical 
confirmation of the threshold condition derived in Sec.~\ref{sec:T3_structure}.

\subsubsection{Confirmation of $\eta \to -\eta$ asymmetry}

With the isotropic setting for $\Tthree$, since $B = 6V\eta\kappa^2\theta_{\NY}$ 
is linear in $\eta$, $\eta \to -\eta$ symmetry is broken for $\theta_{\NY} > 0$. 
In Table~\ref{tab:T3_representatives}:

\begin{itemize}
  \item $\eta < 0$ (points T1, T2): $B < 0$, Type I possible
  \item $\eta > 0$ (point T4): $B > 0$, Type III
\end{itemize}

This asymmetry is visualized in Figure~\ref{fig:phase_T3}, where stable regions 
appear only on the $\eta < 0$ side. In the limit $\theta_{\NY} = 0$, $B = 0$ 
and symmetry is restored, but in this case the entire region is Type III.

%------------------------------------------------------------------------------
\subsection{Representative Points for $\Nilthree \times \Sone$}
\label{sec:Nil3_representatives}

\subsubsection{List of representative points}

\begin{table}[H]
\centering
\caption{$\Nilthree$-FULL representative points and stability metrics.}
\label{tab:Nil3_representatives}
\begin{tabular}{@{}ccccccc@{}}
\toprule
Point & $(V, \eta, \theta_{\NY})$ & Type & $r_0$ & $\log_{10}(r_0)$ & 
$\Delta V$ & $\log_{10}(\Delta V)$ \\
\midrule
N1 & $(2.0, 0.5, 1.0)$ & II & $0.20$ & $-0.70$ & --- & --- \\
N2 & $(2.0, -5.0, 1.0)$ & I & $2.51$ & $0.40$ & $8.9 \times 10^2$ & $2.95$ \\
N3 & $(2.0, -5.0, 2.0)$ & I & $5.01$ & $0.70$ & $3.5 \times 10^3$ & $3.55$ \\
N4 & $(2.0, 2.0, 1.0)$ & III & --- & --- & --- & --- \\
N5 & $(2.0, -0.3, 1.0)$ & II/III boundary & $\approx 0.01$ & $\approx -2$ & --- & --- \\
\bottomrule
\end{tabular}
\end{table}

\subsubsection{Characteristics of each representative point}

\paragraph{Point N1 (Type II, within main band):}
$\eta = 0.5$ is in the region where $C = 36(0.5 - 1/3)^2 - 13 \approx -12 < 0$, 
located within the main band. $r_0$ is small ($\log_{10}(r_0) \approx -0.7$), 
consistent with the main band in the phase diagram being displayed in 
purple-blue colors.

\paragraph{Points N2, N3 (Type I, separate stable region):}
At $\eta = -5$, $C = 36(-5 - 1/3)^2 - 13 \approx 1010 > 0$, but 
$B = 8V\kappa^2\theta_{\NY}(3\eta + 1) = -224\theta_{\NY} < 0$ is sufficiently 
large to form a minimum. Increasing $\theta_{\NY}$ from 1 to 2 approximately 
doubles $r_0$ and quadruples $\Delta V$. This shows that the separate stable 
region expands and deepens with increasing $\theta_{\NY}$.

\paragraph{Point N4 (Type III, outside main band):}
At $\eta = 2$, $C = 36(2 - 1/3)^2 - 13 \approx 87 > 0$, and 
$B = 8V\kappa^2\theta_{\NY}(7) = 112 > 0$. Since $B > 0$, minimum formation 
is inhibited, resulting in Type III.

\paragraph{Point N5 (II/III boundary):}
$\eta \approx -0.27$ is close to the critical value where $C = 0$. A minimum 
exists but $r_0 \approx r_{\min}$, barely determined as Type II by boundary 
criteria.

%------------------------------------------------------------------------------
\subsection{Visualization of Stability Metrics}
\label{sec:metrics_visualization}

\subsubsection{Distribution of $\log_{10}(r_0)$}

The color gradient in phase diagrams (purple $\to$ yellow) represents 
$\log_{10}(r_0)$. Typical ranges for each topology:

\begin{table}[H]
\centering
\begin{tabular}{@{}lll@{}}
\toprule
Topology & $\log_{10}(r_0)$ range & Characteristics \\
\midrule
$\Sthree$ & $[-0.5, 3.0]$ & Increases in negative $\eta$ direction \\
$\Tthree$ & $[0, 5]$ & Proportional to $|\eta|/V$ \\
$\Nilthree$ & $[-2, 3]$ & Small in main band, large in separate region \\
\bottomrule
\end{tabular}
\end{table}

\subsubsection{Distribution of $\log_{10}(\Delta V)$ (Type I only)}

White contour lines in phase diagrams represent $\log_{10}(\Delta V)$. 
Typical barrier heights:

\begin{itemize}
  \item \textbf{$\Sthree$}: $\log_{10}(\Delta V) \in [2, 5]$ in Type I region. 
        Increases with $|\eta|$
  \item \textbf{$\Tthree$}: $\log_{10}(\Delta V) \in [2, 4]$ in Type I region 
        with $\theta_{\NY} \geq 0.87$ and $\eta < 0$. Increases with $|\eta|$ 
        and $\theta_{\NY}$
  \item \textbf{$\Nilthree$}: $\log_{10}(\Delta V) \in [2, 5]$ in separate 
        stable region. Increases with $\theta_{\NY}$
\end{itemize}

\subsubsection{Correlation between barrier height and stable radius}

From the contour patterns in phase diagrams, the following trends can be read:

\begin{enumerate}
  \item \textbf{$\Sthree$}: $\log_{10}(\Delta V)$ and $\log_{10}(r_0)$ show 
        positive correlation. Points with larger $r_0$ have deeper wells 
        and higher barriers
  \item \textbf{$\Tthree$, $\Nilthree$}: Similar positive correlation observed 
        in stable regions with $\eta < 0$
\end{enumerate}

This correlation can be understood from the scaling law 
$\Veff \propto r \times P(r)$ of the effective potential. As $r_0$ increases, 
the region $r < r_0$ forming the barrier also widens, consequently increasing 
$\Delta V$.

Scaling laws for each topology are shown in 
Figures~\ref{fig:scaling_S3}--\ref{fig:scaling_Nil3}.

\begin{figure}[tbp]
  \centering
  \includegraphics[width=0.9\textwidth]{figures/fig16_Scaling_laws_S3.png}
  \caption{Scaling laws: $\Sthree \times \Sone$ ($\eta = -7.0$, $\theta_{\NY} = 2.0$).}
  \label{fig:scaling_S3}
\end{figure}

\begin{figure}[tbp]
  \centering
  \includegraphics[width=0.9\textwidth]{figures/fig17_Scaling_laws_T3.png}
  \caption{Scaling laws: $\Tthree \times \Sone$ ($\eta = -7.0$, $\theta_{\NY} = 2.0$).}
  \label{fig:scaling_T3}
\end{figure}

\begin{figure}[tbp]
  \centering
  \includegraphics[width=0.9\textwidth]{figures/fig18_Scaling_laws_Nil3.png}
  \caption{Scaling laws: $\Nilthree \times \Sone$ ($\eta = -7.0$, $\theta_{\NY} = 2.0$).}
  \label{fig:scaling_Nil3}
\end{figure}

%------------------------------------------------------------------------------
\subsection{Consistency Check Between Analytical Boundaries and Numerical Scan}
\label{sec:consistency_check}

\subsubsection{Type I/II boundary for $\Sthree$}

The analytical boundary condition $C(\eta) = 9(\eta + 4)^2 - 36 = 0$ derived 
in Sec.~\ref{sec:type_transitions} gives $\eta = -2$ and $\eta = -6$.

Comparison with numerical scan results:

\begin{table}[H]
\centering
\begin{tabular}{@{}llll@{}}
\toprule
Boundary & Analytical value & Numerical scan result ($V = 2$, $\theta_{\NY} = 1$) & Deviation \\
\midrule
Upper I/II & $\eta = -2$ & $\eta \approx -2.0$ & $< 0.1$ \\
Lower II/I & $\eta = -6$ & $\eta \approx -5.9$ & $\approx 0.1$ \\
\bottomrule
\end{tabular}
\end{table}

Deviations are on the order of scan grid spacing ($\Delta\eta = 0.1$), showing 
good agreement between analytical predictions and numerical results.

\subsubsection{Type I/III boundary for $\Tthree$}

For $\Tthree$ with isotropic setting, the boundary between Type I and Type III 
is determined by the condition derived in Sec.~\ref{sec:T3_structure}:
\begin{equation}
  \theta_{\NY} = \frac{\sqrt{3}}{2\kappa^2} \approx 0.87 \quad (\kappa = 1).
\end{equation}

Comparison with numerical scan results:

\begin{table}[H]
\centering
\begin{tabular}{@{}llll@{}}
\toprule
Boundary & Analytical value & Numerical scan result ($V = 2$, $\eta = -3$) & Deviation \\
\midrule
$\theta_{\NY}$ threshold & $0.87$ & $\theta_{\NY} \approx 0.9$ & $\approx 0.03$ \\
\bottomrule
\end{tabular}
\end{table}

Additionally, the $\eta = 0$ axis is the boundary where $C = 0$, and for 
$\theta_{\NY} > 0.87$, the Type I/III boundary is observed near this location. 
Type I appears only on the $\eta < 0$ side, with the entire $\eta > 0$ side 
being Type III.

\subsubsection{Main band boundary for $\Nilthree$}

The $C(\eta) = 0$ condition derived in Sec.~\ref{sec:Nil3_structure} gives 
$\eta \approx -0.27$ and $\eta \approx 0.93$.

Comparison with numerical scan results:

\begin{table}[H]
\centering
\begin{tabular}{@{}llll@{}}
\toprule
Boundary & Analytical value & Numerical scan result ($\theta_{\NY} = 0$) & Deviation \\
\midrule
Lower & $\eta \approx -0.27$ & $\eta \approx -0.3$ & $\approx 0.03$ \\
Upper & $\eta \approx 0.93$ & $\eta \approx 1.0$ & $\approx 0.07$ \\
\bottomrule
\end{tabular}
\end{table}

Deviations are on the order of grid spacing, confirming consistency.

%------------------------------------------------------------------------------
\subsection{Summary of This Section}
\label{sec:representative_summary}

This section quantitatively presented stability metrics at representative 
points for each topology and confirmed the following:

\begin{enumerate}
  \item \textbf{Operational definition of Type classification} corresponds 
        consistently with actual $\Veff(r)$ shapes
  \item \textbf{$\log_{10}(r_0)$ and $\log_{10}(\Delta V)$} show systematic 
        patterns on phase diagrams with clear correlation to geometric parameters
  \item \textbf{Analytical boundary conditions} (derived in 
        Sec.~\ref{sec:mechanisms}) and \textbf{numerical scan results} are 
        consistent within grid resolution
\end{enumerate}

These representative point data can be used as reference points in EC+NY 
minisuperspace analysis, and as input for future dynamical analysis 
(WKB tunneling rate calculations, etc.).
  % Representative points and stability quality
%==============================================================================
% Section 7: Geometric Interpretation and Utility as a Tool
%==============================================================================
\section{Geometric Interpretation and Utility as a Tool}
\label{sec:interpretation}

This section interprets the numerical results obtained in Secs.~\ref{sec:numerical}--\ref{sec:representative} 
from a geometric perspective and discusses the applicability and limitations 
of the diagnostic framework provided by this work.

%------------------------------------------------------------------------------
\subsection{Geometric Factors of Topology Dependence}
\label{sec:geometric_factors}

\subsubsection{Correspondence between structure constants and phase structure}

We organize the correspondence between the characteristics of structure 
constants $C^{i}{}_{jk}$ for the three test beds and the observed phase structure.

\begin{table}[H]
\centering
\caption{Correspondence between structure constants and phase structure.}
\label{tab:structure_phase}
\begin{tabularx}{\textwidth}{@{}l>{\raggedright\arraybackslash}X l>{\raggedright\arraybackslash}X@{}}
\toprule
Topology & Structure constant characteristics & Lie algebra & Phase structure characteristics \\
\midrule
$\Sthree$ & $C^{i}{}_{jk} = (4/r)\varepsilon_{ijk}$ (totally antisym.) & 
            $\mathfrak{su}(2)$ (simple) & Complex, high $\theta_{\NY}$ sensitivity \\
$\Tthree$ & $C^{i}{}_{jk} = 0$ (all zero) & Abelian & 
            Threshold-dependent, activates at $\theta_{\NY} > 0.87$ \\
$\Nilthree$ & $C^{2}{}_{01} = -1/r$, $C^{2}{}_{10} = +1/r$, others 0 & 
              Heisenberg (nilpotent) & Intermediate, band + separate region \\
\bottomrule
\end{tabularx}
\end{table}

From this correspondence, it is suggested that the ``complexity'' of structure 
constants (number of nonzero components and symmetry) correlates with the 
complexity of phase structure.

\subsubsection{Role of background curvature}

The background Ricci scalar $R_{\LC}$ from the Levi--Civita connection affects 
the effective potential through the $N_{\REE}$ component of the NY density.

\paragraph{Positive curvature ($\Sthree$): $R_{\LC} = 24/r^2 > 0$}

The factor $(\eta + 4)$ appearing in $N_{\REE} = -2V(\eta + 4)/r$ reflects 
coupling with positive curvature. This factor breaks $\eta \to -\eta$ symmetry 
and becomes a driving force for forming a wide stable region in the $\eta < 0$ 
region.

As a physical interpretation, in the positive curvature background, the 
coupling of the axial component ($\eta$) of torsion with curvature produces 
an ``effective negative contribution,'' promoting potential well formation.

\paragraph{Zero curvature ($\Tthree$): $R_{\LC} = 0$}

Since the background curvature is zero, the curvature coupling term in 
$N_{\REE}$ vanishes, and $N_{\FULL} = N_{\REE} = -2V\eta/r$. However, with 
the isotropic setting $R_1 = R_2 = R_3 = r$, this NY term contributes as the 
$r^2$ coefficient in the effective potential, giving rise to $\theta_{\NY}$ 
sensitivity.

A characteristic of $\Tthree$ is that the coefficient $C = 9\eta^2$ of $r$ 
is always non-negative. Therefore, Type I appearance requires $\theta_{\NY}$ 
to exceed a threshold, with ``activation conditions'' being stricter compared 
to $\Sthree$ or $\Nilthree$.

$\eta \to -\eta$ symmetry is preserved for $\theta_{\NY} = 0$ but broken for 
$\theta_{\NY} > 0$ due to sign reversal of $B = 6V\eta\kappa^2\theta_{\NY}$. 
This breaking originates not from curvature coupling but from the projection 
of the NY term onto the isotropic volume mode.

\paragraph{Negative curvature ($\Nilthree$): $R_{\LC} = -1/(2r^2) < 0$}

The factor $(1 - 3\eta)$ appearing in $N_{\REE} = 2V(1 - 3\eta)/(3r)$ 
reflects coupling with negative curvature. Compared to the $(\eta + 4)$ of 
$\Sthree$, the shift is smaller ($+1/3$ vs $+4$), and the symmetry-breaking 
effect is moderate.

The main band being limited to $\eta \in (-0.27, 0.93)$ originates from 
$\Nilthree$'s structure constants being nonzero ``in only one direction.'' 
This anisotropy makes the quadratic coefficient of $C(\eta)$ larger than 
$\Sthree$'s ($36$ vs $9$), consequently narrowing the range of $\eta$ where 
$C < 0$.

\subsubsection{Isotropy and anisotropy}

$\Sthree$ (SU(2)) is an isotropic homogeneous space where any two directions 
can be mapped to each other by group action. On the other hand, $\Nilthree$ 
(Heisenberg) is anisotropic, with the structure $[E_0, E_1] = (1/R)E_2$ 
giving special treatment to the ``$E_2$ direction.''

This anisotropy manifests in the effective potential as follows:
\begin{itemize}
  \item $\Sthree$: $C(\eta) = 9(\eta + 4)^2 - 36$ (large $\eta$ shift)
  \item $\Nilthree$: $C(\eta) = 36(\eta - 1/3)^2 - 13$ (small $\eta$ shift, 
        large coefficient)
\end{itemize}

For anisotropic topologies, the ``allowed range'' of torsion parameters 
tends to be narrower than in isotropic cases.

%------------------------------------------------------------------------------
\subsection{Role of FULL Revealed by Diagnostic Comparison (TT/REE)}
\label{sec:FULL_role}

\subsubsection{Physical meaning of TT and REE}

In the decomposition $N = N_{\TT} - N_{\REE}$ of the NY density, each component 
carries different geometric information:

\paragraph{$N_{\TT}$ (torsion-torsion term):}
\begin{equation}
  N_{\TT} = \frac{1}{4}\varepsilon^{abcd} T^{e}{}_{ab} T_{ecd}
\end{equation}

This represents ``self-interaction'' of torsion and does not depend on 
background curvature. It takes the form $N_{\TT} \propto V\eta/r$ and is 
an odd function of $\eta$.

\paragraph{$N_{\REE}$ (Riemann-torsion term):}
\begin{equation}
  N_{\REE} = \frac{1}{4}\varepsilon^{abcd} R_{abcd}
\end{equation}

This is the contribution of curvature from the EC connection (including torsion) 
and contains coupling with background curvature. It has different $\eta$ 
dependence for each topology:
\begin{itemize}
  \item $\Sthree$: $N_{\REE} \propto V(\eta + 4)/r$
  \item $\Tthree$: $N_{\REE} \propto V\eta/r$
  \item $\Nilthree$: $N_{\REE} \propto V(1 - 3\eta)/r$
\end{itemize}

\subsubsection{Reason why FULL is optimal}

From the comparison in Figure~\ref{fig:phase_matrix}, FULL was observed to 
have the widest stable region for $\Sthree$. This is due to the following 
mechanism:

The $r^2$ term coefficient in FULL, $B \propto (\eta - 4)$, arises as the 
``difference'' between TT's $B \propto \eta$ and REE's $B \propto (\eta + 4)$:
\begin{equation}
  (\eta - 4) = \eta - (\eta + 4) + 2\eta = 2\eta - 4.
\end{equation}

This combination maximizes the range of $\eta$ where $B < 0$ (the entire 
region $\eta < 4$).

Physically, $N_{\TT}$ and $N_{\REE}$ ``compete with opposite signs,'' 
realizing a wide stable region that cannot be achieved individually. 
This suggests that the exact derivative structure $N = \dd(e^a \wedgep T_a)$ 
of the NY term carries geometric information different from simple torsion 
squared terms or curvature terms.

\subsubsection{Degeneracy and threshold effect in $\Tthree$}

For $\Tthree$, $N_{\FULL} = N_{\REE} = -2V\eta/r$, meaning FULL and REE 
coincide completely. This is because background curvature is zero, causing 
the curvature-dependent part of $N_{\TT}$ and $N_{\REE}$ to vanish, with 
their difference becoming a constant multiple.

However, with the isotropic setting, the NY term contributes to the effective 
potential as the $r^2$ coefficient $B$, so for all 3 variants, Type I appears 
when $\theta_{\NY}$ exceeds the threshold $> 0.87$. TT has twice the coefficient 
of FULL/REE, which may slightly shift the threshold, but the qualitative 
phase boundary structure is similar.

%------------------------------------------------------------------------------
\subsection{Usage as a Reusable Diagnostic Tool}
\label{sec:diagnostic_tool}

\subsubsection{Input and output specifications}

The diagnostic framework provided by this work has the following input/output 
specifications:

\paragraph{Input:}
\begin{itemize}
  \item Topology: $\mathcal{M}_3 \in \{\Sthree, \Tthree, \Nilthree\}$ 
        (or specified by structure constants $C^{i}{}_{jk}$)
  \item Parameters: $(V, \eta, \theta_{\NY})$
  \item NY variant: FULL / TT / REE
\end{itemize}

\paragraph{Output:}
\begin{itemize}
  \item Type classification: I (stable with barrier) / II (rolling) / 
        III (unstable)
  \item Stable radius: $r_0$ (for Type I/II)
  \item Barrier height: $\Delta V$ (for Type I)
  \item Effective potential: Analytical expression for $\Veff(r)$
\end{itemize}

\subsubsection{Application example: Use for parameter design}

When constructing cosmological models based on EC+NY theory, this diagnostic 
framework can be used for the following purposes:

\paragraph{Identification of stable regions:}
For a given topology and NY variant, read the parameter range realizing 
Type I or Type II from phase diagrams.

\paragraph{Prediction of critical conditions:}
Use the analytical boundary conditions derived in Sec.~\ref{sec:mechanisms} 
(e.g., $\eta = -2, -6$ for $\Sthree$) to predict parameters where Type 
transitions occur.

\paragraph{Evaluation of stability quality:}
Quantitatively evaluate the ``robustness'' of metastable states from 
$\log_{10}(\Delta V)$ values. Larger $\Delta V$ means stronger suppression 
of decay by quantum tunneling.

\subsubsection{Extension to other topologies}

This diagnostic framework can be extended to any 3-dimensional Lie group 
admitting left-invariant coframes. Specifically:

\begin{enumerate}
  \item Specify structure constants $C^{i}{}_{jk}$
  \item Execute the computational pipeline from Sec.~\ref{sec:pipeline}
  \item Derive effective potential $\Veff(r)$
  \item Compute Type classification and stability metrics
\end{enumerate}

Among Thurston's eight geometries, those with compact quotients 
($\Sthree$, $\mathbb{E}^3$, $\mathrm{Nil}$, $\mathrm{Sol}$, 
$\widetilde{SL_2(\mathbb{R})}$, $\mathbb{H}^2 \times \mathbb{R}$, 
$S^2 \times \mathbb{R}$) can in principle be treated with this framework. 
However, $S^2 \times \mathbb{R}$ is not parallelizable, requiring caution 
in constructing a global coframe.
  % Interpretation and utility
%==============================================================================
% Section 8: Conclusions and Outlook
%==============================================================================
\section{Conclusions and Outlook}
\label{sec:conclusions}

%------------------------------------------------------------------------------
\subsection{Summary of Results}
\label{sec:summary}

In this work, we studied Einstein--Cartan gravity supplemented with the Nieh--Yan 
term (EC+NY) in a Euclidean-signature minisuperspace framework and performed 
phase classification based on the shape of the effective potential $\Veff(r)$. 
Using three homogeneous spaces---$\Sthree$ (SU(2)), $\Tthree$ (flat), and 
$\Nilthree$ (Heisenberg)---as test beds for the spatial section, we systematically 
compared topology dependence under the same reduction procedure.

The main results are summarized below.

\subsubsection{Establishment of Type classification}

Based on the shape of the effective potential, we operationally defined the 
following three types:

\begin{itemize}
  \item \textbf{Type I} (metastable well with barrier): Local minimum exists 
        with barrier in the $r \to 0$ direction
  \item \textbf{Type II} (rolling): Local minimum exists but no barrier in 
        the $r \to 0$ direction
  \item \textbf{Type III} (unstable): No stable local minimum within the 
        allowed region
\end{itemize}

This classification is based on clear criteria---the extremum condition of 
$\Veff(r)$ and the gradient near the origin---and can be systematically 
applied through numerical scanning.

\subsubsection{Phase structure by topology}

Qualitatively different phase structures were observed for each topology:

\paragraph{$\Sthree \times \Sone$:}
Exhibits the most complex phase structure. For $\theta_{\NY} = 0$, only a 
band-like Type II region exists, but for $\theta_{\NY} > 0$, Type I regions 
appear and the stable region expands significantly with increasing $\theta_{\NY}$. 
Varying $\eta$ from positive to negative produces a re-transition of 
Type I $\to$ Type II $\to$ Type I.

\paragraph{$\Tthree \times \Sone$:}
Exhibits threshold-dependent phase structure. The entire parameter region 
is Type III (unstable) for $\theta_{\NY} < 0.87$, but Type I appears in the 
$\eta < 0$ region for $\theta_{\NY} \geq 0.87$. $\eta \to -\eta$ symmetry 
is broken for $\theta_{\NY} > 0$, with stable regions forming only on the 
$\eta < 0$ side.

\paragraph{$\Nilthree \times \Sone$:}
Exhibits intermediate complexity. The Type II region is limited to a narrow 
main band at $\eta \in (-0.3, 1.0)$, and for $\theta_{\NY} > 0$, a separate 
Type I region appears in the $\eta < 0$ region.

\subsubsection{Geometric origin of phase structure}

The following geometric factors were identified as origins of topology dependence:

\paragraph{Coupling with background curvature:}
Background Ricci scalar $R_{\LC}$ affects the effective potential through the 
$N_{\REE}$ component of the NY density. $\Sthree$ (positive curvature) and 
$\Nilthree$ (negative curvature) have continuous $\theta_{\NY}$ sensitivity 
due to curvature coupling. $\Tthree$ (zero curvature) lacks curvature coupling 
and shows threshold-type response (rapid activation at $\theta_{\NY} > 0.87$).

\paragraph{Symmetry breaking:}
$\eta \to -\eta$ symmetry is preserved for all topologies when $\theta_{\NY} = 0$ 
but broken for $\theta_{\NY} > 0$. For $\Sthree$ and $\Nilthree$, curvature 
coupling terms are the origin; for $\Tthree$, the projection of the NY term 
onto the isotropic volume mode is the origin of symmetry breaking.

\paragraph{Complexity of structure constants:}
The more complex the structure constants of the Lie algebra (more nonzero 
components, higher symmetry), the more complex the phase structure tends to be.

\subsubsection{Elucidation of FULL's role through TT/REE diagnosis}

By decomposing the NY density into $N_{\TT}$ (torsion-torsion) and $N_{\REE}$ 
(Riemann-torsion), we diagnosed the role of each contribution. FULL has the 
widest stable region in $\Sthree$ because $N_{\TT}$ and $N_{\REE}$ ``compete 
with opposite signs,'' maximizing the range of $\eta$ where the $r^2$ term 
coefficient $B < 0$. For $\Tthree$, the $N_{\REE}$ component degenerates, 
weakening the NY term effect.

%------------------------------------------------------------------------------
\subsection{Limitations of This Work}
\label{sec:limitations_detail}

We make explicit the limitations to keep in mind when interpreting the results 
of this work.

\subsubsection{Constraints of minisuperspace truncation}

This work is based on minisuperspace reduction assuming spatial homogeneity. 
This truncation discards the following physics:

\paragraph{Inhomogeneous modes:}
Spatially varying field configurations (gravitational waves, density 
perturbations, etc.) are not included. In actual cosmological scenarios, 
these perturbations may affect stability.

\paragraph{Local structure:}
The minisuperspace ansatz assumes global symmetry, so formation of local 
defects (cosmic strings, domain walls, etc.) cannot be described.

\paragraph{Restriction of dynamical degrees of freedom:}
Since only scale variable $r$ is taken as dynamical variable, anisotropic 
deformations (squashing, etc.) are not considered.

\subsubsection{Limitations of classical analysis}

The Type classification in this work is based on the classical effective 
potential. The following quantum effects are not considered:

\paragraph{Quantum tunneling:}
Tunneling probability through Type I barriers is evaluated by WKB approximation 
as $\Gamma \propto \exp(-B)$ (where $B$ is the bounce action), but this work 
only shows barrier height $\Delta V$ values.

\paragraph{One-loop corrections:}
Corrections to the effective potential from quantum fluctuations of fields 
are not included. In particular, contributions from high-energy modes may 
become important near $r \to 0$.

\paragraph{Renormalization group running:}
Scale dependence of the coupling constant $\theta_{\NY}$ is not considered.

\subsubsection{Constraints of torsion ansatz}

The torsion ansatz (T1 + T2) adopted in this work includes only 2 components 
(axial and vector trace) of the 4-dimensional torsion irreducible decomposition. 
The complete irreducible decomposition includes 3 components (tensor, vector, 
axial), and more general ansatz would show additional parameters and phase 
structure.

Also, torsion parameters $\eta$, $V$ are assumed to be spatially uniform, 
so non-uniform torsion configurations cannot be treated in this framework.

\subsubsection{Connection to Lorentzian signature}

This work performs calculations in Euclidean signature $(+,+,+,+)$. 
Connection to real-time cosmology requires Wick rotation $\tau \to it$, 
with the following points requiring attention:

\begin{itemize}
  \item Minima of effective potential correspond to ``classical turning points'' 
        in Lorentzian signature
  \item Type I barriers define ``tunneling regions'' in real time
  \item Type II rolling corresponds to ``classical evolution'' in real time
\end{itemize}

Physical interpretation of these correspondences is outside the scope of 
this paper.

%------------------------------------------------------------------------------
\subsection{Future Prospects}
\label{sec:outlook}

Based on the diagnostic framework established in this work, the following 
developments are conceivable.

\subsubsection{Generalization of torsion ansatz}

This work adopted T1 (axial) and T2 (vector trace) components, but 
generalization to include the third component (tensor) of the irreducible 
decomposition is a natural next step.

Also, allowing $r$-dependence of torsion parameters in a dynamical ansatz 
(e.g., $\eta(r)$, $V(r)$) may give rise to richer phase structure. As seen 
in Sec.~\ref{sec:mechanisms}, minimum existence is determined by competition 
between coefficients $B$ and $C$, and when these depend on $r$, new 
stabilization mechanisms may arise.

\subsubsection{Extension to other topologies}

As mentioned in Sec.~\ref{sec:diagnostic_tool}, this framework can be extended 
to other 3-dimensional Lie groups admitting left-invariant coframes. 
A particularly interesting candidate is $\mathrm{Sol}$ geometry (Bianchi Type VI$_0$). 
$\mathrm{Sol}$ is anisotropic like $\Nilthree$ but has different structure 
constant forms; what differences appear in phase structure is a future task.

Also, extension to non-parallelizable spaces ($S^2 \times \mathbb{R}$, etc.) 
involves technical challenges but is an important direction for exploring 
the relationship between topological constraints and stability.

\subsubsection{Semiclassical tunneling rate calculation}

Barrier height $\Delta V$ of Type I qualitatively indicates the ``robustness'' 
of metastable states, but quantitative evaluation of tunneling rates requires 
WKB approximation or instanton calculations.

Specifically, one constructs bounce solutions (classical solutions in the 
inverted potential $-\Veff(r)$) and evaluates decay rate 
$\Gamma \propto \exp(-B)$ from their action $B$. The analytical expressions 
for $\Veff(r)$ obtained in this work can be directly used as input for 
this calculation.

\subsubsection{Analysis of dynamical evolution}

This work focused on classification of static effective potentials, but 
by analytic continuation to Lorentzian signature, dynamics including time 
evolution can be discussed.

By deriving Friedmann-type constraint equations and tracking classical 
evolution for Type II (rolling) and post-tunneling evolution for Type I, 
the viability as cosmological scenarios can be evaluated.

\subsubsection{Coupling with matter fields}

This work treated pure EC+NY theory, but coupling with spinor and scalar 
fields would clarify the physical origin and effects of torsion.

In particular, in EC theory, spinor fields are natural sources of torsion, 
and dynamical torsion generation through fermion condensation may provide 
physical basis for the torsion ansatz assumed in this work.

\subsubsection{Connection with self-duality}

In 4-dimensional Euclidean geometry, the self-dual (or anti-self-dual) 
condition $R^{ab} = \pm {}^*R^{ab}$ provides strong constraints. 
Diagnosing whether stable solution candidates obtained in this work satisfy 
self-duality enables bridging to instanton interpretation.

When self-dual solutions exist, their action is directly related to 
topological invariants (Euler number, Hirzebruch signature), providing 
topological constraints on transition probabilities between sectors.

%------------------------------------------------------------------------------
\subsection{Concluding Remarks}
\label{sec:concluding}

This work has shown that in minisuperspace reduction of EC+NY theory, the 
phase structure of effective potentials depends strongly on topology. 
Through the three contrasting test beds of $\Sthree$, $\Tthree$, and $\Nilthree$, 
we clarified that background curvature and structure constants are the 
geometric factors determining phase structure.

The Type I/II/III classification presented here, along with the phase diagrams, 
representative points, and critical conditions based on it, function as 
reproducible diagnostic tools in EC+NY minisuperspace analysis. This framework 
provides a foundation for systematically organizing the correspondence 
between geometric input and phase structure in effective potential analysis 
of gravitational theories including torsion.
  % Conclusions and outlook

%------------------------------------------------------------------------------
% Acknowledgments
%------------------------------------------------------------------------------
\section*{Acknowledgments}

The author acknowledges the use of the following AI tools during manuscript preparation: 
Claude Opus 4.5 (Anthropic; accessed 2025-12), Gemini 3 Pro (Google; accessed 2025-12), 
Grok 4.1 Thinking (xAI; accessed 2025-12), and GPT-5.2 Thinking via ChatGPT (OpenAI; accessed 2025-12). 
These tools were used for language editing (including Japanese-English translation), outlining and 
improving exposition, drafting and reviewing auxiliary code (scripts and pseudocode) that was 
subsequently tested and validated by the author, and proposing candidate consistency checks and 
alternative derivations to be independently verified by the author. 
They were additionally used for literature discovery support (keyword generation and 
preliminary summaries of candidate papers); 
all references and factual claims were verified by the author using primary sources.

The AI tools did not determine the scientific claims of this work and were not used to generate or 
modify research data or evidentiary figures. 
The author takes full responsibility for the content and for any remaining errors.

This work was developed within the informal collaborative project \textbf{DPPU}
(\textbf{D}onut-like topology, \textbf{P}lanck-scale compactness,
\textbf{P}recession dynamics, \textbf{U}niverse).

%------------------------------------------------------------------------------
% Appendices
%------------------------------------------------------------------------------
\appendix
%==============================================================================
% Appendix A: Theoretical Details
%==============================================================================
\section{Theoretical Details}
\label{app:theory}

This appendix describes the detailed derivation process of calculations 
summarized in Sec.~\ref{sec:setup}.

%------------------------------------------------------------------------------
\subsection{Index Conventions}
\label{app:index_conventions}

\begin{itemize}
  \item \textbf{Internal indices (frame indices)}: $a, b, c, \ldots = 0, 1, 2, 3$
  \item \textbf{Spatial internal indices}: $i, j, k, \ldots = 0, 1, 2$
  \item \textbf{Frame metric}: $\eta_{ab} = \mathrm{diag}(+1, +1, +1, +1)$ (Euclidean)
\end{itemize}

%------------------------------------------------------------------------------
\subsection{Derivation of Levi--Civita Connection}
\label{app:LC_connection}

For orthonormal left-invariant frames $\{e^a\}$, the Levi-Civita connection 
is computed via the generalized Koszul formula:
\begin{equation}
  \Gamma^{a}{}_{bc} = \frac{1}{2}\left( C^{a}{}_{bc} + C_{cb}{}^a - C_{b}{}^a{}_c \right).
  \label{eq:koszul}
\end{equation}

\subsubsection{$\Sthree$ (SU(2))}

With $C^{i}{}_{jk} = \frac{4}{r}\varepsilon_{ijk}$, the nonzero components are:
\begin{equation}
  \Gamma^{0}{}_{12} = -\Gamma^{0}{}_{21} = \frac{2}{r}, \quad
  \Gamma^{1}{}_{20} = -\Gamma^{1}{}_{02} = \frac{2}{r}, \quad
  \Gamma^{2}{}_{01} = -\Gamma^{2}{}_{10} = \frac{2}{r}.
\end{equation}

\subsubsection{$\Tthree$ (Abelian)}

With $C^{i}{}_{jk} = 0$: $\Gamma^{a}{}_{bc} = 0$ for all components.

\subsubsection{$\Nilthree$ (Heisenberg)}

With $C^{2}{}_{01} = -1/r$, $C^{2}{}_{10} = +1/r$, the nonzero components are:
\begin{equation}
  \Gamma^{0}{}_{12} = \Gamma^{0}{}_{21} = \frac{1}{2r}, \quad
  \Gamma^{1}{}_{02} = \Gamma^{1}{}_{20} = -\frac{1}{2r}, \quad
  \Gamma^{2}{}_{01} = -\Gamma^{2}{}_{10} = -\frac{1}{2r}.
\end{equation}

%------------------------------------------------------------------------------
\subsection{Calculation of Contortion}
\label{app:contortion}

Following Hehl et al.\ convention:
\begin{equation}
  K_{abc} = \frac{1}{2}\left( T_{abc} + T_{bca} - T_{cab} \right).
\end{equation}

\subsubsection{T1 component (axial)}

For $T^{(1)}{}_{abc} = \frac{2\eta}{r}\varepsilon_{abc}$ (totally antisymmetric):
\begin{equation}
  K^{(1)}{}_{abc} = \frac{3\eta}{r}\varepsilon_{abc}.
\end{equation}

\subsubsection{T2 component (vector trace)}

For $T^{(2)}{}_{abc} = \frac{1}{3}(\eta_{ac}V_b - \eta_{ab}V_c)$ with $V_\mu = (0,0,0,V)$:
\begin{equation}
  K^{(2)}{}_{abc} = \frac{1}{6}\left( 2\eta_{ac}V_b - \eta_{ab}V_c - \eta_{bc}V_a \right).
\end{equation}

%------------------------------------------------------------------------------
\subsection{Effective Potential Formulas}
\label{app:Veff_formulas}

\subsubsection{$\Sthree \times \Sone$}

Scalar quantities:
\begin{align}
  R &= \frac{2(-V^2 r^2 - 9\eta^2 - 72\eta - 108)}{3r^2}, \\
  N_{\TT} &= -\frac{4V\eta}{r}, \quad 
  N_{\REE} = -\frac{2V(\eta + 4)}{r}, \quad 
  N_{\FULL} = \frac{2V(4 - \eta)}{r}.
\end{align}

Effective potential (FULL):
\begin{equation}
  \Veff^{(\Sthree)}(r) = \frac{2\pi^2 L}{3\kappa^2} r 
  \left[ V^2 r^2 + 6V\kappa^2\theta_{\NY}(\eta - 4)r + 9(\eta + 4)^2 - 36 \right].
\end{equation}

\subsubsection{$\Tthree \times \Sone$}

With isotropic setting $R_1 = R_2 = R_3 = r$:
\begin{align}
  R &= -\frac{2V^2}{3} - \frac{6\eta^2}{r^2}, \\
  N_{\FULL} &= N_{\REE} = -\frac{2V\eta}{r}, \quad N_{\TT} = -\frac{4V\eta}{r}.
\end{align}

Effective potential (FULL):
\begin{equation}
  \Veff^{(\Tthree)}(r) = \frac{16\pi^4 L}{3\kappa^2} r 
  \left[ V^2 r^2 + 6V\eta\kappa^2\theta_{\NY} r + 9\eta^2 \right].
\end{equation}

\subsubsection{$\Nilthree \times \Sone$}

Scalar quantities:
\begin{align}
  R &= \frac{-4V^2 r^2 - 36\eta^2 + 24\eta + 9}{6r^2}, \\
  N_{\TT} &= -\frac{4V\eta}{r}, \quad 
  N_{\REE} = \frac{2V(1 - 3\eta)}{3r}, \quad 
  N_{\FULL} = -\frac{2V(3\eta + 1)}{3r}.
\end{align}

Effective potential (FULL):
\begin{equation}
  \Veff^{(\Nilthree)}(r) = \frac{4\pi^4 L}{3\kappa^2} r 
  \left[ 4V^2 r^2 + 8V\kappa^2\theta_{\NY}(3\eta + 1)r + 36\left(\eta - \frac{1}{3}\right)^2 - 13 \right].
\end{equation}

%------------------------------------------------------------------------------
\subsection{Coefficient Summary}
\label{app:coefficients}

Writing $\Veff(r) = \mathcal{N} \cdot r \cdot (Ar^2 + Br + C)$:

\begin{table}[H]
\centering
\caption{Coefficient $B$ (with $\kappa = 1$).}
\begin{tabular}{@{}lccc@{}}
\toprule
 & FULL & TT & REE \\
\midrule
$\Sthree$ & $6V\theta_{\NY}(\eta - 4)$ & $12V\eta\theta_{\NY}$ & $6V\theta_{\NY}(\eta + 4)$ \\
$\Tthree$ & $6V\eta\theta_{\NY}$ & $12V\eta\theta_{\NY}$ & $6V\eta\theta_{\NY}$ \\
$\Nilthree$ & $8V\theta_{\NY}(3\eta + 1)$ & $48V\eta\theta_{\NY}$ & $8V\theta_{\NY}(3\eta - 1)$ \\
\bottomrule
\end{tabular}
\end{table}

\begin{table}[H]
\centering
\caption{Coefficient $C$ (topology-dependent, variant-independent).}
\begin{tabular}{@{}lcc@{}}
\toprule
Topology & $C(\eta)$ & $C = 0$ roots \\
\midrule
$\Sthree$ & $9(\eta + 4)^2 - 36$ & $\eta = -2, -6$ \\
$\Tthree$ & $9\eta^2$ & $\eta = 0$ \\
$\Nilthree$ & $36(\eta - 1/3)^2 - 13$ & $\eta \approx -0.27, 0.94$ \\
\bottomrule
\end{tabular}
\end{table}

%------------------------------------------------------------------------------
\subsection{Analytical Derivation of Critical Conditions}
\label{app:critical_conditions}

\subsubsection{Minimum existence condition}

For $\Veff(r) = \mathcal{N} r(Ar^2 + Br + C)$ with $A > 0$, the condition 
$\dd\Veff/\dd r = 0$ gives:
\begin{equation}
  3Ar^2 + 2Br + C = 0.
\end{equation}

The solutions are:
\begin{equation}
  r_\pm = \frac{-B \pm \sqrt{B^2 - 3AC}}{3A}.
  \label{eq:r_critical}
\end{equation}

Positive real solutions exist when:
\begin{equation}
  B < 0 \quad \text{and} \quad B^2 > 3AC.
\end{equation}

\subsubsection{Type I/II boundary}

The sign of $\dd\Veff/\dd r|_{r \to 0^+}$ is determined by $C$:
\begin{equation}
  C > 0 \Rightarrow \text{Type I}, \quad C < 0 \Rightarrow \text{Type II}.
\end{equation}

For $\Sthree$: $C = 0$ at $\eta = -2, -6$ (I/II boundaries).

\subsubsection{$\theta_{\NY}$ threshold for $\Tthree$}

For $\Tthree$ with $C = 9\eta^2 > 0$ (for $\eta \neq 0$), the condition 
$B^2 > 3AC$ requires:
\begin{equation}
  36V^2\eta^2\theta_{\NY}^2 > 27V^2\eta^2 \quad \Rightarrow \quad 
  \theta_{\NY} > \frac{\sqrt{3}}{2} \approx 0.87.
\end{equation}

%------------------------------------------------------------------------------
\subsection{Sign Conventions}
\label{app:sign_conventions}

The sign conventions adopted throughout this paper are summarized below:

\begin{table}[H]
\centering
\caption{Sign conventions used in this work.}
\label{tab:sign_conventions}
\begin{tabular}{@{}lll@{}}
\toprule
Quantity & Convention & Reference \\
\midrule
Frame metric & $\eta_{ab} = \mathrm{diag}(+1,+1,+1,+1)$ & Euclidean signature \\
Levi--Civita symbol & $\varepsilon_{0123} = +1$ & Standard \\
Structure constants & $\dd\sigma^i = -\frac{1}{2}C^i{}_{jk}\sigma^j \wedge \sigma^k$ & Maurer-Cartan \\
Contortion & $K_{abc} = \frac{1}{2}(T_{abc} + T_{bca} - T_{cab})$ & Hehl et al.\ (1976) \\
Riemann tensor & $R^{a}{}_{bcd} = \partial_c\Gamma^{a}{}_{bd} - \partial_d\Gamma^{a}{}_{bc}$ & Antisymmetric in \\
 & $\quad + \Gamma^{a}{}_{ec}\Gamma^{e}{}_{bd} - \Gamma^{a}{}_{ed}\Gamma^{e}{}_{bc}$ & 3rd \& 4th indices \\
EC connection & $\Gamma^{a}_{\mathrm{EC}} = \Gamma^{a}_{\mathrm{LC}} + K^a$ & Standard definition \\
\bottomrule
\end{tabular}
\end{table}

The contortion sign pattern is $(+1, +1, -1)$, consistent with Hehl et al.\ (1976) 
and the computational engine (DPPUv2 Engine Core v3) used in this work.
  % Theoretical details
%==============================================================================
% Appendix B: Reproducibility and Verification
%==============================================================================
\section{Reproducibility and Verification}
\label{app:reproducibility}

This appendix documents the computational engine specifications and verification 
procedures to ensure reproducibility of the results presented in this paper.

%------------------------------------------------------------------------------
\subsection{Engine Specifications}
\label{app:engine_specs}

\subsubsection{DPPUv2 Engine Core v3}

The calculations in this paper are performed using DPPUv2 Engine Core v3, 
a symbolic computation engine implemented in Python with SymPy.

\begin{table}[H]
\centering
\caption{Engine specifications.}
\begin{tabular}{@{}ll@{}}
\toprule
Item & Specification \\
\midrule
Engine name & DPPUv2 Engine Core v3 \\
Language & Python 3.10+ \\
Symbolic library & SymPy 1.12+ \\
Numerical library & NumPy, SciPy \\
Parallelization & multiprocessing (for parameter scans) \\
\bottomrule
\end{tabular}
\end{table}

\subsubsection{Modular architecture}

The engine consists of the following modules:

\begin{itemize}
  \item \texttt{DPPUv2\_engine\_core\_v3.py}: Core symbolic computation
  \item \texttt{DPPUv2\_parameter\_scan\_v3.py}: Parameter space scanning
  \item \texttt{DPPUv2\_runner\_*.py}: Topology-specific runners
  \item \texttt{DPPUv2\_visualize\_*.py}: Visualization tools
\end{itemize}

%------------------------------------------------------------------------------
\subsection{Sanity Checks}
\label{app:sanity_checks}

The engine performs automatic verification at each computational step. 
The following sanity checks are implemented:

\subsubsection{Metric compatibility}

After computing the Levi--Civita connection, the engine verifies:
\begin{equation}
  \nabla_c \eta_{ab} = \partial_c \eta_{ab} - \Gamma^d{}_{ca}\eta_{db} 
                      - \Gamma^d{}_{cb}\eta_{ad} = 0.
\end{equation}

For frame basis with constant $\eta_{ab} = \delta_{ab}$, this reduces to:
\begin{equation}
  \Gamma_{abc} + \Gamma_{bac} = 0.
\end{equation}

\subsubsection{Riemann tensor antisymmetry (three-stage verification)}

The Riemann tensor $R^a{}_{bcd}$ satisfies the following antisymmetry properties:

\paragraph{Stage 1: Antisymmetry in last two indices}
\begin{equation}
  R^a{}_{bcd} = -R^a{}_{bdc}.
\end{equation}

\paragraph{Stage 2: Antisymmetry in first two indices (after lowering)}
\begin{equation}
  R_{abcd} = -R_{bacd}.
\end{equation}

\paragraph{Stage 3: Pair exchange symmetry}
\begin{equation}
  R_{abcd} = R_{cdab}.
\end{equation}

The engine verifies all three properties and logs any violations.

\subsubsection{Torsion tensor verification}

For the torsion ansatz, the engine verifies:

\paragraph{T1 component:} Total antisymmetry
\begin{equation}
  T^{(1)}_{abc} = T^{(1)}_{[abc]}.
\end{equation}

\paragraph{T2 component:} Trace condition
\begin{equation}
  T^{(2)\lambda}{}_{\mu\lambda} = V_\mu.
\end{equation}

\subsubsection{NY density consistency}

The engine verifies the decomposition relation:
\begin{equation}
  N_{\FULL} = N_{\TT} - N_{\REE}.
\end{equation}

This is checked numerically at representative parameter points.

%------------------------------------------------------------------------------
\subsection{Verification Log Examples}
\label{app:verification_logs}

The following shows excerpts from verification logs generated during 
computation.

\subsubsection{$\Sthree \times \Sone$ verification}

\begin{lstlisting}[basicstyle=\ttfamily\footnotesize]
=== S3S1 MX Mode Verification ===
[PASS] Metric compatibility: max deviation = 0.0
[PASS] Riemann antisymmetry (stage 1): verified
[PASS] Riemann antisymmetry (stage 2): verified
[PASS] Riemann antisymmetry (stage 3): verified
[PASS] NY decomposition: N_FULL = N_TT - N_REE
[INFO] Ricci scalar: 2*(-V**2*r**2 - 9*eta**2 - 72*eta - 108)/(3*r**2)
[INFO] N_FULL: 2*V*(4 - eta)/r
\end{lstlisting}

\subsubsection{$\Tthree \times \Sone$ verification}

\begin{lstlisting}[basicstyle=\ttfamily\footnotesize]
=== T3S1 MX Mode Verification ===
[PASS] Metric compatibility: max deviation = 0.0
[PASS] Riemann antisymmetry (stage 1): verified
[PASS] Riemann antisymmetry (stage 2): verified
[PASS] Riemann antisymmetry (stage 3): verified
[PASS] NY decomposition: N_FULL = N_TT - N_REE
[INFO] Ricci scalar: -2*V**2/3 - 6*eta**2/R1**2
[INFO] N_FULL: -2*V*eta/R1
[NOTE] N_FULL = N_REE (degenerate case for flat background)
\end{lstlisting}

\subsubsection{$\Nilthree \times \Sone$ verification}

\begin{lstlisting}[basicstyle=\ttfamily\footnotesize]
=== Nil3S1 MX Mode Verification ===
[PASS] Metric compatibility: max deviation = 0.0
[PASS] Riemann antisymmetry (stage 1): verified
[PASS] Riemann antisymmetry (stage 2): verified
[PASS] Riemann antisymmetry (stage 3): verified
[PASS] NY decomposition: N_FULL = N_TT - N_REE
[INFO] Ricci scalar: (-4*R**2*V**2 - 36*eta**2 + 24*eta + 9)/(6*R**2)
[INFO] N_FULL: -2*V*(3*eta + 1)/(3*R)
\end{lstlisting}

%------------------------------------------------------------------------------
\subsection{Cross-Validation}
\label{app:cross_validation}

\subsubsection{Analytical vs.\ numerical comparison}

For selected parameter points, we compare analytically derived $\Veff(r)$ 
expressions with direct numerical evaluation:

\begin{table}[H]
\centering
\caption{Analytical vs.\ numerical comparison at test points.}
\begin{tabular}{@{}lcccc@{}}
\toprule
Topology & $(V, \eta, \theta_{\NY})$ & $r$ & $\Veff^{\text{(anal)}}$ & 
$\Veff^{\text{(num)}}$ \\
\midrule
$\Sthree$ & $(2, -4, 1)$ & 1.0 & $-473.74$ & $-473.74$ \\
$\Sthree$ & $(2, -4, 1)$ & 2.0 & $-631.65$ & $-631.65$ \\
$\Tthree$ & $(2, -3, 1)$ & 1.0 & $1263.31$ & $1263.31$ \\
$\Nilthree$ & $(2, 0.5, 1)$ & 0.5 & $-128.30$ & $-128.30$ \\
\bottomrule
\end{tabular}
\end{table}

Agreement to machine precision confirms consistency between symbolic 
derivation and numerical evaluation.

\subsubsection{Limiting case verification}

The engine verifies correct behavior in limiting cases:

\paragraph{$\theta_{\NY} \to 0$ limit:}
NY contribution vanishes, recovering pure EC theory results.

\paragraph{$\eta \to 0$, $V \to 0$ limit:}
Torsion vanishes, recovering Levi--Civita (GR) results.

\paragraph{$\Tthree$ with $C^i{}_{jk} = 0$:}
Recovers flat space results with $R_{\LC} = 0$.

For details on the computational environment and software dependencies, 
see Appendix~\ref{app:access}.

  % Reproducibility and verification
%==============================================================================
% Appendix C: Numerical Computation Details
%==============================================================================
\section{Numerical Computation Details}
\label{app:numerical}

This appendix documents the numerical methods, parameters, and algorithms 
used in the parameter scanning and Type classification.

%------------------------------------------------------------------------------
\subsection{Scan Parameters}
\label{app:scan_parameters}

\subsubsection{Parameter ranges}

\begin{table}[H]
\centering
\caption{Parameter ranges for phase diagram generation.}
\begin{tabular}{@{}lcccc@{}}
\toprule
Parameter & Min & Max & Grid points & Spacing \\
\midrule
$V$ & 0.0 & 5.0 & 51 & 0.1 (linear) \\
$\eta$ & $-10.0$ & 5.0 & 151 & 0.099 (linear) \\
$\theta_{\NY}$ & 0.0 & 5.0 & 51 & 0.098 (linear) \\
\bottomrule
\end{tabular}
\end{table}

\subsubsection{Fixed parameters}

\begin{table}[H]
\centering
\caption{Fixed parameters.}
\begin{tabular}{@{}lll@{}}
\toprule
Parameter & Value & Description \\
\midrule
$\kappa$ & 1.0 & Gravitational coupling \\
$L$ & 1.0 & $S^1$ circumference \\
$r_{\min}$ & 0.01 & Lower bound for $r$ search \\
$r_{\max}$ & $10^6$ & Upper bound for $r$ search \\
\bottomrule
\end{tabular}
\end{table}

\subsubsection{Notes on cutoff dependence}
\label{app:robustness}

The Type classification necessarily depends on numerical cutoffs such as the
$r$-search interval $(r_{\min}, r_{\max})$ and the boundary threshold $\delta$
in the ``bound hit'' criterion.
In practice, these cutoffs are chosen wide enough to separate genuinely boundary-attached
profiles (Type~III) from well-formed minima (Type~I/II). 
Points extremely close to a phase boundary are expected to be the most sensitive 
to finite grid resolution.
For this reason, our interpretation of phase diagrams focuses on robust features
(e.g., the existence or absence of wide Type-I regions and the systematic topology dependence),
rather than on single-grid-point fluctuations near boundaries.

\subsubsection{Total evaluations}

For each topology-variant combination:
\begin{equation}
  N_{\text{eval}} = 51 \times 151 \times 51 = 392{,}751 \text{ points}.
\end{equation}

Total for 3 topologies $\times$ 3 variants = 9 combinations:
\begin{equation}
  N_{\text{total}} = 9 \times 392{,}751 = 3{,}534{,}759 \text{ evaluations}.
\end{equation}

%------------------------------------------------------------------------------
\subsection{Extremum Search Algorithm}
\label{app:search_algorithm}

\subsubsection{Primary method: Brent's method}

For finding minima of $\Veff(r)$, we use Brent's method as implemented in 

\texttt{scipy.optimize.minimize\_scalar} with the \texttt{bounded} option.

\begin{table}[H]
\centering
\caption{Brent's method parameters.}
\begin{tabular}{@{}ll@{}}
\toprule
Parameter & Value \\
\midrule
Method & \texttt{bounded} \\
Bounds & $[r_{\min}, r_{\max}] = [0.01, 10^6]$ \\
Tolerance (\texttt{xatol}) & $10^{-8}$ \\
Max iterations & 500 \\
\bottomrule
\end{tabular}
\end{table}

%------------------------------------------------------------------------------
\subsection{Type Classification Algorithm}
\label{app:classification_algorithm}

The Type classification follows the flowchart in Figure~\ref{fig:flowchart}.

\begin{figure}[htbp]
\centering
  \includegraphics[width=0.8\textwidth]{figures/Fig19_Type_classification_flowchart.png}
  \caption{Type classification flowchart.}
\label{fig:flowchart}
\end{figure}

\subsubsection{Step-by-step algorithm}

\begin{enumerate}
  \item \textbf{Extremum search}: Find candidate minimum $r_0$ using Brent's method
  
  \item \textbf{Existence check}: If no minimum found (optimizer returns boundary), 
        classify as Type III
  
  \item \textbf{Bound hit check}: If $r_0 < r_{\min} + \delta$ or 
        $r_0 > r_{\max} - \delta$ (with $\delta = 0.02$), classify as Type III
  
  \item \textbf{Curvature check}: Compute $\dd^2\Veff/\dd r^2|_{r_0}$ numerically. 
        If $\leq 0$, classify as Type III
  
  \item \textbf{Barrier check}: Evaluate gradient near origin:
    \begin{equation}
      s_0 = \left.\frac{\dd\Veff}{\dd r}\right|_{r = r_{\min}}
    \end{equation}
    \begin{itemize}
      \item If $s_0 > 0$: Type I (barrier exists)
      \item If $s_0 \leq 0$: Type II (rolling, no barrier)
    \end{itemize}
\end{enumerate}

\subsubsection{Numerical differentiation}

Derivatives are computed using central differences:
\begin{align}
  \frac{\dd\Veff}{\dd r} &\approx \frac{\Veff(r + h) - \Veff(r - h)}{2h}, \\
  \frac{\dd^2\Veff}{\dd r^2} &\approx \frac{\Veff(r + h) - 2\Veff(r) + \Veff(r - h)}{h^2},
\end{align}
with step size $h = 10^{-5}$.

%------------------------------------------------------------------------------
\subsection{Barrier Height Calculation}
\label{app:barrier_height}

\subsubsection{Type I: Barrier height}

For Type I configurations, the barrier height $\Delta V$ is computed as:
\begin{equation}
  \Delta V = \max_{r \in [r_{\min}, r_0]} \Veff(r) - \Veff(r_0).
\end{equation}

\begin{enumerate}
  \item Generate dense grid in $[r_{\min}, r_0]$: 100 points, logarithmic spacing
  \item Evaluate $\Veff(r)$ at all grid points
  \item Find maximum value $\Veff^{\max}$
  \item Compute $\Delta V = \Veff^{\max} - \Veff(r_0)$
\end{enumerate}

This represents the depth of the potential well from the near-origin value 
to the minimum.

\begin{itemize}
  \item If $\Veff^{\max} \leq \Veff(r_0)$: $\Delta V = 0$ (no barrier, 
        reclassify as Type II)
  \item If $r_0 < 2 r_{\min}$: Insufficient range, $\Delta V$ marked as undefined
\end{itemize}

\subsubsection{Type II: Well depth}

For Type II configurations, the well depth $\Delta V$ is defined as:
\begin{equation}
  \Delta V = \Veff(r_{\min}) - \Veff(r_0).
\end{equation}


%------------------------------------------------------------------------------
\subsection{Numerical Precision and Uncertainties}
\label{app:numerical_precision}

\subsubsection{Grid resolution uncertainty}

Phase boundary positions have uncertainties on the order of the grid spacing:
\begin{itemize}
  \item $\Delta\eta \approx 0.1$
  \item $\Delta V \approx 0.1$
  \item $\Delta\theta_{\NY} \approx 0.1$
\end{itemize}

As shown in Sec.~\ref{sec:numerical}, the analytical boundary conditions 
and numerical scan results agree within this uncertainty.

\subsubsection{Extremum search precision}

The convergence tolerance of Brent's method ($\sim 10^{-8}$) is sufficient 
for the physically meaningful precision of $r_0$ (3--4 significant digits).

\subsubsection{Curvature verification step size}

The step size $h = 10^{-5}$ for numerical differentiation was chosen to 
balance truncation and round-off errors. Results were verified to be 
insensitive to variations in the range $h = 10^{-4}$ to $10^{-6}$.

%------------------------------------------------------------------------------
\subsection{Output Format}
\label{app:output_format}

\subsubsection{CSV output structure}

Results are saved in CSV format with the following columns:

\begin{table}[H]
\centering
\caption{CSV output columns.}
\begin{tabular}{@{}lll@{}}
\toprule
Column & Type & Description \\
\midrule
\texttt{V} & float & Vector torsion parameter \\
\texttt{eta} & float & Axial torsion parameter \\
\texttt{theta\_NY} & float & Nieh--Yan coupling \\
\texttt{type} & int & Classification (1, 2, or 3) \\
\texttt{r0} & float & Stable radius (NaN for Type III) \\
\texttt{Veff\_min} & float & Potential at minimum \\
\texttt{delta\_V} & float & Barrier height (Type I only) \\
\texttt{log10\_r0} & float & $\log_{10}(r_0)$ \\
\texttt{log10\_deltaV} & float & $\log_{10}(\Delta V)$ \\
\texttt{status} & string & Convergence status \\
\bottomrule
\end{tabular}
\end{table}

\subsubsection{File naming convention}

Output files follow the naming pattern:
\begin{center}
\texttt{<topology>\_<variant>\_<mode>\_<timestamp>.csv}
\end{center}

Example: \texttt{S3S1\_FULL\_MX\_20241214\_135031.csv}

\subsubsection{Code access and execution}

For code access, installation instructions, and detailed execution procedures, 
see Appendix~\ref{app:access}.

%------------------------------------------------------------------------------
\subsection{Error Handling}
\label{app:error_handling}

\subsubsection{Convergence failures}

If the optimizer fails to converge within the maximum iterations:
\begin{itemize}
  \item Log warning with parameter values
  \item Attempt grid search fallback
  \item If still unsuccessful, classify as Type III with status \texttt{FAILED}
\end{itemize}

\subsubsection{Numerical overflow}

For extreme parameter values where $\Veff(r)$ overflows:
\begin{itemize}
  \item Detected by checking for \texttt{inf} or \texttt{nan} values
  \item Point classified as Type III with status \texttt{OVERFLOW}
\end{itemize}

\subsubsection{Statistics}

Typical failure rates in production runs:
\begin{itemize}
  \item Convergence failures: $< 0.1\%$
  \item Overflow errors: $< 0.01\%$
  \item Total valid classifications: $> 99.8\%$
\end{itemize}
  % Numerical computation details
%==============================================================================
% Appendix D: Visualization Tools
%==============================================================================
\section{Visualization Tools}
\label{app:visualization}

This appendix documents the visualization tools developed for analyzing 
and presenting the results of this paper. For code access and installation 
instructions, see Appendix~\ref{app:access}.

%------------------------------------------------------------------------------
\subsection{Overview of Visualization Pipeline}
\label{app:viz_overview}

The visualization pipeline consists of the following components:

\begin{enumerate}
  \item \textbf{Phase map generator}: Creates 2D phase diagrams on $(V, \eta)$ plane
  \item \textbf{Phase matrix generator}: Creates comparison grids across 
        topologies and variants
  \item \textbf{Potential plotter}: Generates $\Veff(r)$ curves for 
        representative points
  \item \textbf{Interactive viewer}: Jupyter-based tool for exploration
\end{enumerate}

%------------------------------------------------------------------------------
\subsection{Phase Diagram Generation}
\label{app:phase_diagram}

\subsubsection{Color mapping}

Phase diagrams use the following visual encoding:

\begin{table}[H]
\centering
\caption{Visual encoding for phase diagrams.}
\begin{tabular}{@{}lll@{}}
\toprule
Element & Encoding & Description \\
\midrule
Type I region & Solid color & Metastable with barrier \\
Type II region & Hatched color & Rolling, no barrier \\
Type III region & White & Unstable \\
Color gradient & Purple $\to$ Yellow & $\log_{10}(r_0)$ value \\
White contours & Solid lines & $\log_{10}(\Delta V)$ levels \\
\bottomrule
\end{tabular}
\end{table}

\subsubsection{Colormap specification}

\begin{itemize}
  \item Base colormap: \texttt{viridis} (perceptually uniform)
  \item Range: $\log_{10}(r_0) \in [-2, 5]$
  \item Out-of-range handling: Clipped to bounds
\end{itemize}

\subsubsection{Hatching for Type II}

Type II regions are distinguished from Type I by diagonal hatching:
\begin{itemize}
  \item Pattern: $45^\circ$ diagonal lines
  \item Density: 4 lines per unit
  \item Color: Semi-transparent gray overlay
\end{itemize}

\subsubsection{Implementation}

\begin{lstlisting}[language=Python, basicstyle=\ttfamily\footnotesize]
import matplotlib.pyplot as plt
import numpy as np

def plot_phase_diagram(V_grid, eta_grid, types, r0_values, 
                       delta_V_values, theta_NY):
    fig, ax = plt.subplots(figsize=(8, 6))
    
    # Create masked arrays for each type
    type1_mask = (types == 1)
    type2_mask = (types == 2)
    
    # Plot Type I (solid)
    log_r0 = np.log10(r0_values)
    im = ax.pcolormesh(V_grid, eta_grid, log_r0, 
                       cmap='viridis', vmin=-2, vmax=5,
                       shading='auto')
    
    # Overlay hatching for Type II
    ax.contourf(V_grid, eta_grid, type2_mask, 
                levels=[0.5, 1.5], hatches=['//'],
                colors='none', alpha=0.3)
    
    # Contours for barrier height
    cs = ax.contour(V_grid, eta_grid, 
                    np.log10(delta_V_values),
                    levels=[1, 2, 3, 4], colors='white',
                    linewidths=0.8)
    ax.clabel(cs, inline=True, fontsize=8)
    
    # Labels and colorbar
    ax.set_xlabel(r'$V$')
    ax.set_ylabel(r'$\eta$')
    ax.set_title(f'Phase diagram ($\\theta_{{NY}} = {theta_NY}$)')
    plt.colorbar(im, ax=ax, label=r'$\log_{10}(r_0)$')
    
    return fig, ax
\end{lstlisting}

%------------------------------------------------------------------------------
\subsection{Phase Matrix Generation}
\label{app:phase_matrix}

Phase matrices display 3 topologies $\times$ 3 variants in a single figure 
for comparison.

\subsubsection{Layout}

\begin{itemize}
  \item Grid: $3 \times 3$ subplots
  \item Rows: Topologies ($\Sthree$, $\Tthree$, $\Nilthree$)
  \item Columns: NY variants (FULL, TT, REE)
  \item Shared colorbar: Single colorbar for entire figure
\end{itemize}

\subsubsection{Filename convention}

Phase matrix files are named:
\begin{center}
\texttt{fig12\_phase\_matrix\_XXXX.png}
\end{center}
where \texttt{XXXX} is Serial number.

%------------------------------------------------------------------------------
\subsection{Potential Curve Plotting}
\label{app:potential_curves}

\subsubsection{Standard plot format}

For representative points, $\Veff(r)$ curves are plotted with:

\begin{itemize}
  \item $x$-axis: $r$ (linear or logarithmic scale)
  \item $y$-axis: $\Veff(r)$ (linear scale)
  \item Markers: Minimum position $r_0$ (if exists)
  \item Annotations: Type classification, $\Delta V$ value
\end{itemize}

\subsubsection{Multi-panel comparison}

For phase-potential correspondence figures (e.g., Figure~\ref{fig:phase_potential_S3}):

\begin{itemize}
  \item Left panel: Phase diagram with marked points
  \item Right panels: $\Veff(r)$ curves for each marked point
  \item Color coding: Consistent colors between diagram markers and curves
\end{itemize}

%------------------------------------------------------------------------------
\subsection{Interactive Viewer}
\label{app:interactive}

\subsubsection{Jupyter notebook interface}

The interactive viewer (\texttt{DPPUv2\_interactive\_viewer\_v3.py}) provides:

\paragraph{Configuration selection}
\begin{itemize}
  \item Topology switching: $\Sthree \times \Sone$, $\Tthree \times \Sone$, $\Nilthree \times \Sone$
  \item Torsion mode: MX (mixed), AX (axial only), VT (vector only)
  \item NY variant: FULL, TT, REE
  \item $\Tthree$ anisotropy parameters: $\alpha = R_2/R_1$, $\beta = R_3/R_1$
\end{itemize}

\paragraph{Phase diagram display}
\begin{itemize}
  \item Type I/II/III classification on $(V, \eta)$ plane
  \item Real-time update via $\theta_{\NY}$ slider
\end{itemize}

\paragraph{Potential plotting}
\begin{itemize}
  \item Click-to-select points on phase diagram (up to 3 simultaneous points)
  \item $\Veff(r)$ shape comparison
  \item Axis scale switching (linear / log / symlog)
  \item Dynamic range adjustment
\end{itemize}

\subsubsection{Usage}

\begin{lstlisting}[language=Python, basicstyle=\ttfamily\footnotesize]
%matplotlib widget
from DPPUv2_interactive_viewer_v3 import DPPUv2InteractiveViewer

# Initialize and launch viewer
viewer = DPPUv2InteractiveViewer()
viewer.display()

# Optional: Customize slider ranges before instantiation
DPPUv2InteractiveViewer.SLIDER_V_MAX_MAX = 30.0
DPPUv2InteractiveViewer.SLIDER_ETA_MIN_MIN = -30.0
viewer = DPPUv2InteractiveViewer()
viewer.display()
\end{lstlisting}

\subsubsection{Operation instructions}

\paragraph{Generating phase diagrams}
\begin{enumerate}
  \item Select topology, mode, and variant in the Configuration panel
  \item Adjust the $\theta_{\NY}$ slider
  \item Click the ``Draw Phase Diagram'' button
\end{enumerate}

\paragraph{Comparing potentials}
\begin{enumerate}
  \item Click any point on the phase diagram (registered as Point 1)
  \item Select ``Point 2'' in Point Selection and click another point
  \item Up to 3 points of $\Veff(r)$ are displayed overlaid in the right panel
\end{enumerate}

\paragraph{Exploring $\Tthree$ anisotropy}
\begin{enumerate}
  \item Select ``T3'' as topology
  \item The Anisotropy panel becomes visible
  \item Adjust $\alpha$ and $\beta$ sliders to set anisotropy ratios
  \item Redraw the phase diagram
\end{enumerate}

\subsubsection{Widget layout}

\begin{table}[H]
\centering
\caption{Interactive viewer widgets.}
\begin{tabular}{@{}lll@{}}
\toprule
Widget & Type & Range/Options \\
\midrule
$V$ slider & FloatSlider & $[0, V_{\max}]$, step 0.1 \\
$\eta$ slider & FloatSlider & $[-10, 5]$, step 0.1 \\
$\theta_{\NY}$ slider & FloatSlider & $[0, 5]$, step 0.1 \\
Topology dropdown & Dropdown & S3, T3, Nil3 \\
Variant dropdown & Dropdown & FULL, TT, REE \\
Scale toggle & ToggleButton & Linear / Log \\
\bottomrule
\end{tabular}
\end{table}

%------------------------------------------------------------------------------
\subsection{Figure Export Settings}
\label{app:export}

\subsubsection{Resolution and format}

\begin{table}[H]
\centering
\caption{Figure export settings.}
\begin{tabular}{@{}lll@{}}
\toprule
Setting & Value & Notes \\
\midrule
Format & PNG & For raster figures \\
DPI & 300 & Publication quality \\
Figure size & Variable & Typically 8$\times$6 inches \\
Font & Computer Modern & LaTeX compatible \\
Font size & 12pt & Axis labels \\
\bottomrule
\end{tabular}
\end{table}

\subsubsection{LaTeX integration}

Figures are generated with LaTeX rendering enabled:

\begin{lstlisting}[language=Python, basicstyle=\ttfamily\footnotesize]
import matplotlib.pyplot as plt

plt.rcParams.update({
    'text.usetex': True,
    'font.family': 'serif',
    'font.serif': ['Computer Modern'],
    'font.size': 12,
    'axes.labelsize': 14,
    'legend.fontsize': 10,
})
\end{lstlisting}

%------------------------------------------------------------------------------
\subsection{List of Generated Figures}
\label{app:figure_list}

\begin{table}[H]
\centering
\caption{List of figures in this paper.}
\begin{tabular}{@{}lll@{}}
\toprule
Figure & Filename & Description \\
\midrule
\ref{fig:schematic} & \texttt{fig01\_Schematic\_Classification.png} & Type I/II/III schematic \\
\ref{fig:variant_S3} & \texttt{fig03\_variant\_comparison\_S3.png} & $\Sthree$ variant comparison \\
\ref{fig:decomposition_S3} & \texttt{fig04\_decomposition\_S3.png} & $\Sthree$ potential decomposition \\
\ref{fig:variant_T3} & \texttt{fig05\_variant\_comparison\_T3.png} & $\Tthree$ variant comparison \\
\ref{fig:decomposition_T3} & \texttt{fig06\_decomposition\_T3.png} & $\Tthree$ potential decomposition \\
\ref{fig:variant_Nil3} & \texttt{fig07\_variant\_comparison\_Nil3.png} & $\Nilthree$ variant comparison \\
\ref{fig:decomposition_Nil3} & \texttt{fig08\_decomposition\_Nil3.png} & $\Nilthree$ potential decomposition \\
\ref{fig:phase_S3} & \texttt{fig09\_phase\_diagram\_S3-FULL.png} & $\Sthree$ phase diagram \\
\ref{fig:phase_T3} & \texttt{fig10\_phase\_diagram\_T3-FULL.png} & $\Tthree$ phase diagram \\
\ref{fig:phase_Nil3} & \texttt{fig11\_phase\_diagram\_Nil3-FULL.png} & $\Nilthree$ phase diagram \\
\ref{fig:phase_matrix} & \texttt{fig12\_phase\_matrix\_0010.png} & Phase matrix ($\theta_{\NY}=1$) \\
\ref{fig:phase_potential_S3} & \texttt{fig13\_Phase\_Potential\_S3.png} & $\Sthree$ phase-potential \\
\ref{fig:phase_potential_T3} & \texttt{fig14\_Phase\_Potential\_T3.png} & $\Tthree$ phase-potential \\
\ref{fig:phase_potential_Nil3} & \texttt{fig15\_Phase\_Potential\_Nil3.png} & $\Nilthree$ phase-potential \\
\ref{fig:scaling_S3} & \texttt{fig16\_Scaling\_laws\_S3.png} & $\Sthree$ scaling laws \\
\ref{fig:scaling_T3} & \texttt{fig17\_Scaling\_laws\_T3.png} & $\Tthree$ scaling laws \\
\ref{fig:scaling_Nil3} & \texttt{fig18\_Scaling\_laws\_Nil3.png} & $\Nilthree$ scaling laws \\
\bottomrule
\end{tabular}
\end{table}
  % Visualization tools
%==============================================================================
% Appendix E: Volume Rescaling Invariance in $T^3 \times S^1$
%==============================================================================
\clearpage
\section{Volume Rescaling Invariance in $\Tthree \times \Sone$}
\label{app:rescaling}

This appendix discusses the volume rescaling properties of the effective 
potential for $\Tthree \times \Sone$ and justifies the isotropic setting 
$R_1 = R_2 = R_3 = r$ adopted in the main text.

%------------------------------------------------------------------------------
\subsection{General Anisotropic Setting}
\label{app:anisotropic}

\subsubsection{Parameterization}

For $\Tthree$, the three circumferences $R_1$, $R_2$, $R_3$ are independent 
parameters. The coframe is:
\begin{equation}
  e^0 = R_1 \, \dd x^1, \quad e^1 = R_2 \, \dd x^2, \quad 
  e^2 = R_3 \, \dd x^3, \quad e^3 = L \, \dd\tau,
\end{equation}
where $x^i \in [0, 2\pi)$ are periodic coordinates.

\subsubsection{Volume element}

The total volume is:
\begin{equation}
  \mathrm{Vol}(\Tthree \times \Sone) = (2\pi)^4 L R_1 R_2 R_3.
\end{equation}

\subsubsection{Effective potential (general form)}

With all three radii independent, the effective potential for the FULL 
variant is:
\begin{equation}
  \Veff^{(\Tthree)}(R_1, R_2, R_3) = \frac{(2\pi)^4 L}{3\kappa^2} 
  \left[ V^2 R_1 R_2 R_3 + 6V\eta\kappa^2\theta_{\NY} R_2 R_3 
         + \frac{9\eta^2 R_2 R_3}{R_1} \right].
  \label{eq:Veff_T3_general}
\end{equation}

Note the asymmetric dependence on $R_1$ versus $R_2$, $R_3$: this arises 
because the torsion ansatz singles out the $e^0$ direction for the axial 
component normalization.

%------------------------------------------------------------------------------
\subsection{Rescaling Properties}
\label{app:rescaling_properties}

\subsubsection{Uniform rescaling}

Under uniform rescaling $R_i \to \lambda R_i$ (for all $i = 1, 2, 3$):
\begin{equation}
  \Veff \to \lambda^3 \cdot \frac{(2\pi)^4 L}{3\kappa^2} 
  \left[ V^2 R_1 R_2 R_3 + \frac{6V\eta\kappa^2\theta_{\NY}}{\lambda} R_2 R_3 
         + \frac{9\eta^2}{\lambda^2} \frac{R_2 R_3}{R_1} \right].
\end{equation}

The three terms scale as $\lambda^3$, $\lambda^2$, and $\lambda^1$ respectively. 
This non-uniform scaling means that the \textit{shape} of the potential 
changes under rescaling, but the \textit{Type classification} (existence 
of minima, barriers) is preserved.

\subsubsection{Anisotropic rescaling relative to $R_1$}

Consider the parameterization $R_2 = \alpha R_1$ and $R_3 = \beta R_1$, 
where $\alpha$ and $\beta$ are dimensionless anisotropy ratios. 
Substituting into Eq.~\eqref{eq:Veff_T3_general}:
\begin{equation}
  \frac{R_2 R_3}{R_1} = \alpha \beta R_1,
\end{equation}
which yields the key relation:
\begin{equation}
  \Veff(\alpha, \beta) = \alpha \beta \, \Veff(1, 1).
  \label{eq:Veff_rescaling}
\end{equation}

This demonstrates that anisotropic rescaling merely multiplies $\Veff$ by 
an overall factor $\alpha\beta$, leaving the extremum position with respect 
to $R_1$ and the sign of $\partial_{R_1}\Veff$ unchanged. Consequently, 
the \textbf{phase boundary positions are invariant} under this rescaling, 
while the potential depth scales as $\alpha\beta$.

\subsubsection{Shape-preserving property}

Define dimensionless ratios:
\begin{equation}
  \rho_1 = \frac{R_1}{R_{\text{ref}}}, \quad 
  \rho_2 = \frac{R_2}{R_{\text{ref}}}, \quad 
  \rho_3 = \frac{R_3}{R_{\text{ref}}},
\end{equation}
where $R_{\text{ref}}$ is a reference scale.

The Type classification depends only on the ratios $\rho_i$ and the 
dimensionless combinations:
\begin{equation}
  \tilde{V} = V R_{\text{ref}}, \quad 
  \tilde{\eta} = \eta, \quad 
  \tilde{\theta} = \theta_{\NY}.
\end{equation}

This means that phase diagrams are independent of the overall scale 
$R_{\text{ref}}$, justifying the use of $\kappa = L = 1$ units.

%------------------------------------------------------------------------------
\subsection{Isotropic Reduction}
\label{app:isotropic}

\subsubsection{Motivation}

For comparison with $\Sthree$ and $\Nilthree$ (which have single scale 
parameter $r$), we adopt the isotropic setting:
\begin{equation}
  R_1 = R_2 = R_3 = r.
\end{equation}

This reduces the 3-parameter family to a 1-parameter family, enabling 
direct comparison of phase structures across topologies.

\subsubsection{Resulting potential}

Substituting $R_1 = R_2 = R_3 = r$ into Eq.~\eqref{eq:Veff_T3_general}:
\begin{equation}
  \Veff^{(\Tthree,\text{iso})}(r) = \frac{(2\pi)^4 L}{3\kappa^2} 
  \left[ V^2 r^3 + 6V\eta\kappa^2\theta_{\NY} r^2 + 9\eta^2 r \right].
\end{equation}

This has the same $r \cdot (Ar^2 + Br + C)$ structure as $\Sthree$ and 
$\Nilthree$, with:
\begin{equation}
  A = V^2, \quad B = 6V\eta\kappa^2\theta_{\NY}, \quad C = 9\eta^2.
\end{equation}

\subsubsection{Comparison with anisotropic case}

The key difference from the anisotropic case:

\begin{itemize}
  \item \textbf{Anisotropic}: Three independent variables $(R_1, R_2, R_3)$; 
        minima form a 2-dimensional surface in parameter space
  \item \textbf{Isotropic}: Single variable $r$; minima are isolated points 
        on the $r$-axis
\end{itemize}

The isotropic setting captures the ``diagonal slice'' of the full 
3-dimensional configuration space.

%------------------------------------------------------------------------------
\subsection{Validity of Isotropic Approximation}
\label{app:validity}

\subsubsection{Stability analysis}

For the isotropic minimum $r = r_0$ to be stable against anisotropic 
perturbations, we require:
\begin{equation}
  \left.\frac{\partial^2 \Veff}{\partial R_i \partial R_j}\right|_{R_1=R_2=R_3=r_0} 
  \quad \text{positive definite}.
\end{equation}

\subsubsection{Hessian computation}

The Hessian matrix at the isotropic point has the structure:
\begin{equation}
  H_{ij} = \begin{pmatrix}
    H_{11} & H_{12} & H_{13} \\
    H_{12} & H_{22} & H_{23} \\
    H_{13} & H_{23} & H_{33}
  \end{pmatrix},
\end{equation}
where the specific values depend on $(V, \eta, \theta_{\NY}, r_0)$.

For the parameter ranges studied in this paper, numerical evaluation shows 
that the Hessian is positive definite at Type I minima, confirming stability 
of the isotropic configuration.

\subsubsection{Physical interpretation}

The isotropic setting corresponds to a ``cubic torus'' where all three 
directions are equivalent. Perturbations toward anisotropy (e.g., elongation 
in one direction) increase the potential, so the system returns to the 
isotropic state.

This justifies using the isotropic reduction for phase classification, 
as it captures the stable configurations.

%------------------------------------------------------------------------------
\subsection{Alternative Reduction Scheme (Not Adopted)}
\label{app:alternatives}

\subsubsection{Alternative: Fix $R_2 = R_3 = 1$}

During the development of this work, an alternative approach was considered 
that fixes two dimensions and varies only $R_1 = r$:

\begin{equation}
  \Veff^{(\text{alt})}(r) = \frac{(2\pi)^4 L}{3\kappa^2} 
  \left[ V^2 r + 6V\eta\kappa^2\theta_{\NY} + \frac{9\eta^2}{r} \right].
\end{equation}

This gives a different functional form ($r + \text{const} + 1/r$) that 
does not match $\Sthree$ or $\Nilthree$.

\subsubsection{Comparison of phase structures}

\begin{table}[H]
\centering
\caption{Comparison of $\Tthree$ reduction schemes.}
\begin{tabular}{@{}lll@{}}
\toprule
Scheme & Potential structure & Compatible with $\Sthree$/$\Nilthree$? \\
\midrule
Isotropic ($R_1=R_2=R_3=r$) & $r^3 + r^2 + r$ & Yes \\
Fixed dimensions ($R_2=R_3=1$) & $r + \text{const} + 1/r$ & No \\
\bottomrule
\end{tabular}
\end{table}

The isotropic scheme is preferred for cross-topology comparison because 
it produces the same polynomial structure across all three test beds.

%------------------------------------------------------------------------------
\subsection{Summary}
\label{app:rescaling_summary}

\begin{enumerate}
  \item The isotropic setting $R_1 = R_2 = R_3 = r$ is adopted for $\Tthree$ 
        to enable fair comparison with $\Sthree$ and $\Nilthree$
  
  \item This setting produces the same $r \cdot (Ar^2 + Br + C)$ structure 
        as the other topologies
  
  \item Stability analysis confirms that isotropic minima are stable against 
        anisotropic perturbations in the parameter ranges studied
  
  \item The Type classification is independent of overall scale, depending 
        only on dimensionless parameter combinations
\end{enumerate}
  % Volume rescaling invariance in T3×S1
%==============================================================================
% Appendix F: Code and Data Access
%==============================================================================
\section{Code and Data Access}
\label{app:access}

This appendix provides information on accessing the computational code 
and data associated with this paper.

%------------------------------------------------------------------------------
\subsection{Code Repository}
\label{app:repository}

\subsubsection{Repository location}

The complete source code for this project is available at:

\begin{center}
\texttt{https://github.com/Muacca/DPPUv2-paper01}
\end{center}

\subsubsection{Repository structure}

\begin{verbatim}
DPPUv2/
|-- README.md                    # Project overview
|-- LICENSE                      # License information
|
|-- script/                         # Source code
|   |-- DPPUv2_engine_core_v3.py      # Core symbolic engine
|   |-- DPPUv2_parameter_scan_v3.py   # Parameter scanning
|   |-- DPPUv2_runner_S3S1_v3.py      # S3 topology runner
|   |-- DPPUv2_runner_T3S1_v3.py      # T3 topology runner
|   |-- DPPUv2_runner_Nil3S1_v3.py    # Nil3 topology runner
|   |-- DPPUv2_visualize_phasemap_v3.py    # Phase diagram generator
|   |-- DPPUv2_visualize_phasematrix_v3.py # Phase matrix generator
|   |-- DPPUv2_interactive_viewer_v3.py    # Interactive viewer
|   |-- DPPUv2_visualize_notebook_v3.ipynb # Visualize notebook
|   `-- requirements.txt                   # Python dependencies
|
|-- data/                        # Output data and image
|
|-- LaTeX/                       # LaTeX source and figures of the paper
|
`-- *.md                         # Documentation
\end{verbatim}

\subsubsection{File inventory}

\begin{table}[H]
\centering
\caption{Provided files and their associated appendices.}
\begin{tabular}{@{}llll@{}}
\toprule
Category & File & Description & Reference \\
\midrule
\textbf{Engine} & \texttt{DPPUv2\_engine\_core\_v3.py} & Core symbolic engine & App.~\ref{app:reproducibility} \\
 & \texttt{DPPUv2\_runner\_S3S1\_v3.py} & $\Sthree \times \Sone$ runner &  \\
 & \texttt{DPPUv2\_runner\_T3S1\_v3.py} & $\Tthree \times \Sone$ runner &  \\
 & \texttt{DPPUv2\_runner\_Nil3S1\_v3.py} & $\Nilthree \times \Sone$ runner &  \\
\midrule
\textbf{Scan} & \texttt{DPPUv2\_parameter\_scan\_v3.py} & Parameter scanning & App.~\ref{app:numerical} \\
\midrule
\textbf{Visualization} & \texttt{DPPUv2\_visualize\_notebook\_v3.ipynb} & Visualize notebook & App.~\ref{app:visualization} \\
 & \texttt{DPPUv2\_interactive\_viewer\_v3.py} & Interactive viewer &  \\
 & \texttt{DPPUv2\_visualize\_phasemap\_v3.py} & Phase diagram generator &  \\
 & \texttt{DPPUv2\_visualize\_phasematrix\_v3.py} & Phase matrix generator &  \\
\midrule
\textbf{Logs} & \texttt{*\_\{timestamp\}.log} & Verification logs & App.~\ref{app:reproducibility} \\
\midrule
\textbf{Data} & \texttt{dppu\_scan\_*.csv} & Scan results (CSV) & App.~\ref{app:numerical} \\
\bottomrule
\end{tabular}
\end{table}

%------------------------------------------------------------------------------
\subsection{Software Requirements}
\label{app:requirements}

\subsubsection{Python version}

Python 3.10 or higher is required.

\subsubsection{Dependencies}

\begin{table}[H]
\centering
\caption{Required Python packages.}
\begin{tabular}{@{}lll@{}}
\toprule
Package & Version & Purpose \\
\midrule
\texttt{numpy} & $\geq$ 1.20 & Numerical arrays \\
\texttt{sympy} & $\geq$ 1.12 & Symbolic computation \\
\texttt{mpmath} & $\geq$ 1.3 & Multi-precision arithmetic \\
\texttt{scipy} & $\geq$ 1.9 & Optimization (Brent's method) \\
\texttt{pandas} & $\geq$ 1.5 & Data handling \\
\texttt{matplotlib} & $\geq$ 3.5 & Visualization \\
\texttt{jupyter} & $\geq$ 1.0 & Interactive notebooks \\
\texttt{ipywidgets} & $\geq$ 8.0 & Interactive viewer widgets \\
\texttt{ipympl} & $\geq$ 0.9 & Jupyter matplotlib backend \\
\bottomrule
\end{tabular}
\end{table}

\subsubsection{Installation}

\begin{lstlisting}[language=bash, basicstyle=\ttfamily\footnotesize]
# Clone repository
git clone [repository-url]
cd DPPUv2

# Create virtual environment (recommended)
python -m venv venv
source venv/bin/activate  # Linux/Mac
# or: venv\Scripts\activate  # Windows

# Install dependencies
pip install -r requirements.txt
\end{lstlisting}

For JupyterLab users, additional extensions may be required:
\begin{lstlisting}[language=bash, basicstyle=\ttfamily\footnotesize]
# JupyterLab widget support (if needed)
jupyter labextension install @jupyter-widgets/jupyterlab-manager jupyter-matplotlib
\end{lstlisting}

%------------------------------------------------------------------------------
\subsection{Quick Start Guide}
\label{app:quickstart}

\subsubsection{Running a single topology}

\begin{lstlisting}[language=bash, basicstyle=\ttfamily\footnotesize]
# Run S3 topology with all variants
python src/DPPUv2_runner_S3S1_v3.py

# Output: logs and symbolic expressions to stdout
# Verification results saved to data/logs/
\end{lstlisting}

\subsubsection{Running parameter scans}

\begin{lstlisting}[language=bash, basicstyle=\ttfamily\footnotesize]
# Full parameter scan (default: all topologies, all variants)
python DPPUv2_parameter_scan_v3.py

# Single topology/variant
python DPPUv2_parameter_scan_v3.py \
    --topologies S3 --ny-variants FULL

# Custom parameter ranges
python DPPUv2_parameter_scan_v3.py \
    --topologies S3 --ny-variants FULL \
    --V-min 0.0 --V-max 5.0 --V-points 51 \
    --eta-min -10.0 --eta-max 5.0 --eta-points 151 \
    --theta-min 0.0 --theta-max 5.0 --theta-points 51

# Output: CSV files in output/
\end{lstlisting}

\subsubsection{Generating figures}

\begin{lstlisting}[language=bash, basicstyle=\ttfamily\footnotesize]
# Generate phase diagrams
python src/DPPUv2_visualize_phasemap_v3.py \
    --input data/scans/S3S1_FULL_MX_*.csv \
    --output figures/

# Generate phase matrix
python src/DPPUv2_visualize_phasematrix_v3.py \
    --theta 1.0 \
    --output figures/fig12_phase_matrix_0010.png
\end{lstlisting}

\subsubsection{Interactive exploration}

\begin{lstlisting}[language=bash, basicstyle=\ttfamily\footnotesize]
# Launch Jupyter notebook
jupyter notebook DPPUv2_visualize_notebook_v3.ipynb
\end{lstlisting}

In the notebook, run:
\begin{lstlisting}[language=Python, basicstyle=\ttfamily\footnotesize]
%matplotlib widget
from DPPUv2_interactive_viewer_v3 import DPPUv2InteractiveViewer

viewer = DPPUv2InteractiveViewer()
viewer.display()
\end{lstlisting}

%------------------------------------------------------------------------------
\subsection{Data Files}
\label{app:data_files}

\subsubsection{Parameter scan results}

Scan results are provided as CSV files with naming convention:
\begin{center}
\texttt{<topology>\_<variant>\_<mode>\_<timestamp>.csv}
\end{center}

Example files included in the repository:

\begin{itemize}
  \item \texttt{S3S1\_FULL\_MX\_20241214\_135031.csv}
  \item \texttt{S3S1\_TT\_MX\_20241214\_135031.csv}
  \item \texttt{S3S1\_REE\_MX\_20241214\_135031.csv}
  \item \texttt{T3S1\_FULL\_MX\_20241214\_135031.csv}
  \item (etc., 9 files total)
\end{itemize}

\subsubsection{Verification logs}

Log files documenting sanity checks:

\begin{itemize}
  \item \texttt{S3S1\_*\_20241214\_135031.log}
  \item \texttt{T3S1\_*\_20241214\_135031.log}
  \item \texttt{Nil3S1\_*\_20241214\_135031.log}
\end{itemize}

These logs contain:
\begin{itemize}
  \item Metric compatibility verification results
  \item Riemann tensor antisymmetry checks
  \item NY density decomposition verification
  \item Symbolic expressions for all computed quantities
\end{itemize}

%------------------------------------------------------------------------------
\subsection{License}
\label{app:license}

The code and data are released under the MIT License:

\begin{quote}
\small
MIT License

Copyright (c) 2026 Muacca

Permission is hereby granted, free of charge, to any person obtaining a copy
of this software and associated documentation files (the ``Software''), to deal
in the Software without restriction, including without limitation the rights
to use, copy, modify, merge, publish, distribute, sublicense, and/or sell
copies of the Software, and to permit persons to whom the Software is
furnished to do so, subject to the following conditions:

The above copyright notice and this permission notice shall be included in all
copies or substantial portions of the Software.

THE SOFTWARE IS PROVIDED ``AS IS'', WITHOUT WARRANTY OF ANY KIND, EXPRESS OR
IMPLIED, INCLUDING BUT NOT LIMITED TO THE WARRANTIES OF MERCHANTABILITY,
FITNESS FOR A PARTICULAR PURPOSE AND NONINFRINGEMENT. IN NO EVENT SHALL THE
AUTHORS OR COPYRIGHT HOLDERS BE LIABLE FOR ANY CLAIM, DAMAGES OR OTHER
LIABILITY, WHETHER IN AN ACTION OF CONTRACT, TORT OR OTHERWISE, ARISING FROM,
OUT OF OR IN CONNECTION WITH THE SOFTWARE OR THE USE OR OTHER DEALINGS IN THE
SOFTWARE.
\end{quote}

%------------------------------------------------------------------------------
\subsection{Version History}
\label{app:version_history}

\begin{table}[H]
\centering
\caption{Version history of DPPUv2 code.}
\begin{tabular}{@{}lll@{}}
\toprule
Version & Date & Major Changes \\
\midrule
v1.0 & 2025-11 & Initial implementation (teleparallel gravity base) \\
v2.0 & 2025-12 & Migration to Einstein--Cartan framework; $\Sthree \times \Sone$ implementation \\
v3.0 & 2025-12 & Mode/Variant enumeration; 3-stage Riemann verification; \\
     &         & $\Tthree \times \Sone$ and $\Nilthree \times \Sone$ added \\
\bottomrule
\end{tabular}
\end{table}

All results in this paper are based on v3.0.

%------------------------------------------------------------------------------
\subsection{Citation}
\label{app:citation}

If you use this code or data in your research, please cite:

\begin{quote}
\small
Muacca, ``Topology-Dependent Phase Classification of Effective Potentials 
in Einstein--Cartan + Nieh--Yan Minisuperspace,'' 
arXiv:[to be assigned] (2026).
\end{quote}

The Zenodo DOI for the code repository is:
\begin{center}
\texttt{10.5281/zenodo.[to be assigned]}
\end{center}

%------------------------------------------------------------------------------
\subsection{Contact}
\label{app:contact}

For questions, bug reports, or collaboration inquiries:

\begin{itemize}
  \item Email: \texttt{muacca@dmwp.jp}
  \item GitHub Issues: \texttt{https://github.com/Muacca/DPPUv2-paper01/issues}
\end{itemize}
  % Code and data access

%------------------------------------------------------------------------------
% References
%------------------------------------------------------------------------------
\bibliographystyle{unsrt}
% \bibliography{references}  % Uncomment when bibliography file is available

% Placeholder references
\begin{thebibliography}{99}
\bibitem{Hehl1976}
F.~W.~Hehl, P.~von der Heyde, G.~D.~Kerlick, and J.~M.~Nester,
``General relativity with spin and torsion: Foundations and prospects,''
Rev.\ Mod.\ Phys.\ \textbf{48}, 393 (1976).

\bibitem{Chandia1997}
O.~Chandia and J.~Zanelli,
``Topological invariants, instantons, and the chiral anomaly on spaces with torsion,''
Phys.\ Rev.\ D \textbf{55}, 7580 (1997).

\bibitem{NiehYan1982}
H.~T.~Nieh and M.~L.~Yan,
``An identity in Riemann-Cartan geometry,''
J.\ Math.\ Phys.\ \textbf{23}, 373 (1982).
\end{thebibliography}

%==============================================================================
\end{document}
%==============================================================================
