%==============================================================================
% Section 6: Representative Points and Stability Quality
%==============================================================================
\section{Representative Points and Stability Quality}
\label{sec:representative}

This section presents the specific shapes of the effective potential $\Veff(r)$ 
and stability metrics at representative parameter points on the phase diagrams. 
This clarifies how the operational definitions of Type I/II/III 
(Sec.~\ref{sec:type_definition}) correspond to actual potential shapes.

%------------------------------------------------------------------------------
\subsection{Selection Criteria for Representative Points}
\label{sec:selection_criteria}

For each topology, representative points are selected according to the 
following criteria:

\begin{enumerate}
  \item \textbf{Type I representative}: Points showing metastable wells with 
        barriers. $\Delta V > 0$ and $\dd\Veff/\dd r|_{r \to 0^+} > 0$
  \item \textbf{Type II representative}: Points showing rolling. Minimum exists 
        but no barrier in $r \to 0$ direction
  \item \textbf{Type III representative}: Points showing instability. No stable 
        minimum within allowed region
  \item \textbf{Near phase boundary}: Points in parameter regions where Type 
        transitions occur
\end{enumerate}

Below, we fix $\kappa = L = 1$ and focus on results for the FULL variant.

%------------------------------------------------------------------------------
\subsection{Representative Points for $\Sthree \times \Sone$}
\label{sec:S3_representatives}

\subsubsection{List of representative points}

\begin{table}[H]
\centering
\caption{$\Sthree$-FULL representative points and stability metrics.}
\label{tab:S3_representatives}
\begin{tabular}{@{}ccccccc@{}}
\toprule
Point & $(V, \eta, \theta_{\NY})$ & Type & $r_0$ & $\log_{10}(r_0)$ & 
$\Delta V$ & $\log_{10}(\Delta V)$ \\
\midrule
S1 & $(2.0, -4.0, 1.0)$ & II & $1.26$ & $0.10$ & --- & --- \\
S2 & $(2.0, -7.0, 1.0)$ & I & $3.98$ & $0.60$ & $1.2 \times 10^3$ & $3.1$ \\
S3 & $(2.0, 0.0, 1.0)$ & I & $0.32$ & $-0.50$ & $2.1 \times 10^2$ & $2.3$ \\
S4 & $(2.0, 1.0, 1.0)$ & III & --- & --- & --- & --- \\
S5 & $(2.0, -2.0, 1.0)$ & I/II boundary & $0.63$ & $-0.20$ & $\approx 0$ & --- \\
S6 & $(2.0, -4.0, 2.0)$ & II & $2.51$ & $0.40$ & --- & --- \\
\bottomrule
\end{tabular}
\end{table}

\subsubsection{Characteristics of each representative point}

\paragraph{Point S1 (Type II, $\eta = -4$):}
This is the point where $C = 9(\eta + 4)^2 - 36 = -36$ takes its minimum value. 
$\Veff(r)$ starts from negative values as $r \to 0$, monotonically increases, 
passes through a minimum, and increases again. Near the origin, 
$\dd\Veff/\dd r < 0$, so no barrier exists.

\paragraph{Point S2 (Type I, $\eta = -7$):}
$C = 9(-7 + 4)^2 - 36 = 45 > 0$. $\Veff(r)$ starts from positive values as 
$r \to 0$, forms a barrier before falling into the well. Barrier height 
$\Delta V \approx 10^3$ suggests strong suppression of quantum tunneling 
from the metastable state.

\paragraph{Point S3 (Type I, $\eta = 0$):}
$C = 9(4)^2 - 36 = 108 > 0$. Due to the effect of $\theta_{\NY} = 1$, 
$B = 6V\kappa^2\theta_{\NY}(\eta - 4) = -48 < 0$, and a minimum forms even 
at $\eta = 0$. This is an example where a region that was Type III at 
$\theta_{\NY} = 0$ transitions to Type I at $\theta_{\NY} > 0$.

\paragraph{Point S4 (Type III, $\eta = 1$):}
$C = 9(5)^2 - 36 = 189 > 0$, but the effect of $B = -36$ is insufficient, 
and no minimum forms within the allowed region. Search reaches $r_{\max}$.

\paragraph{Point S5 (I/II boundary, $\eta = -2$):}
This is the critical point where $C = 9(2)^2 - 36 = 0$. 
$\dd\Veff/\dd r|_{r=0} = 0$, positioned at the boundary between Type I and Type II.

%------------------------------------------------------------------------------
\subsection{Representative Points for $\Tthree \times \Sone$}
\label{sec:T3_representatives}

\subsubsection{List of representative points}

\begin{table}[H]
\centering
\caption{$\Tthree$-FULL representative points and stability metrics.}
\label{tab:T3_representatives}
\begin{tabular}{@{}ccccccc@{}}
\toprule
Point & $(V, \eta, \theta_{\NY})$ & Type & $r_0$ & $\log_{10}(r_0)$ & 
$\Delta V$ & $\log_{10}(\Delta V)$ \\
\midrule
T1 & $(2.0, -3.0, 1.0)$ & I & $2.5$ & $0.40$ & $4.2 \times 10^2$ & $2.6$ \\
T2 & $(2.0, -5.0, 1.5)$ & I & $5.0$ & $0.70$ & $2.1 \times 10^3$ & $3.3$ \\
T3 & $(2.0, 0.0, 1.0)$ & III & --- & --- & --- & --- \\
T4 & $(2.0, 2.0, 1.0)$ & III & --- & --- & --- & --- \\
T5 & $(2.0, -2.0, 0.5)$ & III & --- & --- & --- & --- \\
\bottomrule
\end{tabular}
\end{table}

\textbf{Note:} For $\theta_{\NY} < 0.87$, the entire parameter region is Type III. 
Type I appears only for $\theta_{\NY} \geq 0.87$ and $\eta < 0$.

\subsubsection{Characteristics of each representative point}

\paragraph{Points T1, T2 (Type I, $\eta < 0$, $\theta_{\NY} \geq 1$):}
With the isotropic setting for $\Tthree$, Type I (metastable well with barrier) 
forms when $\theta_{\NY}$ exceeds the threshold ($\approx 0.87$) and $\eta < 0$. 
The more negative $\eta$ and the larger $\theta_{\NY}$, the larger $r_0$ and 
$\Delta V$ become.

\paragraph{Points T3, T4 (Type III, $\eta \geq 0$ or small $\theta_{\NY}$):}
In the region $\eta \geq 0$, or when $\theta_{\NY} < 0.87$, 
$B = 6V\eta\kappa^2\theta_{\NY}$ does not contribute to minimum formation 
($B \geq 0$ or $B$'s effect is weak), resulting in Type III.

\paragraph{Point T5 (Type III, $\theta_{\NY}$ below threshold):}
Even with $\eta = -2 < 0$, at $\theta_{\NY} = 0.5 < 0.87$, the Type I condition 
$B^2 > 3V^2 C$ is not satisfied, resulting in Type III. This is numerical 
confirmation of the threshold condition derived in Sec.~\ref{sec:T3_structure}.

\subsubsection{Confirmation of $\eta \to -\eta$ asymmetry}

With the isotropic setting for $\Tthree$, since $B = 6V\eta\kappa^2\theta_{\NY}$ 
is linear in $\eta$, $\eta \to -\eta$ symmetry is broken for $\theta_{\NY} > 0$. 
In Table~\ref{tab:T3_representatives}:

\begin{itemize}
  \item $\eta < 0$ (points T1, T2): $B < 0$, Type I possible
  \item $\eta > 0$ (point T4): $B > 0$, Type III
\end{itemize}

This asymmetry is visualized in Figure~\ref{fig:phase_T3}, where stable regions 
appear only on the $\eta < 0$ side. In the limit $\theta_{\NY} = 0$, $B = 0$ 
and symmetry is restored, but in this case the entire region is Type III.

%------------------------------------------------------------------------------
\subsection{Representative Points for $\Nilthree \times \Sone$}
\label{sec:Nil3_representatives}

\subsubsection{List of representative points}

\begin{table}[H]
\centering
\caption{$\Nilthree$-FULL representative points and stability metrics.}
\label{tab:Nil3_representatives}
\begin{tabular}{@{}ccccccc@{}}
\toprule
Point & $(V, \eta, \theta_{\NY})$ & Type & $r_0$ & $\log_{10}(r_0)$ & 
$\Delta V$ & $\log_{10}(\Delta V)$ \\
\midrule
N1 & $(2.0, 0.5, 1.0)$ & II & $0.20$ & $-0.70$ & --- & --- \\
N2 & $(2.0, -5.0, 1.0)$ & I & $2.51$ & $0.40$ & $8.9 \times 10^2$ & $2.95$ \\
N3 & $(2.0, -5.0, 2.0)$ & I & $5.01$ & $0.70$ & $3.5 \times 10^3$ & $3.55$ \\
N4 & $(2.0, 2.0, 1.0)$ & III & --- & --- & --- & --- \\
N5 & $(2.0, -0.3, 1.0)$ & II/III boundary & $\approx 0.01$ & $\approx -2$ & --- & --- \\
\bottomrule
\end{tabular}
\end{table}

\subsubsection{Characteristics of each representative point}

\paragraph{Point N1 (Type II, within main band):}
$\eta = 0.5$ is in the region where $C = 36(0.5 - 1/3)^2 - 13 \approx -12 < 0$, 
located within the main band. $r_0$ is small ($\log_{10}(r_0) \approx -0.7$), 
consistent with the main band in the phase diagram being displayed in 
purple-blue colors.

\paragraph{Points N2, N3 (Type I, separate stable region):}
At $\eta = -5$, $C = 36(-5 - 1/3)^2 - 13 \approx 1010 > 0$, but 
$B = 8V\kappa^2\theta_{\NY}(3\eta + 1) = -224\theta_{\NY} < 0$ is sufficiently 
large to form a minimum. Increasing $\theta_{\NY}$ from 1 to 2 approximately 
doubles $r_0$ and quadruples $\Delta V$. This shows that the separate stable 
region expands and deepens with increasing $\theta_{\NY}$.

\paragraph{Point N4 (Type III, outside main band):}
At $\eta = 2$, $C = 36(2 - 1/3)^2 - 13 \approx 87 > 0$, and 
$B = 8V\kappa^2\theta_{\NY}(7) = 112 > 0$. Since $B > 0$, minimum formation 
is inhibited, resulting in Type III.

\paragraph{Point N5 (II/III boundary):}
$\eta \approx -0.27$ is close to the critical value where $C = 0$. A minimum 
exists but $r_0 \approx r_{\min}$, barely determined as Type II by boundary 
criteria.

%------------------------------------------------------------------------------
\subsection{Visualization of Stability Metrics}
\label{sec:metrics_visualization}

\subsubsection{Distribution of $\log_{10}(r_0)$}

The color gradient in phase diagrams (purple $\to$ yellow) represents 
$\log_{10}(r_0)$. Typical ranges for each topology:

\begin{table}[H]
\centering
\begin{tabular}{@{}lll@{}}
\toprule
Topology & $\log_{10}(r_0)$ range & Characteristics \\
\midrule
$\Sthree$ & $[-0.5, 3.0]$ & Increases in negative $\eta$ direction \\
$\Tthree$ & $[0, 5]$ & Proportional to $|\eta|/V$ \\
$\Nilthree$ & $[-2, 3]$ & Small in main band, large in separate region \\
\bottomrule
\end{tabular}
\end{table}

\subsubsection{Distribution of $\log_{10}(\Delta V)$ (Type I only)}

White contour lines in phase diagrams represent $\log_{10}(\Delta V)$. 
Typical barrier heights:

\begin{itemize}
  \item \textbf{$\Sthree$}: $\log_{10}(\Delta V) \in [2, 5]$ in Type I region. 
        Increases with $|\eta|$
  \item \textbf{$\Tthree$}: $\log_{10}(\Delta V) \in [2, 4]$ in Type I region 
        with $\theta_{\NY} \geq 0.87$ and $\eta < 0$. Increases with $|\eta|$ 
        and $\theta_{\NY}$
  \item \textbf{$\Nilthree$}: $\log_{10}(\Delta V) \in [2, 5]$ in separate 
        stable region. Increases with $\theta_{\NY}$
\end{itemize}

\subsubsection{Correlation between barrier height and stable radius}

From the contour patterns in phase diagrams, the following trends can be read:

\begin{enumerate}
  \item \textbf{$\Sthree$}: $\log_{10}(\Delta V)$ and $\log_{10}(r_0)$ show 
        positive correlation. Points with larger $r_0$ have deeper wells 
        and higher barriers
  \item \textbf{$\Tthree$, $\Nilthree$}: Similar positive correlation observed 
        in stable regions with $\eta < 0$
\end{enumerate}

This correlation can be understood from the scaling law 
$\Veff \propto r \times P(r)$ of the effective potential. As $r_0$ increases, 
the region $r < r_0$ forming the barrier also widens, consequently increasing 
$\Delta V$.

Scaling laws for each topology are shown in 
Figures~\ref{fig:scaling_S3}--\ref{fig:scaling_Nil3}.

\begin{figure}[tbp]
  \centering
  \includegraphics[width=0.9\textwidth]{figures/fig16_Scaling_laws_S3.png}
  \caption{Scaling laws: $\Sthree \times \Sone$ ($\eta = -7.0$, $\theta_{\NY} = 2.0$).}
  \label{fig:scaling_S3}
\end{figure}

\begin{figure}[tbp]
  \centering
  \includegraphics[width=0.9\textwidth]{figures/fig17_Scaling_laws_T3.png}
  \caption{Scaling laws: $\Tthree \times \Sone$ ($\eta = -7.0$, $\theta_{\NY} = 2.0$).}
  \label{fig:scaling_T3}
\end{figure}

\begin{figure}[tbp]
  \centering
  \includegraphics[width=0.9\textwidth]{figures/fig18_Scaling_laws_Nil3.png}
  \caption{Scaling laws: $\Nilthree \times \Sone$ ($\eta = -7.0$, $\theta_{\NY} = 2.0$).}
  \label{fig:scaling_Nil3}
\end{figure}

%------------------------------------------------------------------------------
\subsection{Consistency Check Between Analytical Boundaries and Numerical Scan}
\label{sec:consistency_check}

\subsubsection{Type I/II boundary for $\Sthree$}

The analytical boundary condition $C(\eta) = 9(\eta + 4)^2 - 36 = 0$ derived 
in Sec.~\ref{sec:type_transitions} gives $\eta = -2$ and $\eta = -6$.

Comparison with numerical scan results:

\begin{table}[H]
\centering
\begin{tabular}{@{}llll@{}}
\toprule
Boundary & Analytical value & Numerical scan result ($V = 2$, $\theta_{\NY} = 1$) & Deviation \\
\midrule
Upper I/II & $\eta = -2$ & $\eta \approx -2.0$ & $< 0.1$ \\
Lower II/I & $\eta = -6$ & $\eta \approx -5.9$ & $\approx 0.1$ \\
\bottomrule
\end{tabular}
\end{table}

Deviations are on the order of scan grid spacing ($\Delta\eta = 0.1$), showing 
good agreement between analytical predictions and numerical results.

\subsubsection{Type I/III boundary for $\Tthree$}

For $\Tthree$ with isotropic setting, the boundary between Type I and Type III 
is determined by the condition derived in Sec.~\ref{sec:T3_structure}:
\begin{equation}
  \theta_{\NY} = \frac{\sqrt{3}}{2\kappa^2} \approx 0.87 \quad (\kappa = 1).
\end{equation}

Comparison with numerical scan results:

\begin{table}[H]
\centering
\begin{tabular}{@{}llll@{}}
\toprule
Boundary & Analytical value & Numerical scan result ($V = 2$, $\eta = -3$) & Deviation \\
\midrule
$\theta_{\NY}$ threshold & $0.87$ & $\theta_{\NY} \approx 0.9$ & $\approx 0.03$ \\
\bottomrule
\end{tabular}
\end{table}

Additionally, the $\eta = 0$ axis is the boundary where $C = 0$, and for 
$\theta_{\NY} > 0.87$, the Type I/III boundary is observed near this location. 
Type I appears only on the $\eta < 0$ side, with the entire $\eta > 0$ side 
being Type III.

\subsubsection{Main band boundary for $\Nilthree$}

The $C(\eta) = 0$ condition derived in Sec.~\ref{sec:Nil3_structure} gives 
$\eta \approx -0.27$ and $\eta \approx 0.93$.

Comparison with numerical scan results:

\begin{table}[H]
\centering
\begin{tabular}{@{}llll@{}}
\toprule
Boundary & Analytical value & Numerical scan result ($\theta_{\NY} = 0$) & Deviation \\
\midrule
Lower & $\eta \approx -0.27$ & $\eta \approx -0.3$ & $\approx 0.03$ \\
Upper & $\eta \approx 0.93$ & $\eta \approx 1.0$ & $\approx 0.07$ \\
\bottomrule
\end{tabular}
\end{table}

Deviations are on the order of grid spacing, confirming consistency.

%------------------------------------------------------------------------------
\subsection{Summary of This Section}
\label{sec:representative_summary}

This section quantitatively presented stability metrics at representative 
points for each topology and confirmed the following:

\begin{enumerate}
  \item \textbf{Operational definition of Type classification} corresponds 
        consistently with actual $\Veff(r)$ shapes
  \item \textbf{$\log_{10}(r_0)$ and $\log_{10}(\Delta V)$} show systematic 
        patterns on phase diagrams with clear correlation to geometric parameters
  \item \textbf{Analytical boundary conditions} (derived in 
        Sec.~\ref{sec:mechanisms}) and \textbf{numerical scan results} are 
        consistent within grid resolution
\end{enumerate}

These representative point data can be used as reference points in EC+NY 
minisuperspace analysis, and as input for future dynamical analysis 
(WKB tunneling rate calculations, etc.).
