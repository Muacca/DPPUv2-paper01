%==============================================================================
% Section 4: Numerical Results: Phase Diagrams and Boundaries
%==============================================================================
\section{Numerical Results: Phase Diagrams and Boundaries}
\label{sec:numerical}

This section presents the results of numerical scanning on the effective 
potentials derived in Sec.~\ref{sec:reductions}. We first explain the Type 
classification procedure and how to read phase diagrams, then show the phase 
diagrams for each topology. The mechanisms of phase boundary formation and 
geometric interpretation are discussed in Sec.~\ref{sec:mechanisms}.

%------------------------------------------------------------------------------
\subsection{Type Classification Procedure}
\label{sec:classification_procedure}

\subsubsection{Search region and numerical method}

For each parameter point $(V, \eta, \theta_{\NY})$, we numerically search for 
extrema of $\Veff(r)$ in the range $r \in [r_{\min}, r_{\max}] = [0.01, 10^6]$. 
We use Brent's method (\texttt{scipy.optimize.minimize\_scalar}) for extremum 
search and verify the curvature condition $\dd^2 V/\dd r^2 > 0$ by numerical 
differentiation.

\subsubsection{Type determination criteria}

Based on the definitions in Sec.~\ref{sec:type_definition}, we determine Types 
using the following criteria:

\begin{itemize}
  \item \textbf{Type I (stable with barrier)}: A local minimum $r_0$ exists 
        within the allowed region, and a barrier ($\Delta V > 0$) exists in 
        the $r \to 0$ direction
  \item \textbf{Type II (rolling)}: A local minimum $r_0$ exists within the 
        allowed region, but no barrier in the $r \to 0$ direction 
        ($\dd\Veff/\dd r|_{r \to 0^+} < 0$)
  \item \textbf{Type III (unstable)}: No stable local minimum exists within 
        the allowed region, or search reaches boundary ($r_{\min}$ or $r_{\max}$)
\end{itemize}

\subsubsection{Scan resolution}

The scan resolution is 51 points in the $V$ direction, 151 points in the 
$\eta$ direction, and 51 points in the $\theta_{\NY}$ direction (range $[0, 5]$), 
giving approximately 390,000 evaluations per topology. Note that this paper 
primarily presents results for $\theta_{\NY} \lesssim 2$. The uncertainty in 
boundary positions is on the order of the grid spacing 
($\Delta\eta \approx 0.1$, $\Delta V \approx 0.1$).

%------------------------------------------------------------------------------
\subsection{How to Read Phase Diagrams}
\label{sec:reading_diagrams}

In the phase diagrams of this paper, Type classification is displayed on the 
$(V, \eta)$ plane using the following visual conventions:

\begin{itemize}
  \item \textbf{Colored region (no hatching)}: Type I (metastable well with barrier)
  \item \textbf{Colored region (with hatching)}: Type II (rolling, no barrier)
  \item \textbf{White region}: Type III (unstable / boundary-attached)
\end{itemize}

The color gradient represents the logarithmic value of the stable radius 
$\log_{10}(r_0)$, ranging from purple (small) to yellow (large). White contour 
lines show the logarithmic barrier height/well depth $\log_{10}(\Delta V)$.

%------------------------------------------------------------------------------
\subsection{Phase Diagrams by Topology: FULL Variant}
\label{sec:phase_diagrams}

\subsubsection{$\Sthree \times \Sone$}

Figure~\ref{fig:phase_S3} shows the phase diagrams for $\Sthree$-FULL at 
$\theta_{\NY} = 0, 1, 2$.

\begin{figure}[htbp]
  \centering
  \includegraphics[width=\textwidth]{figures/fig09_phase_diagram_S3-FULL.png}
  \caption{Phase diagram: $\Sthree \times \Sone$ ($\theta_{\NY} = 0, 1, 2$).}
  \label{fig:phase_S3}
\end{figure}

\paragraph{For $\theta_{\NY} = 0$:}
\begin{itemize}
  \item $\eta \gtrsim -2$: Type III (white) dominates
  \item $-6 \lesssim \eta \lesssim -2$: Band-like region of Type II (hatched)
  \item $\eta \lesssim -6$: Type III again
\end{itemize}

\paragraph{For $\theta_{\NY} = 1$:}
\begin{itemize}
  \item $\eta \gtrsim 0.5$: Type III (white)
  \item $-2 \lesssim \eta \lesssim 0.5$: Type I (no hatching)
  \item $-6 \lesssim \eta \lesssim -2$: Band-like region of Type II (hatched)
  \item $\eta \lesssim -6$: Type I (no hatching) reappears
\end{itemize}

\paragraph{For $\theta_{\NY} = 2$:}
\begin{itemize}
  \item Stable region (Type I + II) expands toward $\eta > 0$
  \item I/III boundary moves to around $\eta \approx 2$
  \item Type II band also expands
\end{itemize}

\paragraph{Key observations:}
\begin{enumerate}
  \item $\theta_{\NY} = 0$: Only main band of Type II
  \item $\theta_{\NY} > 0$: Type I region appears and expands
  \item $\eta \to -\eta$ symmetry is broken
\end{enumerate}

\subsubsection{$\Tthree \times \Sone$}

Figure~\ref{fig:phase_T3} shows the phase diagrams for $\Tthree$-FULL at 
$\theta_{\NY} = 0, 1, 2$.

\begin{figure}[htbp]
  \centering
  \includegraphics[width=\textwidth]{figures/fig10_phase_diagram_T3-FULL.png}
  \caption{Phase diagram: $\Tthree \times \Sone$ ($\theta_{\NY} = 0, 1, 2$).}
  \label{fig:phase_T3}
\end{figure}

\paragraph{For $\theta_{\NY} = 0$:}
\begin{itemize}
  \item Entire region is Type III (monotonically increasing)
  \item No stable minimum exists
\end{itemize}

\paragraph{For $\theta_{\NY} = 1$:}
\begin{itemize}
  \item Type I (stable well) appears in wide region of $\eta < 0$
  \item $\eta \gtrsim 0$ is entirely Type III
\end{itemize}

\paragraph{For $\theta_{\NY} = 2$:}
\begin{itemize}
  \item Type I region expands further
  \item Contour lines of $r_0$ extend in negative $\eta$ direction
\end{itemize}

\paragraph{Key observations:}
\begin{enumerate}
  \item $\theta_{\NY} = 0$: Entire region Type III
  \item $\theta_{\NY} > 0$: Type I appears
  \item $\eta \to -\eta$ symmetry is broken
\end{enumerate}

\subsubsection{$\Nilthree \times \Sone$}

Figure~\ref{fig:phase_Nil3} shows the phase diagrams for $\Nilthree$-FULL at 
$\theta_{\NY} = 0, 1, 2$.

\begin{figure}[htbp]
  \centering
  \includegraphics[width=\textwidth]{figures/fig11_phase_diagram_Nil3-FULL.png}
  \caption{Phase diagram: $\Nilthree \times \Sone$ ($\theta_{\NY} = 0, 1, 2$).}
  \label{fig:phase_Nil3}
\end{figure}

\paragraph{For $\theta_{\NY} = 0$:}
\begin{itemize}
  \item $-0.3 \lesssim \eta \lesssim 1$: Narrow stable band (Type II)
  \item $\eta < -0.3$ and $\eta > 1$: Type III (white) dominates almost 
        the entire region
\end{itemize}

\paragraph{For $\theta_{\NY} = 1$:}
\begin{itemize}
  \item Main band ($-0.3 < \eta < 1$) is maintained
  \item \textbf{Separate stable region (Type I)} appears at $\eta \lesssim -4$
  \item Main band and separate stable region are separated by Type III region
\end{itemize}

\paragraph{For $\theta_{\NY} = 2$:}
\begin{itemize}
  \item Lower stable region expands, distributing widely at $\eta \lesssim -0.5$
  \item Width of main band ($-0.3 < \eta < 1$) gradually decreases in 
        large $V$ region
\end{itemize}

\paragraph{Key observations:}
\begin{enumerate}
  \item $\theta_{\NY} = 0$: Only main band of Type II
  \item $\theta_{\NY} > 0$: Type I region appears and expands
  \item $\eta \to -\eta$ symmetry is broken
\end{enumerate}

%------------------------------------------------------------------------------
\subsection{Summary of $\theta_{\NY}$ Dependence}
\label{sec:theta_dependence}

Tables~\ref{tab:boundary_S3}--\ref{tab:boundary_Nil3} summarize the 
$\theta_{\NY}$ dependence of phase boundaries for each topology.

\begin{table}[H]
\centering
\caption{$\Sthree$: $\theta_{\NY}$ dependence of phase boundary positions 
(FULL variant).}
\label{tab:boundary_S3}
\begin{tabular}{@{}llll@{}}
\toprule
Type & $\theta_{\NY} = 0$ & $\theta_{\NY} = 1$ & $\theta_{\NY} = 2$ \\
\midrule
Type I region & None & $\eta < -6$, $-2 < \eta < 0.5$ & Expanded \\
Type II (main band) & $-6 \lesssim \eta \lesssim -2$ & 
                      $-6 \lesssim \eta \lesssim -2$ & 
                      $-6 \lesssim \eta \lesssim -2$ \\
Type III region & $\eta \lesssim -6$, $-2 \lesssim \eta$ & 
                  $0.5 \lesssim \eta$ & Shrunk \\
\bottomrule
\end{tabular}
\end{table}

\begin{table}[H]
\centering
\caption{$\Tthree$: $\theta_{\NY}$ dependence of phase boundary positions 
(FULL variant).}
\label{tab:boundary_T3}
\begin{tabular}{@{}llll@{}}
\toprule
Type & $\theta_{\NY} = 0$ & $\theta_{\NY} = 1$ & $\theta_{\NY} = 2$ \\
\midrule
Type I region & None & $\eta \lesssim 0$ & $\eta \lesssim 0$ \\
Type II region & None & None & None \\
Type III region & Entire & $0 \lesssim \eta$ & $0 \lesssim \eta$ \\
\bottomrule
\end{tabular}
\end{table}

\begin{table}[H]
\centering
\caption{$\Nilthree$: $\theta_{\NY}$ dependence of phase boundary positions 
(FULL variant).}
\label{tab:boundary_Nil3}
\begin{tabular}{@{}llll@{}}
\toprule
Type & $\theta_{\NY} = 0$ & $\theta_{\NY} = 1$ & $\theta_{\NY} = 2$ \\
\midrule
Type I region & None & $\eta \lesssim -4$ & $\eta < -0.5$ \\
Type II (main band) & $-0.3 \lesssim \eta \lesssim 1$ & 
                      $-0.3 \lesssim \eta \lesssim 1$ & 
                      $-0.3 \lesssim \eta \lesssim 1$ \\
Type III region & $\eta \lesssim -0.3$, $1 \lesssim \eta$ & 
                  $-4 \lesssim \eta \lesssim -0.3$, $1 \lesssim \eta$ & 
                  $-0.5 \lesssim \eta \lesssim -0.3$, $1 \lesssim \eta$ \\
\bottomrule
\end{tabular}
\end{table}

\paragraph{Key observations:}
\begin{enumerate}
  \item \textbf{$\Sthree$}: Shows strong dependence on $\theta_{\NY}$, with 
        stable region expanding toward $\eta > 0$
  \item \textbf{$\Tthree$}: Entire region is Type III for 
        $\theta_{\NY} \lesssim 0.9$, but Type I appears in $\eta < 0$ for 
        $\theta_{\NY} \gtrsim 0.9$
  \item \textbf{$\Nilthree$}: Main band is insensitive to $\theta_{\NY}$, 
        but stable region in $\eta < 0$ grows with $\theta_{\NY}$
\end{enumerate}

%------------------------------------------------------------------------------
\subsection{Comparison with TT/REE Variants}
\label{sec:variant_comparison}

Figure~\ref{fig:phase_matrix} shows the phase diagram matrix for 
3 topologies $\times$ 3 variants at $\theta_{\NY} = 1.0$.

\begin{figure}[htbp]
  \centering
  \includegraphics[width=\textwidth]{figures/fig12_phase_matrix_0010.png}
  \caption{Phase diagram matrix: $\theta_{\NY} = 1.0$.}
  \label{fig:phase_matrix}
\end{figure}

\paragraph{$\Sthree$:}
\begin{itemize}
  \item Phase boundary positions clearly differ among FULL, TT, REE
  \item FULL has the widest stable region
  \item TT and REE have narrower stable regions than FULL
\end{itemize}

\paragraph{$\Tthree$:}
\begin{itemize}
  \item FULL and REE have completely identical phase diagrams 
        (because $N_{\FULL} = N_{\REE}$)
  \item TT differs by a factor of 2 in coefficient, but phase boundary 
        positions are similar
  \item For $\theta_{\NY} \gtrsim 0.9$, Type I region appears in $\eta < 0$ 
        for all variants
\end{itemize}

\paragraph{$\Nilthree$:}
\begin{itemize}
  \item Moderate differences observed among the 3 variants
  \item Main band positions are nearly identical
  \item Shape and extent of lower stable region differ
\end{itemize}
