%==============================================================================
% Section 8: Conclusions and Outlook
%==============================================================================
\section{Conclusions and Outlook}
\label{sec:conclusions}

%------------------------------------------------------------------------------
\subsection{Summary of Results}
\label{sec:summary}

In this work, we studied Einstein--Cartan gravity supplemented with the Nieh--Yan 
term (EC+NY) in a Euclidean-signature minisuperspace framework and performed 
phase classification based on the shape of the effective potential $\Veff(r)$. 
Using three homogeneous spaces---$\Sthree$ (SU(2)), $\Tthree$ (flat), and 
$\Nilthree$ (Heisenberg)---as test beds for the spatial section, we systematically 
compared topology dependence under the same reduction procedure.

The main results are summarized below.

\subsubsection{Establishment of Type classification}

Based on the shape of the effective potential, we operationally defined the 
following three types:

\begin{itemize}
  \item \textbf{Type I} (metastable well with barrier): Local minimum exists 
        with barrier in the $r \to 0$ direction
  \item \textbf{Type II} (rolling): Local minimum exists but no barrier in 
        the $r \to 0$ direction
  \item \textbf{Type III} (unstable): No stable local minimum within the 
        allowed region
\end{itemize}

This classification is based on clear criteria---the extremum condition of 
$\Veff(r)$ and the gradient near the origin---and can be systematically 
applied through numerical scanning.

\subsubsection{Phase structure by topology}

Qualitatively different phase structures were observed for each topology:

\paragraph{$\Sthree \times \Sone$:}
Exhibits the most complex phase structure. For $\theta_{\NY} = 0$, only a 
band-like Type II region exists, but for $\theta_{\NY} > 0$, Type I regions 
appear and the stable region expands significantly with increasing $\theta_{\NY}$. 
Varying $\eta$ from positive to negative produces a re-transition of 
Type I $\to$ Type II $\to$ Type I.

\paragraph{$\Tthree \times \Sone$:}
Exhibits threshold-dependent phase structure. The entire parameter region 
is Type III (unstable) for $\theta_{\NY} < 0.87$, but Type I appears in the 
$\eta < 0$ region for $\theta_{\NY} \geq 0.87$. $\eta \to -\eta$ symmetry 
is broken for $\theta_{\NY} > 0$, with stable regions forming only on the 
$\eta < 0$ side.

\paragraph{$\Nilthree \times \Sone$:}
Exhibits intermediate complexity. The Type II region is limited to a narrow 
main band at $\eta \in (-0.3, 1.0)$, and for $\theta_{\NY} > 0$, a separate 
Type I region appears in the $\eta < 0$ region.

\subsubsection{Geometric origin of phase structure}

The following geometric factors were identified as origins of topology dependence:

\paragraph{Coupling with background curvature:}
Background Ricci scalar $R_{\LC}$ affects the effective potential through the 
$N_{\REE}$ component of the NY density. $\Sthree$ (positive curvature) and 
$\Nilthree$ (negative curvature) have continuous $\theta_{\NY}$ sensitivity 
due to curvature coupling. $\Tthree$ (zero curvature) lacks curvature coupling 
and shows threshold-type response (rapid activation at $\theta_{\NY} > 0.87$).

\paragraph{Symmetry breaking:}
$\eta \to -\eta$ symmetry is preserved for all topologies when $\theta_{\NY} = 0$ 
but broken for $\theta_{\NY} > 0$. For $\Sthree$ and $\Nilthree$, curvature 
coupling terms are the origin; for $\Tthree$, the projection of the NY term 
onto the isotropic volume mode is the origin of symmetry breaking.

\paragraph{Complexity of structure constants:}
The more complex the structure constants of the Lie algebra (more nonzero 
components, higher symmetry), the more complex the phase structure tends to be.

\subsubsection{Elucidation of FULL's role through TT/REE diagnosis}

By decomposing the NY density into $N_{\TT}$ (torsion-torsion) and $N_{\REE}$ 
(Riemann-torsion), we diagnosed the role of each contribution. FULL has the 
widest stable region in $\Sthree$ because $N_{\TT}$ and $N_{\REE}$ ``compete 
with opposite signs,'' maximizing the range of $\eta$ where the $r^2$ term 
coefficient $B < 0$. For $\Tthree$, the $N_{\REE}$ component degenerates, 
weakening the NY term effect.

%------------------------------------------------------------------------------
\subsection{Limitations of This Work}
\label{sec:limitations_detail}

We make explicit the limitations to keep in mind when interpreting the results 
of this work.

\subsubsection{Constraints of minisuperspace truncation}

This work is based on minisuperspace reduction assuming spatial homogeneity. 
This truncation discards the following physics:

\paragraph{Inhomogeneous modes:}
Spatially varying field configurations (gravitational waves, density 
perturbations, etc.) are not included. In actual cosmological scenarios, 
these perturbations may affect stability.

\paragraph{Local structure:}
The minisuperspace ansatz assumes global symmetry, so formation of local 
defects (cosmic strings, domain walls, etc.) cannot be described.

\paragraph{Restriction of dynamical degrees of freedom:}
Since only scale variable $r$ is taken as dynamical variable, anisotropic 
deformations (squashing, etc.) are not considered.

\subsubsection{Limitations of classical analysis}

The Type classification in this work is based on the classical effective 
potential. The following quantum effects are not considered:

\paragraph{Quantum tunneling:}
Tunneling probability through Type I barriers is evaluated by WKB approximation 
as $\Gamma \propto \exp(-B)$ (where $B$ is the bounce action), but this work 
only shows barrier height $\Delta V$ values.

\paragraph{One-loop corrections:}
Corrections to the effective potential from quantum fluctuations of fields 
are not included. In particular, contributions from high-energy modes may 
become important near $r \to 0$.

\paragraph{Renormalization group running:}
Scale dependence of the coupling constant $\theta_{\NY}$ is not considered.

\subsubsection{Constraints of torsion ansatz}

The torsion ansatz (T1 + T2) adopted in this work includes only 2 components 
(axial and vector trace) of the 4-dimensional torsion irreducible decomposition. 
The complete irreducible decomposition includes 3 components (tensor, vector, 
axial), and more general ansatz would show additional parameters and phase 
structure.

Also, torsion parameters $\eta$, $V$ are assumed to be spatially uniform, 
so non-uniform torsion configurations cannot be treated in this framework.

\subsubsection{Connection to Lorentzian signature}

This work performs calculations in Euclidean signature $(+,+,+,+)$. 
Connection to real-time cosmology requires Wick rotation $\tau \to it$, 
with the following points requiring attention:

\begin{itemize}
  \item Minima of effective potential correspond to ``classical turning points'' 
        in Lorentzian signature
  \item Type I barriers define ``tunneling regions'' in real time
  \item Type II rolling corresponds to ``classical evolution'' in real time
\end{itemize}

Physical interpretation of these correspondences is outside the scope of 
this paper.

%------------------------------------------------------------------------------
\subsection{Future Prospects}
\label{sec:outlook}

Based on the diagnostic framework established in this work, the following 
developments are conceivable.

\subsubsection{Generalization of torsion ansatz}

This work adopted T1 (axial) and T2 (vector trace) components, but 
generalization to include the third component (tensor) of the irreducible 
decomposition is a natural next step.

Also, allowing $r$-dependence of torsion parameters in a dynamical ansatz 
(e.g., $\eta(r)$, $V(r)$) may give rise to richer phase structure. As seen 
in Sec.~\ref{sec:mechanisms}, minimum existence is determined by competition 
between coefficients $B$ and $C$, and when these depend on $r$, new 
stabilization mechanisms may arise.

\subsubsection{Extension to other topologies}

As mentioned in Sec.~\ref{sec:diagnostic_tool}, this framework can be extended 
to other 3-dimensional Lie groups admitting left-invariant coframes. 
A particularly interesting candidate is $\mathrm{Sol}$ geometry (Bianchi Type VI$_0$). 
$\mathrm{Sol}$ is anisotropic like $\Nilthree$ but has different structure 
constant forms; what differences appear in phase structure is a future task.

Also, extension to non-parallelizable spaces ($S^2 \times \mathbb{R}$, etc.) 
involves technical challenges but is an important direction for exploring 
the relationship between topological constraints and stability.

\subsubsection{Semiclassical tunneling rate calculation}

Barrier height $\Delta V$ of Type I qualitatively indicates the ``robustness'' 
of metastable states, but quantitative evaluation of tunneling rates requires 
WKB approximation or instanton calculations.

Specifically, one constructs bounce solutions (classical solutions in the 
inverted potential $-\Veff(r)$) and evaluates decay rate 
$\Gamma \propto \exp(-B)$ from their action $B$. The analytical expressions 
for $\Veff(r)$ obtained in this work can be directly used as input for 
this calculation.

\subsubsection{Analysis of dynamical evolution}

This work focused on classification of static effective potentials, but 
by analytic continuation to Lorentzian signature, dynamics including time 
evolution can be discussed.

By deriving Friedmann-type constraint equations and tracking classical 
evolution for Type II (rolling) and post-tunneling evolution for Type I, 
the viability as cosmological scenarios can be evaluated.

\subsubsection{Coupling with matter fields}

This work treated pure EC+NY theory, but coupling with spinor and scalar 
fields would clarify the physical origin and effects of torsion.

In particular, in EC theory, spinor fields are natural sources of torsion, 
and dynamical torsion generation through fermion condensation may provide 
physical basis for the torsion ansatz assumed in this work.

\subsubsection{Connection with self-duality}

In 4-dimensional Euclidean geometry, the self-dual (or anti-self-dual) 
condition $R^{ab} = \pm {}^*R^{ab}$ provides strong constraints. 
Diagnosing whether stable solution candidates obtained in this work satisfy 
self-duality enables bridging to instanton interpretation.

When self-dual solutions exist, their action is directly related to 
topological invariants (Euler number, Hirzebruch signature), providing 
topological constraints on transition probabilities between sectors.

%------------------------------------------------------------------------------
\subsection{Concluding Remarks}
\label{sec:concluding}

This work has shown that in minisuperspace reduction of EC+NY theory, the 
phase structure of effective potentials depends strongly on topology. 
Through the three contrasting test beds of $\Sthree$, $\Tthree$, and $\Nilthree$, 
we clarified that background curvature and structure constants are the 
geometric factors determining phase structure.

The Type I/II/III classification presented here, along with the phase diagrams, 
representative points, and critical conditions based on it, function as 
reproducible diagnostic tools in EC+NY minisuperspace analysis. This framework 
provides a foundation for systematically organizing the correspondence 
between geometric input and phase structure in effective potential analysis 
of gravitational theories including torsion.
