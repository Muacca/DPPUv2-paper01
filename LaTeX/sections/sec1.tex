%==============================================================================
% Section 1: Introduction
%==============================================================================
\section{Introduction}
\label{sec:introduction}

%------------------------------------------------------------------------------
\subsection{Background: Correspondence Between Geometric Input and Phase Structure of Effective Potentials}
\label{sec:background}

In theories where the coframe and connection are treated as independent variables, 
curvature and torsion are introduced on equal footing as geometric degrees of freedom. 
In particular, Einstein--Cartan (EC) theory and its extensions allow torsion and 
topological densities to appear in the action, potentially making the structure 
of the reduced effective theory sensitive to geometric input.

Meanwhile, minisuperspace reduction based on symmetry assumptions provides a 
powerful procedure for mapping infinite-dimensional field theories to 
finite-dimensional systems. However, even within the same reduction framework, 
how differences in spatial topology or structure constants of homogeneous spaces 
affect the shape of the effective potential $\Veff(r)$---including the presence 
or absence of minima and barriers, and transitions to rolling behavior---has not 
been systematically organized.

For this reason, it is important to classify and visualize, in a reproducible manner, 
the correspondence between geometric input and the phase structure of the reduced 
effective potential. This work organizes this correspondence using representative 
homogeneous spaces as test beds, presenting the results as phase diagrams and 
representative points.

%------------------------------------------------------------------------------
\subsection{Our Approach: Treating EC+NY as a ``Phase Classification Problem for Effective Potentials''}
\label{sec:approach}

In this work, we take EC+NY (Einstein--Cartan gravity with the Nieh--Yan term) as 
our object of study and classify the phases of the effective potential in a 
Euclidean-signature minisuperspace from the perspective of topology dependence. 
In EC theory, torsion is naturally introduced as geometry, and while the NY density 
(a 4-form) can be written as an exact derivative, it is geometrically defined 
through the coframe and torsion. Therefore, by comparing how the effective potential 
changes when varying the spatial topology under the same minisuperspace ansatz, 
we diagnose the phase structure provided by EC+NY from the viewpoint of geometric input.

Although EC+NY originates as a gravitational theory, the focus of this paper is 
on systematically organizing the influence of geometric input (topology, structure 
constants, torsion, and NY contributions) on the phase structure after reduction.

%------------------------------------------------------------------------------
\subsection{Scope: Three Homogeneous Spaces as Test Beds}
\label{sec:scope}

To enable comparison from the perspective of topology dependence, we adopt spatial 
sections that possess homogeneity and can be uniformly described using left-invariant 
coframes. Specifically, as three representative examples with contrasting properties 
in structure constants and curvature, we adopt:
\begin{enumerate}[label=(\roman*)]
  \item $\Sthree$ (SU(2); isotropic positive curvature),
  \item $\Tthree$ (flat),
  \item $\Nilthree$ (compact quotient of the Heisenberg group; anisotropic geometry).
\end{enumerate}
These serve as a minimal set for comparing how ``geometric data (structure constants, 
curvature characteristics)'' affect the phase structure of the effective potential 
under the same reduction procedure.

%------------------------------------------------------------------------------
\subsection{Deliverables: Type Classification, Critical Conditions, Phase Diagrams, and Representative Points}
\label{sec:deliverables}

The central output of this work is an operational classification based on the 
shape of $\Veff(r)$. Specifically, we introduce three types:
\begin{itemize}
  \item[(I)] Metastable well with barrier,
  \item[(II)] Barrier-free rolling,
  \item[(III)] Unstable/boundary-attached,
\end{itemize}
and visualize how these are arranged in parameter space for each topology as 
phase diagrams. Furthermore, we identify critical conditions corresponding to 
well formation/disappearance and barrier collapse through numerical scanning, 
and organize their geometric dependence.

For the NY term, we take the complete form (FULL) as the primary object, while 
also using TT and REE as diagnostic comparisons to disentangle the contributions 
(definitions in Sec.~\ref{sec:setup}). Note that TT/REE are not proposed as 
independent fundamental theories but serve as auxiliary comparisons for 
understanding the effects of FULL.

Through this approach, we provide a reproducible diagnostic pipeline that returns 
the phase (Type I/II/III) and critical boundaries of $\Veff(r)$ for given geometric 
input (topology, parameters). The resulting phase diagrams, representative points, 
and critical conditions can be used as foundational data for explicitly incorporating 
topology dependence in EC+NY minisuperspace analysis.

%------------------------------------------------------------------------------
\subsection{Paper Organization}
\label{sec:organization}

The organization of this paper is as follows. Section~\ref{sec:setup} summarizes 
the formulation of EC+NY and minisuperspace reduction, as well as the definitions 
and numerical criteria for Type I/II/III. Section~\ref{sec:reductions} presents 
the reduction results for $\Sthree$, $\Tthree$, and $\Nilthree$, showing the form 
of the effective potential. Section~\ref{sec:numerical} presents numerical scanning 
results as phase diagrams. Section~\ref{sec:mechanisms} interprets the phase 
diagram features based on the analytical structure of the effective potential. 
Section~\ref{sec:representative} summarizes representative points and stability 
metrics (minimum position, barrier height, etc.). Section~\ref{sec:interpretation} 
discusses geometric interpretation and the utility of this framework. 
Section~\ref{sec:conclusions} presents conclusions and future prospects.

%------------------------------------------------------------------------------
\subsection{Scope and Limitations}
\label{sec:limitations}

This work focuses on phase classification of effective potentials based on 
minisuperspace reduction. The following items are outside the scope of this paper 
and are discussed as future research directions in Sec.~\ref{sec:conclusions}.

\paragraph{Dynamical evolution:}
We do not treat time evolution including Friedmann-type constraint equations, 
nor quantitative calculations of tunneling rates via WKB approximation. 
The distinction between Type I/II is an operational classification based on 
the shape of $\Veff(r)$ and does not provide actual transition probabilities.

\paragraph{Quantum corrections:}
We do not consider quantum corrections at one-loop or higher, nor renormalization 
group running of coupling constants. The results of this work are based on the 
classical (tree-level) effective potential.

\paragraph{Matter field coupling:}
We do not include interactions with matter fields such as spinor or scalar fields. 
We focus on the geometric structure of pure EC+NY theory.

\paragraph{Inhomogeneous perturbations:}
We do not analyze inhomogeneous fluctuations beyond the minisuperspace ansatz 
(spatial homogeneity). Consequently, our results are limited to the homogeneous sector.

\paragraph{Analytic continuation to Lorentzian signature:}
This work performs calculations in Euclidean signature $(+,+,+,+)$. 
Wick rotation to Lorentzian signature $(-,+,+,+)$ and interpretation in 
real-time cosmology are not discussed in detail in this paper.
