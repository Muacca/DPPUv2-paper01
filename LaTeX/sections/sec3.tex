%==============================================================================
% Section 3: Topology-Specific Reductions
%==============================================================================
\section{Topology-Specific Reductions}
\label{sec:reductions}

This section applies the minisuperspace reduction defined in Sec.~\ref{sec:setup} 
to the three test beds ($\Sthree$, $\Tthree$, $\Nilthree$) and derives the 
explicit form of the effective potential $\Veff(r)$ for each topology.

%------------------------------------------------------------------------------
\subsection{$\Sthree \times \Sone$: Reduced Action and Effective Potential}
\label{sec:S3_reduction}

\subsubsection{Geometric setup}

$\Sthree$ is realized as the SU(2) group manifold, with left-invariant coframes 
$\{\sigma^i\}$ given by the Maurer-Cartan forms of SU(2). The structure constants are
\begin{equation}
  C^{i}{}_{jk} = \frac{4}{r} \varepsilon_{ijk},
\end{equation}
and the background Ricci scalar (without torsion) is
\begin{equation}
  R_{\LC} = \frac{24}{r^2} > 0
\end{equation}
(positive curvature). The volume element is
\begin{equation}
  \mathrm{Vol}(\Sthree \times \Sone) = 2\pi^2 L r^3.
\end{equation}

\subsubsection{Scalar quantities}

In MX mode ($\eta \neq 0$, $V \neq 0$), the scalar quantities computed by 
DPPUv2 Engine Core v3 are as follows:

\paragraph{Ricci scalar:}
\begin{equation}
  R = \frac{2(-V^2 r^2 - 9\eta^2 - 72\eta - 108)}{3r^2}
\end{equation}

\paragraph{Torsion scalar:}
\begin{equation}
  T_{abc}T^{abc} = \frac{2V^2}{3} + \frac{24\eta^2}{r^2}
\end{equation}

\paragraph{NY densities (each variant):}
\begin{equation}
  N_{\TT} = -\frac{4V\eta}{r}, \quad 
  N_{\REE} = -\frac{2V(\eta + 4)}{r}, \quad 
  N_{\FULL} = \frac{2V(4 - \eta)}{r}
\end{equation}

\subsubsection{Effective potential}

Integrating the Lagrangian $\mathcal{L} = R/(2\kappa^2) + \theta_{\NY} N$ 
over the volume and extracting $\Veff(r) = -S_{\mathrm{eff}}$, we obtain 
for each NY variant:

\paragraph{FULL ($N = N_{\FULL}$):}
\begin{equation}
  \Veff^{(\Sthree,\FULL)}(r) = \frac{2\pi^2 L}{3\kappa^2} r 
  \left[ V^2 r^2 + 6V\kappa^2\theta_{\NY}(\eta - 4) r + 9\eta^2 + 72\eta + 108 \right]
  \label{eq:Veff_S3_FULL}
\end{equation}

\paragraph{TT ($N = N_{\TT}$):}
\begin{equation}
  \Veff^{(\Sthree,\TT)}(r) = \frac{2\pi^2 L}{3\kappa^2} r 
  \left[ V^2 r^2 + 12V\eta\kappa^2\theta_{\NY} r + 9\eta^2 + 72\eta + 108 \right]
\end{equation}

\paragraph{REE ($N = N_{\REE}$):}
\begin{equation}
  \Veff^{(\Sthree,\REE)}(r) = \frac{2\pi^2 L}{3\kappa^2} r 
  \left[ V^2 r^2 + 6V\kappa^2\theta_{\NY}(\eta + 4) r + 9\eta^2 + 72\eta + 108 \right]
\end{equation}

These potential shapes are shown in Figure~\ref{fig:variant_S3}.

\begin{figure}[htbp]
  \centering
  \includegraphics[width=1.0\textwidth]{figures/fig03_variant_comparison_S3.png}
  \caption{Nieh--Yan variant comparison for $\Sthree \times \Sone$.}
  \label{fig:variant_S3}
\end{figure}

\subsubsection{Structural analysis}

The effective potential for $\Sthree$ has a cubic polynomial structure in $r$ 
(of the form $r^3 + r^2 + r$):
\begin{equation}
  \Veff(r) \propto r \cdot \left[ A r^2 + B r + C \right],
\end{equation}
where:
\begin{itemize}
  \item $A = V^2 > 0$: Governs divergence as $r \to \infty$ (always positive)
  \item $B$: Proportional to NY coupling $\theta_{\NY}$, depends on $\eta$ 
        (sign can vary)
  \item $C = 9\eta^2 + 72\eta + 108 = 9(\eta + 4)^2 - 36$: Takes minimum 
        value $-36$ at $\eta = -4$
\end{itemize}

From this structure, a local minimum can appear when $B < 0$ and $|B|$ is 
sufficiently large. In particular, for the FULL variant, $B \propto (\eta - 4)$, 
so $B < 0$ for $\eta < 4$, meaning local minima can appear even in the 
$\eta > 0$ region.

The contributions of $r^3$, $r^2$, and $r$ terms to the potential are shown 
in Figure~\ref{fig:decomposition_S3}.

\begin{figure}[htbp]
  \centering
  \includegraphics[width=0.9\textwidth]{figures/fig04_decomposition_S3.png}
  \caption{Potential decomposition for $\Sthree \times \Sone$.}
  \label{fig:decomposition_S3}
\end{figure}

%------------------------------------------------------------------------------
\subsection{$\Tthree \times \Sone$: Reduced Action and Effective Potential}
\label{sec:T3_reduction}

\subsubsection{Geometric setup}

$\Tthree$ is an Abelian group (discrete quotient of $\mathbb{R}^3$), with all 
structure constants vanishing:
\begin{equation}
  C^i{}_{jk} = 0.
\end{equation}
The background Ricci scalar is
\begin{equation}
  R_{\LC} = 0
\end{equation}
(flat). The volume element is
\begin{equation}
  \mathrm{Vol}(\Tthree \times \Sone) = (2\pi)^4 L R_1 R_2 R_3,
\end{equation}
where $R_1, R_2, R_3$ are the circumferences in each direction of $\Tthree$. 
In this paper, for comparison with $\Sthree$ and $\Nilthree$, we assume 
isotropic expansion $R_1 = R_2 = R_3 = r$. This allows description with a 
single scale parameter $r$, as with the other topologies.

\subsubsection{Scalar quantities}

In MX mode, the scalar quantities are:

\paragraph{Ricci scalar:}
\begin{equation}
  R = -\frac{2V^2}{3} - \frac{6\eta^2}{r^2}
\end{equation}

\paragraph{Torsion scalar:}
\begin{equation}
  T_{abc}T^{abc} = \frac{2V^2}{3} + \frac{24\eta^2}{r^2}
\end{equation}

\paragraph{NY densities (each variant):}
\begin{equation}
  N_{\TT} = -\frac{4V\eta}{r}, \quad 
  N_{\REE} = -\frac{2V\eta}{r}, \quad 
  N_{\FULL} = -\frac{2V\eta}{r}
\end{equation}

Notably, for $\Tthree$, $N_{\FULL} = N_{\REE}$. This is due to the zero 
background curvature.

\subsubsection{Effective potential}

With the isotropic setting $R_1 = R_2 = R_3 = r$, for each NY variant:

\paragraph{FULL (= REE):}
\begin{equation}
  \Veff^{(\Tthree,\FULL)}(r) = \frac{16\pi^4 L}{3\kappa^2} 
  \left[ V^2 r^3 + 6V\eta\kappa^2\theta_{\NY} r^2 + 9\eta^2 r \right]
  \label{eq:Veff_T3_FULL}
\end{equation}

\paragraph{TT:}
\begin{equation}
  \Veff^{(\Tthree,\TT)}(r) = \frac{16\pi^4 L}{3\kappa^2} 
  \left[ V^2 r^3 + 12V\eta\kappa^2\theta_{\NY} r^2 + 9\eta^2 r \right]
\end{equation}

These have the same $r^3 + r^2 + r$ structure as $\Sthree$.

The potential shapes are shown in Figure~\ref{fig:variant_T3}.

\begin{figure}[htbp]
  \centering
  \includegraphics[width=1.0\textwidth]{figures/fig05_variant_comparison_T3.png}
  \caption{Nieh--Yan variant comparison for $\Tthree \times \Sone$.}
  \label{fig:variant_T3}
\end{figure}

\subsubsection{Structural analysis}

With the isotropic setting, the effective potential for $\Tthree$ has the 
same structure as $\Sthree$ and $\Nilthree$:
\begin{equation}
  \Veff(r) \propto r \cdot \left[ V^2 r^2 + B r + C \right],
\end{equation}
where $B = 6V\eta\kappa^2\theta_{\NY}$ and $C = 9\eta^2$.

From this form:
\begin{itemize}
  \item $r \to 0$: $\Veff \to 0^+$ (always positive for $C > 0$ when $\eta \neq 0$)
  \item $r \to \infty$: $\Veff \to +\infty$ due to the $r^3$ term
  \item Minimum existence condition: $3V^2 r^2 + 2Br + C = 0$ has positive 
        real solutions
\end{itemize}

A characteristic of $\Tthree$ is that the coefficient $C = 9\eta^2$ of $r$ 
is always non-negative (zero only when $\eta = 0$). Therefore:
\begin{itemize}
  \item For $\theta_{\NY} = 0$, $B = 0$, and 
        $\dd\Veff/\dd r \propto 3V^2 r^2 + 9\eta^2 > 0$ (for $r > 0$), 
        so it is always monotonically increasing (Type III)
  \item For $\theta_{\NY} > 0$ and $\eta < 0$, $B < 0$, and minima can form
\end{itemize}

Under $\eta \to -\eta$ transformation, $B \to -B$, so for $\theta_{\NY} \neq 0$, 
symmetry is broken. For $\theta_{\NY} > 0$, stable regions appear only on 
the $\eta < 0$ side.

The contributions of $r^3$, $r^2$, and $r$ terms are shown in 
Figure~\ref{fig:decomposition_T3}.

\begin{figure}[htbp]
  \centering
  \includegraphics[width=0.9\textwidth]{figures/fig06_decomposition_T3.png}
  \caption{Potential decomposition for $\Tthree \times \Sone$.}
  \label{fig:decomposition_T3}
\end{figure}

%------------------------------------------------------------------------------
\subsection{$\Nilthree \times \Sone$: Reduced Action and Effective Potential}
\label{sec:Nil3_reduction}

\subsubsection{Geometric setup}

$\Nilthree$ is a compact quotient of the Heisenberg group (Bianchi Type II), 
with structure constants
\begin{equation}
  C^{2}{}_{01} = -\frac{1}{r}, \quad C^{2}{}_{10} = +\frac{1}{r}, \quad 
  \text{others } 0.
\end{equation}
An important point is that $\Nilthree$ is \textbf{not bi-invariant} (the 
Heisenberg group is nilpotent, not semisimple). Therefore, the generalized 
Koszul formula (see Appendix~\ref{app:theory}) must be used for connection 
calculations.

The background Ricci scalar is
\begin{equation}
  R_{\LC} = -\frac{1}{2r^2} < 0
\end{equation}
(negative curvature). The volume element is
\begin{equation}
  \mathrm{Vol}(\Nilthree \times \Sone) = (2\pi)^4 L r^3.
\end{equation}

\subsubsection{Scalar quantities}

In MX mode, the scalar quantities are:

\paragraph{Ricci scalar:}
\begin{equation}
  R = \frac{-4V^2 r^2 - 36\eta^2 + 24\eta + 9}{6r^2}
\end{equation}

\paragraph{Torsion scalar:}
\begin{equation}
  T_{abc}T^{abc} = \frac{2V^2}{3} + \frac{24\eta^2}{r^2}
\end{equation}

\paragraph{NY densities (each variant):}
\begin{equation}
  N_{\TT} = -\frac{4V\eta}{r}, \quad 
  N_{\REE} = \frac{2V(1 - 3\eta)}{3r}, \quad 
  N_{\FULL} = -\frac{2V(3\eta + 1)}{3r}
\end{equation}

\subsubsection{Effective potential}

For each NY variant:

\paragraph{FULL:}
\begin{equation}
  \Veff^{(\Nilthree,\FULL)}(r) = \frac{4\pi^4 L}{3\kappa^2} r 
  \left[ 4V^2 r^2 + 8V\kappa^2\theta_{\NY}(3\eta + 1) r 
         + 36\eta^2 - 24\eta - 9 \right]
  \label{eq:Veff_Nil3_FULL}
\end{equation}

\paragraph{TT:}
\begin{equation}
  \Veff^{(\Nilthree,\TT)}(r) = \frac{4\pi^4 L}{3\kappa^2} r 
  \left[ 4V^2 r^2 + 48V\eta\kappa^2\theta_{\NY} r + 36\eta^2 - 24\eta - 9 \right]
\end{equation}

\paragraph{REE:}
\begin{equation}
  \Veff^{(\Nilthree,\REE)}(r) = \frac{4\pi^4 L}{3\kappa^2} r 
  \left[ 4V^2 r^2 + 8V\kappa^2\theta_{\NY}(3\eta - 1) r 
         + 36\eta^2 - 24\eta - 9 \right]
\end{equation}

The potential shapes are shown in Figure~\ref{fig:variant_Nil3}.

\begin{figure}[htbp]
  \centering
  \includegraphics[width=1.0\textwidth]{figures/fig07_variant_comparison_Nil3.png}
  \caption{Nieh--Yan variant comparison for $\Nilthree \times \Sone$.}
  \label{fig:variant_Nil3}
\end{figure}

\subsubsection{Structural analysis}

The effective potential for $\Nilthree$ also has the $r^3 + r^2 + r$ structure 
like $\Sthree$, but the sign of coefficients brings differences:
\begin{equation}
  C_{\Nilthree} = 36\eta^2 - 24\eta - 9 = 36\left(\eta - \frac{1}{3}\right)^2 - 13.
\end{equation}

This quadratic function takes minimum value $-13$ at $\eta = 1/3$, and becomes 
negative in the range $\eta \in (-0.27, 0.94)$.

When the coefficient $C$ of $r$ is negative, $\Veff(r) \to 0^-$ as $r \to 0$, 
meaning the potential takes negative values near the origin. This contrasts 
with $\Sthree$ (where $C > 0$ over a wide region), suggesting that for 
$\Nilthree$ with small $\theta_{\NY}$ values, the stable region is limited 
to a narrow band.

The contributions of $r^3$, $r^2$, and $r$ terms are shown in 
Figure~\ref{fig:decomposition_Nil3}.

\begin{figure}[htbp]
  \centering
  \includegraphics[width=0.9\textwidth]{figures/fig08_decomposition_Nil3.png}
  \caption{Potential decomposition for $\Nilthree \times \Sone$.}
  \label{fig:decomposition_Nil3}
\end{figure}

%------------------------------------------------------------------------------
\subsection{Structural Comparison: Sources of Topology Dependence}
\label{sec:structure_comparison}

We compare the structure of effective potentials across the three topologies 
and organize the sources of topology dependence.

\subsubsection{General form of effective potential}

For all topologies, the effective potential in MX mode is expressed in the 
general form:
\begin{equation}
  \Veff(r) = \mathcal{N} \cdot r^{\alpha} \cdot P(r),
\end{equation}
where $\mathcal{N}$ is a normalization factor, $\alpha$ is a topology-dependent 
exponent, and $P(r)$ is a polynomial in $r$.

\begin{table}[H]
\centering
\caption{Effective potential structure comparison.}
\label{tab:structure_comparison}
\begin{tabular}{@{}llll@{}}
\toprule
Topology & Structure & $r \to 0$ & $r \to \infty$ \\
\midrule
$\Sthree$ & $r \cdot [Ar^2 + Br + C]$ & $\to 0$ & $\to +\infty$ \\
$\Tthree$ & $r \cdot [Ar^2 + Br + C]$ & $\to 0$ & $\to +\infty$ \\
$\Nilthree$ & $r \cdot [Ar^2 + Br + C]$ & $\to 0$ & $\to +\infty$ \\
\bottomrule
\end{tabular}
\end{table}

All of $\Sthree$, $\Tthree$, and $\Nilthree$ share the $r \cdot [Ar^2 + Br + C]$ 
structure.

\subsubsection{Comparison of NY densities}

\begin{table}[H]
\centering
\caption{Comparison of NY densities ($N_{\FULL}$) and background curvature 
dependence.}
\label{tab:NY_comparison}
\begin{tabular}{@{}lll@{}}
\toprule
Topology & $N_{\FULL}$ & Background curvature dependence \\
\midrule
$\Sthree$ & $\displaystyle\frac{2V(4 - \eta)}{r}$ & 
            $(4 - \eta)$: contribution from positive curvature \\
$\Tthree$ & $\displaystyle-\frac{2V\eta}{r}$ & 
            $\eta$ only: no curvature contribution \\
$\Nilthree$ & $\displaystyle-\frac{2V(3\eta + 1)}{3r}$ & 
              $(3\eta + 1)$: contribution from negative curvature \\
\bottomrule
\end{tabular}
\end{table}

The factor $(4 - \eta)$ appearing in $N_{\FULL}$ for $\Sthree$ originates from 
the $(\eta + 4)$ term in $N_{\REE}$. This $+4$ reflects coupling with the 
positive background curvature ($R_{\LC} = 24/r^2$).

In contrast, for $\Tthree$, the background curvature is zero, so there is no 
curvature contribution to $N_{\FULL}$.

\subsubsection{Geometric factors affecting stability}

The existence and position of the effective potential minimum are primarily 
determined by the following factors:

\begin{enumerate}
  \item \textbf{Coefficient of $r^3$ term} ($\propto V^2$): Always positive. 
        Governs growth as $r \to \infty$.
  
  \item \textbf{Coefficient of $r^2$ term} ($\propto V\theta_{\NY} \times f(\eta)$): 
        Function of NY coupling and $\eta$. When negative, promotes minimum formation.
  
  \item \textbf{Coefficient of $r$ term}: Topology-specific geometric contribution.
        \begin{itemize}
          \item $\Sthree$: $9(\eta + 4)^2 - 36$
          \item $\Tthree$: $9\eta^2$
          \item $\Nilthree$: $36(\eta - 1/3)^2 - 13$
        \end{itemize}
\end{enumerate}

The competition among these factors produces different phase structures for 
each topology. Detailed analysis of phase boundaries is presented in 
Sec.~\ref{sec:numerical}.
