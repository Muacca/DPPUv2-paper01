%==============================================================================
% Section 7: Geometric Interpretation and Utility as a Tool
%==============================================================================
\section{Geometric Interpretation and Utility as a Tool}
\label{sec:interpretation}

This section interprets the numerical results obtained in Secs.~\ref{sec:numerical}--\ref{sec:representative} 
from a geometric perspective and discusses the applicability and limitations 
of the diagnostic framework provided by this work.

%------------------------------------------------------------------------------
\subsection{Geometric Factors of Topology Dependence}
\label{sec:geometric_factors}

\subsubsection{Correspondence between structure constants and phase structure}

We organize the correspondence between the characteristics of structure 
constants $C^{i}{}_{jk}$ for the three test beds and the observed phase structure.

\begin{table}[H]
\centering
\caption{Correspondence between structure constants and phase structure.}
\label{tab:structure_phase}
\begin{tabularx}{\textwidth}{@{}l>{\raggedright\arraybackslash}X l>{\raggedright\arraybackslash}X@{}}
\toprule
Topology & Structure constant characteristics & Lie algebra & Phase structure characteristics \\
\midrule
$\Sthree$ & $C^{i}{}_{jk} = (4/r)\varepsilon_{ijk}$ (totally antisym.) & 
            $\mathfrak{su}(2)$ (simple) & Complex, high $\theta_{\NY}$ sensitivity \\
$\Tthree$ & $C^{i}{}_{jk} = 0$ (all zero) & Abelian & 
            Threshold-dependent, activates at $\theta_{\NY} > 0.87$ \\
$\Nilthree$ & $C^{2}{}_{01} = -1/r$, $C^{2}{}_{10} = +1/r$, others 0 & 
              Heisenberg (nilpotent) & Intermediate, band + separate region \\
\bottomrule
\end{tabularx}
\end{table}

From this correspondence, it is suggested that the ``complexity'' of structure 
constants (number of nonzero components and symmetry) correlates with the 
complexity of phase structure.

\subsubsection{Role of background curvature}

The background Ricci scalar $R_{\LC}$ from the Levi--Civita connection affects 
the effective potential through the $N_{\REE}$ component of the NY density.

\paragraph{Positive curvature ($\Sthree$): $R_{\LC} = 24/r^2 > 0$}

The factor $(\eta + 4)$ appearing in $N_{\REE} = -2V(\eta + 4)/r$ reflects 
coupling with positive curvature. This factor breaks $\eta \to -\eta$ symmetry 
and becomes a driving force for forming a wide stable region in the $\eta < 0$ 
region.

As a physical interpretation, in the positive curvature background, the 
coupling of the axial component ($\eta$) of torsion with curvature produces 
an ``effective negative contribution,'' promoting potential well formation.

\paragraph{Zero curvature ($\Tthree$): $R_{\LC} = 0$}

Since the background curvature is zero, the curvature coupling term in 
$N_{\REE}$ vanishes, and $N_{\FULL} = N_{\REE} = -2V\eta/r$. However, with 
the isotropic setting $R_1 = R_2 = R_3 = r$, this NY term contributes as the 
$r^2$ coefficient in the effective potential, giving rise to $\theta_{\NY}$ 
sensitivity.

A characteristic of $\Tthree$ is that the coefficient $C = 9\eta^2$ of $r$ 
is always non-negative. Therefore, Type I appearance requires $\theta_{\NY}$ 
to exceed a threshold, with ``activation conditions'' being stricter compared 
to $\Sthree$ or $\Nilthree$.

$\eta \to -\eta$ symmetry is preserved for $\theta_{\NY} = 0$ but broken for 
$\theta_{\NY} > 0$ due to sign reversal of $B = 6V\eta\kappa^2\theta_{\NY}$. 
This breaking originates not from curvature coupling but from the projection 
of the NY term onto the isotropic volume mode.

\paragraph{Negative curvature ($\Nilthree$): $R_{\LC} = -1/(2r^2) < 0$}

The factor $(1 - 3\eta)$ appearing in $N_{\REE} = 2V(1 - 3\eta)/(3r)$ 
reflects coupling with negative curvature. Compared to the $(\eta + 4)$ of 
$\Sthree$, the shift is smaller ($+1/3$ vs $+4$), and the symmetry-breaking 
effect is moderate.

The main band being limited to $\eta \in (-0.27, 0.93)$ originates from 
$\Nilthree$'s structure constants being nonzero ``in only one direction.'' 
This anisotropy makes the quadratic coefficient of $C(\eta)$ larger than 
$\Sthree$'s ($36$ vs $9$), consequently narrowing the range of $\eta$ where 
$C < 0$.

\subsubsection{Isotropy and anisotropy}

$\Sthree$ (SU(2)) is an isotropic homogeneous space where any two directions 
can be mapped to each other by group action. On the other hand, $\Nilthree$ 
(Heisenberg) is anisotropic, with the structure $[E_0, E_1] = (1/R)E_2$ 
giving special treatment to the ``$E_2$ direction.''

This anisotropy manifests in the effective potential as follows:
\begin{itemize}
  \item $\Sthree$: $C(\eta) = 9(\eta + 4)^2 - 36$ (large $\eta$ shift)
  \item $\Nilthree$: $C(\eta) = 36(\eta - 1/3)^2 - 13$ (small $\eta$ shift, 
        large coefficient)
\end{itemize}

For anisotropic topologies, the ``allowed range'' of torsion parameters 
tends to be narrower than in isotropic cases.

%------------------------------------------------------------------------------
\subsection{Role of FULL Revealed by Diagnostic Comparison (TT/REE)}
\label{sec:FULL_role}

\subsubsection{Physical meaning of TT and REE}

In the decomposition $N = N_{\TT} - N_{\REE}$ of the NY density, each component 
carries different geometric information:

\paragraph{$N_{\TT}$ (torsion-torsion term):}
\begin{equation}
  N_{\TT} = \frac{1}{4}\varepsilon^{abcd} T^{e}{}_{ab} T_{ecd}
\end{equation}

This represents ``self-interaction'' of torsion and does not depend on 
background curvature. It takes the form $N_{\TT} \propto V\eta/r$ and is 
an odd function of $\eta$.

\paragraph{$N_{\REE}$ (Riemann-torsion term):}
\begin{equation}
  N_{\REE} = \frac{1}{4}\varepsilon^{abcd} R_{abcd}
\end{equation}

This is the contribution of curvature from the EC connection (including torsion) 
and contains coupling with background curvature. It has different $\eta$ 
dependence for each topology:
\begin{itemize}
  \item $\Sthree$: $N_{\REE} \propto V(\eta + 4)/r$
  \item $\Tthree$: $N_{\REE} \propto V\eta/r$
  \item $\Nilthree$: $N_{\REE} \propto V(1 - 3\eta)/r$
\end{itemize}

\subsubsection{Reason why FULL is optimal}

From the comparison in Figure~\ref{fig:phase_matrix}, FULL was observed to 
have the widest stable region for $\Sthree$. This is due to the following 
mechanism:

The $r^2$ term coefficient in FULL, $B \propto (\eta - 4)$, arises as the 
``difference'' between TT's $B \propto \eta$ and REE's $B \propto (\eta + 4)$:
\begin{equation}
  (\eta - 4) = \eta - (\eta + 4) + 2\eta = 2\eta - 4.
\end{equation}

This combination maximizes the range of $\eta$ where $B < 0$ (the entire 
region $\eta < 4$).

Physically, $N_{\TT}$ and $N_{\REE}$ ``compete with opposite signs,'' 
realizing a wide stable region that cannot be achieved individually. 
This suggests that the exact derivative structure $N = \dd(e^a \wedgep T_a)$ 
of the NY term carries geometric information different from simple torsion 
squared terms or curvature terms.

\subsubsection{Degeneracy and threshold effect in $\Tthree$}

For $\Tthree$, $N_{\FULL} = N_{\REE} = -2V\eta/r$, meaning FULL and REE 
coincide completely. This is because background curvature is zero, causing 
the curvature-dependent part of $N_{\TT}$ and $N_{\REE}$ to vanish, with 
their difference becoming a constant multiple.

However, with the isotropic setting, the NY term contributes to the effective 
potential as the $r^2$ coefficient $B$, so for all 3 variants, Type I appears 
when $\theta_{\NY}$ exceeds the threshold $> 0.87$. TT has twice the coefficient 
of FULL/REE, which may slightly shift the threshold, but the qualitative 
phase boundary structure is similar.

%------------------------------------------------------------------------------
\subsection{Usage as a Reusable Diagnostic Tool}
\label{sec:diagnostic_tool}

\subsubsection{Input and output specifications}

The diagnostic framework provided by this work has the following input/output 
specifications:

\paragraph{Input:}
\begin{itemize}
  \item Topology: $\mathcal{M}_3 \in \{\Sthree, \Tthree, \Nilthree\}$ 
        (or specified by structure constants $C^{i}{}_{jk}$)
  \item Parameters: $(V, \eta, \theta_{\NY})$
  \item NY variant: FULL / TT / REE
\end{itemize}

\paragraph{Output:}
\begin{itemize}
  \item Type classification: I (stable with barrier) / II (rolling) / 
        III (unstable)
  \item Stable radius: $r_0$ (for Type I/II)
  \item Barrier height: $\Delta V$ (for Type I)
  \item Effective potential: Analytical expression for $\Veff(r)$
\end{itemize}

\subsubsection{Application example: Use for parameter design}

When constructing cosmological models based on EC+NY theory, this diagnostic 
framework can be used for the following purposes:

\paragraph{Identification of stable regions:}
For a given topology and NY variant, read the parameter range realizing 
Type I or Type II from phase diagrams.

\paragraph{Prediction of critical conditions:}
Use the analytical boundary conditions derived in Sec.~\ref{sec:mechanisms} 
(e.g., $\eta = -2, -6$ for $\Sthree$) to predict parameters where Type 
transitions occur.

\paragraph{Evaluation of stability quality:}
Quantitatively evaluate the ``robustness'' of metastable states from 
$\log_{10}(\Delta V)$ values. Larger $\Delta V$ means stronger suppression 
of decay by quantum tunneling.

\subsubsection{Extension to other topologies}

This diagnostic framework can be extended to any 3-dimensional Lie group 
admitting left-invariant coframes. Specifically:

\begin{enumerate}
  \item Specify structure constants $C^{i}{}_{jk}$
  \item Execute the computational pipeline from Sec.~\ref{sec:pipeline}
  \item Derive effective potential $\Veff(r)$
  \item Compute Type classification and stability metrics
\end{enumerate}

Among Thurston's eight geometries, those with compact quotients 
($\Sthree$, $\mathbb{E}^3$, $\mathrm{Nil}$, $\mathrm{Sol}$, 
$\widetilde{SL_2(\mathbb{R})}$, $\mathbb{H}^2 \times \mathbb{R}$, 
$S^2 \times \mathbb{R}$) can in principle be treated with this framework. 
However, $S^2 \times \mathbb{R}$ is not parallelizable, requiring caution 
in constructing a global coframe.
